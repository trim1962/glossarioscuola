% !TeX TS-program = pdflatex
% !BIB TS-program = biber
% !TeX encoding = UTF-8
% !TeX spellcheck = it_IT
% !TeX root = glossario.tex
\documentclass[a4paper,oneside]{book}%
% 21/01/2018 :: 22:12:46 :: \documentclass{book}%
\usepackage{cmap}
\frenchspacing%
\usepackage{base}
\usepackage[big]{layaureo}
\usepackage{grafica}
%\usepackage{stand_class}
%\usepackage{matematica}
%\usepackage{tabelle}
\usepackage{date}
\usepackage{pagina}
%\usepackage{unita_misura}
\usepackage{copyright}
\usepackage{CDloghi}
\newcommand{\HRule}{\rule{\linewidth}{0.5mm}}

\usepackage[grumpy,mark,markifdirty,raisemark=0.95\paperheight]{gitinfo2}  
\usepackage[toc,page]{appendix}
\renewcommand{\appendixtocname}{Appendici}
\renewcommand{\appendixpagename}{Appendici}
\usepackage[style=italian]{csquotes}
\usepackage[%
style=philosophy-modern,
annotation=true,
hyperref,
backend=biber,
backref]{biblatex}
\addbibresource{biblioscuola.bib}
\usepackage[italian]{varioref}
\usepackage{hyperxmp}
\usepackage[pdfpagelabels,plainpages=false]{hyperref}
\usepackage[italian]{cleveref}
\title{Glossario}
\author{Claudio Duchi}
\date{\datetime}
\hypersetup{%
	pdfencoding=auto,
	urlcolor={blue},
	pdftitle={Nuove parole per la scuola},
	pdfsubject={Glossario},
	pdfstartview={FitH},
	pdfpagemode={UseOutlines},
	pdflicenseurl={http://creativecommons.org/licenses/by-nc-nd/3.0/},
	pdflang={it},
	pdfmetalang={it},
	pdfkeywords={Scuola},
	pdfcopyright={Copyright (C) 2020, Claudio Duchi},
	pdfcontacturl={http://breviariomatematico.altervista.org},
	pdfcontactpostcode={06128},
	pdfcontactphone={},
	pdfcontactemail={claduc},
	pdfcontactcountry={Italy},
	pdfcontactcity={Perugia},
	pdfcontactaddress={},
	pdfcaptionwriter={Claudio Duchi},
	pdfauthortitle={},%
	pdfauthor={Claudio Duchi},
	linkcolor={blue},
	colorlinks=true,
	citecolor={red},
	breaklinks,
	bookmarksopen,
	verbose,
	baseurl={http://breviariomatematico.altervista.org}
}
\usepackage[acronyms,translate=false,nonumberlist,toc,numberedsection,counter=chapter,automake]{glossaries-extra}
\usepackage{glossary-mcols}
\usepackage{glossary-longragged}
\usepackage{glossaries-babel}
\makeglossaries
\renewcommand*{\glsxtrpostdescgeneral}{%
	\ifglshasfield{see}{\glscurrententrylabel}
	{, \glsxtrusesee{\glscurrententrylabel}}%
	{}%
}
\setglossarystyle{altlistgroup}
\loadglsentries{glossari1}
\listfiles
\makeatletter
\renewcommand\frontmatter{%
	\cleardoublepage
	\@mainmatterfalse
	\pagenumbering{arabic}}
\renewcommand\mainmatter{%
	\cleardoublepage
	\@mainmattertrue}
\makeatother
\begin{document}

	
	\frontmatter
		\hypersetup{pageanchor=false}
		\begin{titlepage}\parindent=0pt
			\centering
	\begin{center}
	\Lgrandedue\\[1cm]
	\textsc{\LARGE Claudio Duchi}\\[1.2cm]
	\HRule \\[0.4cm]
	{ \huge \bfseries Nuove parole per la scuola}\\[0.4cm]
	\HRule \\[1.2cm]
	\vfill
	\polylogo[5.5]{15}		
	{\large $-$\DTMnow$-$}	
\end{center}
{\centering
	Release:\gitReln\ (\gitAbbrevHash)\ Autore:\gitAuthorName\ 
	\gitCommitterDate \\
}
		\end{titlepage}
	\setcounter{page}{2}
		\CDcopyright
		\tableofcontents
			%\chapter{Introduzione}
\renewcommand{\abstractname}{Introduzione}
\addcontentsline{toc}{chapter}{Introduzione}
\begin{abstract}
Quando ho cominciato a scrivere queste note non pensavo alla necessità di un'introduzione, poi andando avanti,con il pensiero, mi sono visto nel 1968 che varcavo la porta della scuola dove avrei fatto la mia prima elementare. Tante cose sono cambiate, era il 31 ottobre, san Remigio e nel pomeriggio, avrei visto in TV la festa dei Remigini.\par  Oggi non troverei più la mia scuola che ha ormai cambiato nome in scuola primaria, non esiste più la festa dei Remigini perché ora la scuola inizia a settembre con un calendario deciso dalla Regione in cui vivo.\par  Tante cose sono cambiate, in bene o in peggio non so, molti di questi cambiamenti sono strutturali e difficilmente  etichettatili a questa o a quella forza politica.\par  Sono cambiamenti imposti dal mercato, brutta parola, o dalla Comunità Europea che per omogenizzare ha imposto le sue direttive e raccomandazioni, o dalla necessità d'inseguire  miglioramenti statistici a volte discutibili o anche dal bisogno di un adeguamento ai tempi che cambiano.\par  Quello che osservo, è la nascita un nuovo linguaggio, una neo-lingua  sostituisce la vecchia visione del mondo e le vecchie abitudini mentali e speriamo non renda impossibile ogni altra forma di pensiero.\par  Ascoltando, tornano in mente i rosari che sentivo in gioventù, con cui, in un improbabile latino, le vecchie della parrocchia pregavano, non comprendendo probabilmente, le parole che usavano.\par  Questo va evitato. Bisogna combattere contro la formazione di una  neo-lingua, che rischia di diventare una cesura netta con la società, isolando la scuola e  chi ci vive.\par  Quello che segue è un glossario in cui ho solo riportato le definizioni trovate nelle leggi, decreti, raccomandazioni prodotte negli anni e che sono riuscito a reperire. 
\end{abstract}
\section*{Avvertenza}
I testi  di seguito riportati non hanno valore ufficiale ma solo informativo. L'unico testo valido e definitivo è quello pubblicato dalle fonti ufficiali.


	\mainmatter
\cleardoublepage
\glsaddall	
%\twocolumn
\printglossaries
%\onecolumn		
\nocite{*}
 \addcontentsline{toc}{chapter}{Bibliografia italiana}
\printbibliography[keyword=LEX,title={Bibliografia italiana}]
 \addcontentsline{toc}{chapter}{Bibliografia europea}
\printbibliography[keyword=EU,title={Bibliografia europea}]
 \addcontentsline{toc}{chapter}{Sitografia}
\printbibliography[keyword=WWW,title={Sitografia}]
\backmatter
\appendix
\chapter{Mezzi usati}
\CDMezziUsati
\end{document}
