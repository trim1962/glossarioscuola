% !TeX TS-program = pdflatex
% !BIB TS-program = biber
% !TeX encoding = UTF-8
% !TeX spellcheck = it_IT
% !TeX root = glossario.tex
\documentclass[a4paper]{book}%
% 21/01/2018 :: 22:12:46 :: \documentclass{book}%
\usepackage{cmap}
\frenchspacing%
\usepackage{base}
\usepackage[big]{layaureo}
\usepackage{grafica}
%\usepackage{stand_class}
%\usepackage{matematica}
\usepackage{indice}
\usepackage{tabelle}
\usepackage{adjustbox}
\usepackage{notoccite}
\usepackage{date}
\usepackage{pagina}
%\usepackage{unita_misura}
\usepackage{copyright}
\usepackage{CDloghi}
\usepackage{imakeidx}
\makeindex[options=-s ../Mod_base/oldclaudio.sti]
\makeindex[name=due,title=Anno]
\indexsetup{level=\section*,toclevel=section,noclearpage}
\newcommand{\mioindex}[2]{\index{#2!#1}\index[due]{#1!#2}}
\newcommand{\HRule}{\rule{\linewidth}{0.5mm}}

\usepackage[grumpy,mark,markifdirty,raisemark=0.95\paperheight]{gitinfo2}  
%\usepackage[toc,page]{appendix}
%\renewcommand{\appendixtocname}{Appendici}
%\renewcommand{\appendixpagename}{Appendici}

\usepackage[style=italian]{csquotes}
\usepackage[%
style=philosophy-modern,
annotation=true,
hyperref,
backend=biber,
backref]{biblatex}
\addbibresource{biblioscuola.bib}

\usepackage[italian]{varioref}
\usepackage{hyperxmp}
\usepackage[pdfpagelabels,plainpages=false]{hyperref}
\usepackage[italian]{cleveref}
\title{Nuove parole per la scuola}
\author{Claudio Duchi}
\date{\datetime}
\hypersetup{%
	pdfencoding=auto,
	urlcolor={blue},
	pdftitle={Nuove parole per la scuola},
	pdfsubject={Glossario},
	pdfstartview={FitH},
	pdfpagemode={UseOutlines},
	%pdflicenseurl={http://creativecommons.org/licenses/by-nc-nd/3.0/},
	pdflang={it},
	pdfmetalang={it},
	pdfkeywords={Scuola},
	pdfcopyright={Copyright (C) 2020, Claudio Duchi},
	pdfcontacturl={http://breviariomatematico.altervista.org},
	pdfcontactpostcode={06128},
	pdfcontactphone={},
	pdfcontactemail={claduc},
	pdfcontactcountry={Italy},
	pdfcontactcity={Perugia},
	pdfcontactaddress={},
	pdfcaptionwriter={Claudio Duchi},
	pdfauthortitle={},%
	pdfauthor={Claudio Duchi},
	linkcolor={blue},
	colorlinks=true,
	citecolor={red},
	breaklinks,
	bookmarksopen,
	verbose,
	baseurl={http://breviariomatematico.altervista.org}
}
\usepackage[acronyms,translate=false,nonumberlist,toc,numberedsection,counter=chapter,automake]{glossaries-extra}
\usepackage{glossary-mcols}
\usepackage{glossary-longragged}
\usepackage{glossaries-babel}
\makeglossaries
%\renewcommand*{\glsxtrpostdescgeneral}{%
%	\ifglshasfield{see}{\glscurrententrylabel}
%	{, \glsxtrusesee{\glscurrententrylabel}}%
%	{}%
%}

\setglossarystyle{altlistgroup}
\loadglsentries{glossari3}
\loadglsentries{acronym}

\listfiles
\makeatletter
\renewcommand\frontmatter{%
	\cleardoublepage
	\@mainmatterfalse
	\pagenumbering{arabic}}
\renewcommand\mainmatter{%
	\cleardoublepage
	\@mainmattertrue}
\makeatother
\newenvironment{abstract}%
{\cleardoublepage%
	\thispagestyle{empty}%
	\null \vfill\begin{center}%
		\bfseries \abstractname \end{center}}%
{\vfill\null}
\includeonly{tabelle,regUE-9102014}
\usepackage{multirow, makecell}
%\usepackage{notoccite}
\usepackage{trajan}
\listfiles
\begin{document}
	\pagestyle{fancy}
	\fncyfront
	\frontmatter
		\hypersetup{pageanchor=false}
		\begin{titlepage}\parindent=0pt
			\centering
	\begin{center}
	\Lgrandedue\\[1cm]
	\textsc{\trjnfamily\LARGE CLAUDIO DUCHI}\\[1.2cm]
	\HRule \\[0.4cm]
	{ \trjnfamily\huge \bfseries NUOVE PAROLE PER LA SCUOLA}\\[0.4cm]
	\HRule \\[1.2cm]
	\vfill
	\polylogo[5.5]{17}		
	{\large $-$\DTMnow$-$}	
\end{center}
{\centering
	Release:\gitReln\ (\gitAbbrevHash)\ Autore:\gitAuthorName\ 
	\gitCommitterDate \\
}
		\end{titlepage}
	\setcounter{page}{2}
		\CDcopyright
	\microtypesetup{protrusion = false }  % disabilita la sporgenza localmente nel documento 	
			%	\phantomsection
		\tableofcontents
	%	\phantomsection
			\listoftables
\microtypesetup{protrusion = true}  % abilita la sporgenza			
			%\phantomsection
			%\chapter{Introduzione}
\renewcommand{\abstractname}{Introduzione}
\addcontentsline{toc}{chapter}{Introduzione}
\begin{abstract}
Quando ho cominciato a scrivere queste note non pensavo alla necessità di un'introduzione, poi andando avanti,con il pensiero, mi sono visto nel 1968 che varcavo la porta della scuola dove avrei fatto la mia prima elementare. Tante cose sono cambiate, era il 31 ottobre, san Remigio e nel pomeriggio, avrei visto in TV la festa dei Remigini.\par  Oggi non troverei più la mia scuola che ha ormai cambiato nome in scuola primaria, non esiste più la festa dei Remigini perché ora la scuola inizia a settembre con un calendario deciso dalla Regione in cui vivo.\par  Tante cose sono cambiate, in bene o in peggio non so, molti di questi cambiamenti sono strutturali e difficilmente  etichettatili a questa o a quella forza politica.\par  Sono cambiamenti imposti dal mercato, brutta parola, o dalla Comunità Europea che per omogenizzare ha imposto le sue direttive e raccomandazioni, o dalla necessità d'inseguire  miglioramenti statistici a volte discutibili o anche dal bisogno di un adeguamento ai tempi che cambiano.\par  Quello che osservo, è la nascita un nuovo linguaggio, una neo-lingua  sostituisce la vecchia visione del mondo e le vecchie abitudini mentali e speriamo non renda impossibile ogni altra forma di pensiero.\par  Ascoltando, tornano in mente i rosari che sentivo in gioventù, con cui, in un improbabile latino, le vecchie della parrocchia pregavano, non comprendendo probabilmente, le parole che usavano.\par  Questo va evitato. Bisogna combattere contro la formazione di una  neo-lingua, che rischia di diventare una cesura netta con la società, isolando la scuola e  chi ci vive.\par  Quello che segue è un glossario in cui ho solo riportato le definizioni trovate nelle leggi, decreti, raccomandazioni prodotte negli anni e che sono riuscito a reperire. 
\end{abstract}
\section*{Avvertenza}
I testi  di seguito riportati non hanno valore ufficiale ma solo informativo. L'unico testo valido e definitivo è quello pubblicato dalle fonti ufficiali.


	\fncymain
	\phantomsection
	\mainmatter
\cleardoublepage
\glsaddall	
%\twocolumn
\printglossaries
%\onecolumn	
\chapter{Scuola}
\begin{adjustbox}{max width=\textwidth}
\begin{tabular}{m{4cm}p{12.0cm}}
\toprule
% \multicolumn{2}{c}{Raccomandazione del Consiglio Europeo relativa alle competenze chiave per
% 	l'apprendimento permanente}\\
% \midrule
\multirowcell{14}{Competenza personale,\\sociale e capacità\\ di imparare\\ a imparare}& Capacità di riflettere su se stessi e individuare le proprie attitudini\\
	&Capacità di gestire efficacemente il tempo e le informazioni  \\
	& Capacità di imparare e di lavorare sia in modalità collaborativa sia in maniera
	autonoma\\
	&  Capacità di lavorare con gli altri in maniera costruttiva\\
	&  Capacità di comunicare costruttivamente in ambienti diversi\\
	&  Capacità di creare fiducia e provare empatia\\
	&  Capacità di esprimere e comprendere punti di vista diversi\\
	&  Capacità di negoziare\\
	&  Capacità di concentrarsi, di riflettere criticamente e di prendere decisioni\\
	&  Capacità di gestire il proprio apprendimento e la propria carriera\\
	&  Capacità di gestire l'incertezza, la complessità e lo stress\\
	&  Capacità di mantenersi resilienti\\
	&  Capacità di favorire il proprio benessere fisico ed emotivo\\
\midrule
\multirowcell{1}{Competenze in materia\\di cittadinanza
}	& Capacità di impegnarsi efficacemente con gli altri per un interesse comune o
Pubblico \\
	& Capacità di pensiero critico e abilità integrate nella soluzione dei problemi \\
\midrule
	\multirowcell{12}{Competenza\\imprenditoriale}& Creatività e immaginazione\\
&Capacità di pensiero strategico e risoluzione dei problemi\\
& Capacità di trasformare le idee in azioni\\
& Capacità di riflessione critica e costruttiva \\
	&Capacità di assumere l'iniziativa\\
	& Capacità di lavorare sia in modalità collaborativa in gruppo sia in maniera autonoma\\
	& Capacità di mantenere il ritmo dell'attività\\
	& Capacità di comunicare e negoziare efficacemente con gli altri\\
	& Capacità di gestire l'incertezza, l'ambiguità e il rischio\\
	& Capacità di possedere spirito di iniziativa e autoconsapevolezza\\
	& Capacità di essere proattivi e lungimiranti\\
	& Capacità di coraggio e perseveranza nel raggiungimento degli obiettivi\\
	& Capacità di motivare gli altri e valorizzare le loro idee, di provare empatia\\
	& Capacità di accettare la responsabilità  \\
	\midrule
	\multirowcell{3}{Competenza in materia\\ di
		consapevolezza ed\\ espressione culturali
}& Capacità di esprimere esperienze ed emozioni con empatia\\
	& Capacità di riconoscere e realizzare le opportunità di valorizzazione personale,
sociale o commerciale mediante le arti e le atre forme culturali\\
& Capacità di impegnarsi in processi creativi sia individualmente che collettivamente\\
& Curiosità nei confronti del mondo, apertura per immaginare nuove possibilità\\
\bottomrule
\end{tabular}
\end{adjustbox}
\captionof{table}{Raccomandazione del Consiglio Europeo relativa alle competenze chiave per l'apprendimento permanente}
\begin{sidewaystable}
\centering
\begin{adjustbox}{max width=\textwidth}
\begin{tabular}{@{}>{\centering\arraybackslash}p{1.0cm}p{6cm}p{12cm}p{12cm}}
	\toprule
\multicolumn{1}{c}{Livello}	&\multicolumn{1}{c}{Conoscenze}  & \multicolumn{1}{c}{Abilità} & \multicolumn{1}{c}{Autonomia e Responsabilità} \\
	\midrule
1& Conoscenze concrete, di base, di limitata ampiezza, finalizzate ad eseguire un compito semplice in contesti noti e struttura &Applicare saperi, materiali e strumenti per svolgere un compito semplice, coinvolgendo abilità cognitive, relazionali e sociali di base.Tipicamente: CONCENTRAZIONE e INTERAZIONE   & Svolgere il compito assegnato nel rispetto dei parametri previsti, sotto diretta supervisione nello svolgimento delle attività, in un contesto strutturato. \\
\midrule
2	&Conoscenze concrete,di base, di moderata ampiezza,finalizzate ad eseguire compiti semplici in sequenze diversificate. & Applicare saperi, materiali e strumenti per svolgere compiti semplici insequenze diversificate, coinvolgendo abilità cognitive,relazionali e sociali necessarie per svolgere compiti semplici all'interno di una gamma definita di variabili di contesto.Tipicamente: MEMORIA e PARTECIPAZIONE & Eseguire i compiti assegnati secondo criteri prestabiliti, assicurando la conformità delle attività svolte, sotto supervisione per il conseguimento del risultato, in un contesto strutturato, con un numero limitato di situazioni diversificate \\
	\midrule
3	& Gamma di conoscenze, prevalentemente concrete, con elementi concettuali finalizzati a creare collegamenti logici.Capacità interpretativa. & Utilizzare anche attraverso adattamenti, riformulazioni e rielaborazioni una gamma di saperi, metodi, materiali e strumenti per raggiungere i risultati previsti, attivando un set di abilità cognitive, relazionali, sociali e di attivazione che facilitano l'adattamento nelle situazioni mutevoli.Tipicamente: COGNIZIONE, COLLABORAZIONE e ORIENTAMENTO AL RISULTATO  & Raggiungere i risultati previsti assicurandone la conformità e individuando le modalità di realizzazione più adeguate, in un contesto strutturato, con situazioni mutevoli che richiedono una modifica del proprio operato. \\
	\midrule
4	& Ampia gamma di conoscenze, integrate dal punto di vista della dimensione fattuale e/o concettuale, approfondite in alcune aree.Capacità interpretativa. & Utilizzare anche attraverso adattamenti, riformulazioni e rielaborazioni una gamma di saperi, metodi, prassi e protocolli, materiali e strumenti, per risolvere problemi, attivando un set di abilità cognitive, relazionali, sociali e di attivazione necessarie per superare difficoltà crescenti.Tipicamente: PROBLEM SOLVING, COOPERAZIONE e MULTITASKING & Provvedere al conseguimento degli obiettivi, coordinando e integrando le attività e i risultati anche di altri, partecipando al processo decisionale e attuativo, in un contesto di norma prevedibile, soggetto a cambiamenti imprevisti. \\
	\midrule
5	&Conoscenze integrate, complete, approfondite e specializzate.Consapevolezza degli ambiti di conoscenza.&Utilizzare anche attraverso adattamenti, riformulazioni e rielaborazioni un'ampia gamma di metodi, prassi, protocolli e strumenti, in modo consapevole e selettivo anche al fine di modificarli, attivando un set esauriente di abilità cognitive, relazionali, sociali e di attivazione che consentono di trovare soluzioni tecniche anche non convenzionali.Tipicamente: ANALISI E VALUTAZIONE, COMUNICAZIONE EFFICACE RISPETTO ALL'AMBITO TECNICO e GESTIONE DI CRITICITÀ&Garantire la conformità degli obiettivi conseguiti in proprio e da altre risorse, identificando e programmando interventi di revisione e sviluppo, identificando le decisioni e concorrendo al processo attuativo, in un contesto determinato, complesso ed esposto a cambiamenti ricorrenti e imprevisti.\\
\midrule
6&Conoscenze integrate, avanzate in un ambito, trasferibili da un contesto ad un altro.Consapevolezza critica di teorie e principi in un ambito&Trasferire in contesti diversi i metodi, le prassi e i protocolli necessari per risolvere problemi complessi e imprevedibili, mobilitando abilità cognitive, relazionali, sociali e di attivazione avanzate, necessarie per portare a sintesi operativa le istanze di revisione e quelle di indirizzo, attraverso soluzioni innovative e originali.Tipicamente: VISIONE DI SINTESI, CAPACITA' DI NEGOZIARE E MOTIVARE e PROGETTAZIONE &Presidiare gli obiettivi e i processi di persone e gruppi, favorendo la gestione corrente e la stabilità delle condizioni, decidendo in modo autonomo e negoziando obiettivi e modalità di attuazione, in un contesto non determinato, esposto a cambiamenti imprevedibili.\\
\midrule
7&Conoscenze integrate, altamente specializzate, alcune delle quali all'avanguardia in un ambito.Consapevolezza critica di teorie e principi in più ambiti di conoscenza&Integrare e trasformare saperi, metodi, prassi e protocolli, mobilitando abilità cognitive, relazionali, sociali e di attivazione specializzate, necessarie per indirizzare scenari di sviluppo, ideare e attuare nuove attività e procedure.Tipicamente: VISIONE SISTEMICA, LEADERSHIP, GESTIONE DI RETI RELAZIONALI E INTERAZIONI SOCIALI COMPLESSE e PIANIFICAZIONE &Governare i processi di integrazione e trasformazione, elaborando le strategie di attuazione e indirizzando lo sviluppo dei risultati e delle risorse, decidendo in modo indipendente e indirizzando obiettivi e modalità di attuazione, in un contesto non determinato, esposto a cambiamenti continui, di norma confrontabili rispetto a variabili note, soggetto ad innovazione\\
\midrule
8&Conoscenze integrate, esperte e all'avanguardia in un ambito e nelle aree comuni ad ambiti diversi.Consapevolezza critica di teorie e principi in più ambiti di conoscenza&Concepire nuovi saperi, metodi, prassi e protocolli, mobilitando abilità cognitive, relazionali, sociali e di attivazione esperte, necessarie a intercettare e rispondere alla domanda di innovazione.Tipicamente: VISIONE STRATEGICA, CREATIVITÀ e CAPACITÀ DI PROIEZIONE ED EVOLUZIONE&Promuovere processi di innovazione e sviluppo strategico, prefigurando scenari e soluzioni e valutandone i possibili effetti, in un contesto di avanguardia non confrontabile con situazioni e contesti precedenti.\\
\bottomrule
\end{tabular}
\end{adjustbox}
\caption[Quadro Nazionale delle Qualificazioni NQF Italia]{Quadro Nazionale delle Qualificazioni NQF Italia~\cite{DL2018}}
\end{sidewaystable}

\begin{sidewaystable}
	\centering
	\begin{adjustbox}{max width=\textwidth}
		\begin{tabular}{@{}>{\centering\arraybackslash}p{1.0cm}p{6cm}p{12cm}p{12cm}}
			\toprule
			\multicolumn{1}{c}{Livello}	&\multicolumn{1}{c}{Conoscenze}  & \multicolumn{1}{c}{Abilità} & \multicolumn{1}{c}{Responsabilità e autonomia} \\
			\midrule
		1&Nel contesto dell'EQF, le conoscenze sono descritte come teoriche e/o pratiche.&Nel contesto dell'EQF, le abilità sono descritte come cognitive (comprendenti l'uso del pensiero logico, intuitivo e creativo) e pratiche (comprendenti la manualità e l'uso di metodi, materiali, strumenti e utensili).&Nel contesto dell'EQF, la responsabilità e l'autonomia sono descritte come la capacità del discente di applicare le conoscenze e le abilità in modo autonomo e responsabile.\\
		\midrule
		2&Conoscenze generali di base&Abilità di base necessarie a svolgere compiti semplici&Lavoro o studio, sotto supervisione diretta, in un contesto strutturato\\
		\midrule
		3&Conoscenze pratiche di base in un ambito di lavoro o di studio&Abilità cognitive e pratiche di base necessarie all'uso di informazioni pertinenti per svolgere compiti e risolvere problemi ricorrenti usando strumenti e regole semplici&Lavoro o studio, sotto supervisione, con un certo grado di autonomia\\
		\midrule
		3&Conoscenza di fatti, principi, processi e concetti generali, in un ambito di lavoro o di studio&Una gamma di abilità cognitive e pratiche necessarie a svolgere compiti e risolvere problemi scegliendo e applicando metodi di base, strumenti, materiali ed informazioni&Assumere la responsabilità di portare a termine compiti nell'ambito del lavoro o dello studio
		
		Adeguare il proprio comportamento alle circostanze nella soluzione dei problemi\\
		\midrule
		4&Conoscenze pratiche e teoriche in ampi contesti in un ambito di lavoro o di studio&Una gamma di abilità cognitive e pratiche necessarie a risolvere problemi specifici in un ambito di lavoro o di studio&Sapersi gestire autonomamente, nel quadro di istruzioni in un contesto di lavoro o di studio, di solito prevedibili ma soggetti a cambiamenti
		
		Sorvegliare il lavoro di routine di altri, assumendo una certa responsabilità per la valutazione e il miglioramento di attività lavorative o di studio\\
		\midrule
		5&Conoscenze pratiche e teoriche esaurienti e specializzate, in un ambito di lavoro o di studio, e consapevolezza dei limiti di tali conoscenze&Una gamma esauriente di abilità cognitive e pratiche necessarie a dare soluzioni creative a problemi astratti&Saper gestire e sorvegliare attività nel contesto di attività lavorative o di studio esposte a cambiamenti imprevedibili
		
		Esaminare e sviluppare le prestazioni proprie e di altri\\
		\midrule
		6&Conoscenze avanzate in un ambito di lavoro o di studio, che presuppongono una comprensione critica di teorie e principi&Abilità avanzate, che dimostrino padronanza e innovazione necessarie a risolvere problemi complessi ed imprevedibili in un ambito specializzato di lavoro o di studio&Gestire attività o progetti tecnico/professionali complessi assumendo la responsabilità di decisioni in contesti di lavoro o di studio imprevedibili.

		Assumere la responsabilità di gestire lo sviluppo professionale di persone e gruppi\\
		\midrule
		7&Conoscenze altamente specializzate, parte delle quali all'avanguardia in un ambito di lavoro o di studio, come base del pensiero originale e/o della ricerca
		
		Consapevolezza critica di questioni legate alla conoscenza in un ambito e all'intersezione tra ambiti diversi&Abilità specializzate, orientate alla soluzione di problemi, necessarie nella ricerca e/o nell'innovazione al fine di sviluppare conoscenze e procedure nuove e integrare le conoscenze ottenute in ambiti diversi& Gestire e trasformare contesti di lavoro o di studio complessi, imprevedibili e che richiedono nuovi approcci strategici
		
		Assumere la responsabilità di contribuire alla conoscenza e alla pratica professionale e/o di verificare le prestazioni strategiche dei gruppi\\
		\midrule
		8&Le conoscenze più all'avanguardia in un ambito di lavoro o di studio e all'intersezione tra ambiti diversi&Le abilità e le tecniche più avanzate e specializzate, comprese le capacità di sintesi e di valutazione, necessarie a risolvere problemi complessi della ricerca e/o dell'innovazione e ad estendere e ridefinire le conoscenze o le pratiche professionali esistenti.&Dimostrare effettiva autorità, capacità di innovazione, autonomia, integrità tipica dello studioso e del professionista e impegno continuo nello sviluppo di nuove idee o processi all'avanguardia in contesti di lavoro, di studio e di ricerca.\\
		
			\bottomrule
		\end{tabular}
	\end{adjustbox}
	\caption[Descrittori che definiscono i livelli del quadro europeo delle qualifiche (EQF)]{Descrittori che definiscono i livelli del quadro europeo delle qualifiche (EQF)~\cite{RA2017}}
\end{sidewaystable}
\begin{center}
	\begin{tabular}{cl}
\toprule
Codice&Decodifica\\
\midrule
AA&Scuola Materna\\
EE&Scuola Elementare\\
IC&Istituto Comprensivo\\
IS&Istituto di Istruzione Secondaria Superiore\\
MM&Scuola Media\\
PC& Liceo Classico\\
PL&Liceo Linguistico\\
PM&Istituto Magistrale\\
PQ&Scuola Magistrale\\
PS& Liceo Scientifico\\
RA&Ist.Prof. per l'agricoltura\\
RB&Scuola Tecnica per l'arte Bianca\\
RC&Ist.Prof.Commerciale\\
RE&Ist.Prof.Ind. e Art.per Ciechi\\
RF&Ist.Prof.Femminile\\
RH&Ist.Prof.Alberghiero\\
RI&Ist.Prof.Industria e Artigianato\\
RM&Ist.Prof.Ind.e Attività Marinare\\
RN&Ist.Prof.per l'alimentazione\\
RS&Ist.Prof.Ind.e Art.per Sordomuti\\
RT&Ist.Prof.per l'industria Edile\\
RV&Ist.Prof.Cinematografia e Televisione\\
SD&Istituto d'arte\\
SL&Liceo Artistico\\
SM&Accademia di Belle Arti\\
SN&Accademia Nazionale di Danza\\
SR&Accademia Nazionale d'arte Drammatica\\
ST&Conservatorio di Musica\\
TA&Istituto Tecnico Agrario\\
TB&Istituto Tecnico Aeronautico\\
TD&Istituto Tecnico Commerciale e per Geometri\\
TD&Istituto Tecnico Commerciale\\
TE&Istituto Tecnico Femminile\\
TF&Istituto Tecnico Industriale\\
TH&Istituto Tecnico Nautico\\
TL&Istituto Tecnico per Geometri\\
TN&Istituto Tecnico per Il Turismo\\
VC&Convitto Nazionale\\
VE&Educandato\\
\bottomrule
\end{tabular}
\captionof{table}{Codifica scuole}
\end{center}
\chapter{Regolamento UE n. 910/2014}
\section{Allegato I}\label{sec:allegatoIreg9102014}
Requisiti per i certificati qualificati di firma elettronica\par I certificati qualificati di firma elettronica contengono:
\begin{enumerate}
	\item 	un’indicazione, almeno in una forma adatta al trattamento automatizzato, del fatto che il certificato è stato rilasciato quale certificato qualificato di firma elettronica;
	\item un insieme di dati che rappresenta in modo univoco il prestatore di servizi fiduciari qualificato che rilascia i certificati qualificati e include almeno lo Stato membro in cui tale prestatore è stabilito e
	\begin{enumerate}
		\item per una persona giuridica: il nome e, se del caso, il numero di registrazione quali figurano nei documenti ufficiali,
		\item per una persona fisica: il nome della persona;
	\end{enumerate}
\item è chiaramente indicato almeno il nome del firmatario, o uno pseudonimo, qualora sia usato uno pseudonimo;
\item i dati di convalida della firma elettronica che corrispondono ai dati per la creazione di una firma elettronica;
\item l’indicazione dell’inizio e della fine del periodo di validità del certificato;
\item il codice di identità del certificato che deve essere unico per il prestatore di servizi fiduciari qualificato;
\item 	
la firma elettronica avanzata o il sigillo elettronico avanzato del prestatore di servizi fiduciari qualificato che rilascia il certificato;
\item il luogo in cui il certificato relativo alla firma elettronica avanzata o al sigillo elettronico avanzato di cui al numero 7) è disponibile gratuitamente;
\item 	
l’ubicazione dei servizi a cui ci si può rivolgere per informarsi sulla validità del certificato qualificato;
\item qualora i dati per la creazione di una firma elettronica connessi ai dati di convalida della firma elettronica siano ubicati in un dispositivo per la creazione di una firma elettronica qualificata, un’indicazione appropriata di questo fatto, almeno in una forma adatta al trattamento automatizzato.
\end{enumerate}
\section{Allegato II}\label{sec:allegatoIIreg9102014}
Requisiti per i dispositivi per la creazione di una firma elettronica qualificata
\begin{enumerate}
	\item I dispositivi per la creazione di una firma elettronica qualificata garantiscono, mediante mezzi tecnici e procedurali appropriati, almeno quanto segue:
	\begin{enumerate}
		\item 	
		è ragionevolmente assicurata la riservatezza dei dati per la creazione di una firma elettronica utilizzati per creare una firma elettronica;
		\item 	
		i dati per la creazione di una firma elettronica utilizzati per creare una firma elettronica possono comparire in pratica una sola volta;
		\item 	
		i dati per la creazione di una firma elettronica utilizzati per creare una firma elettronica non possono, con un grado ragionevole di sicurezza, essere derivati e la firma elettronica è attendibilmente protetta da contraffazioni compiute con l’impiego di tecnologie attualmente disponibili;
		\item i dati per la creazione di una firma elettronica utilizzati nella creazione della stessa possono essere attendibilmente protetti dal firmatario legittimo contro l’uso da parte di terzi.
	\end{enumerate}
\item I dispositivi per la creazione di una firma elettronica qualificata non alterano i dati da firmare né impediscono che tali dati siano presentati al firmatario prima della firma.
\item La generazione o la gestione dei dati per la creazione di una firma elettronica per conto del firmatario può essere effettuata solo da un prestatore di servizi fiduciari qualificato.
\item Fatto salvo il punto 1, lettera d), i prestatori di servizi fiduciari qualificati che gestiscono dati per la creazione di una firma elettronica per conto del firmatario possono duplicare i dati per la creazione di una firma elettronica solo a fini di back-up, purché rispettino i seguenti requisiti:
\begin{enumerate}
	\item 	
	la sicurezza degli insiemi di dati duplicati deve essere dello stesso livello della sicurezza degli insiemi di dati originali;
	\item 	
	il numero di insiemi di dati duplicati non eccede il minimo necessario per garantire la continuità del servizio.
\end{enumerate}
\end{enumerate}
\section{Allegato III}\label{sec:allegatoIIIreg9102014}
Requisiti per i certificati qualificati dei sigilli elettronici\par 
I certificati qualificati dei sigilli elettronici contengono:
\begin{enumerate}
	\item un’indicazione, almeno in una forma adatta al trattamento automatizzato, del fatto che il certificato è stato rilasciato quale certificato qualificato di sigillo elettronico;
	\item 	
	un insieme di dati che rappresenta in modo univoco il prestatore di servizi fiduciari qualificato che rilascia i certificati qualificati e include almeno lo Stato membro in cui tale prestatore è stabilito e
	\begin{enumerate}
		\item per una persona giuridica: il nome e, se del caso, il numero di registrazione quali appaiono nei documenti ufficiali,
		\item per una persona fisica: il nome della persona;
	\end{enumerate}
\item almeno il nome del creatore del sigillo e, se del caso, il numero di registrazione quali appaiono nei documenti ufficiali;
\item 	
i dati di convalida del sigillo elettronico che corrispondono ai dati per la creazione di un sigillo elettronico;
\item l’indicazione dell’inizio e della fine del periodo di validità del certificato;
\item il codice di identità del certificato che deve essere unico per il prestatore di servizi fiduciari qualificato;
\item la firma elettronica avanzata o il sigillo elettronico avanzato del prestatore di servizi fiduciari qualificato che rilascia il certificato;
\item 	
il luogo in cui il certificato relativo alla firma elettronica avanzata o al sigillo elettronico avanzato di cui alla lettera g) è disponibile gratuitamente;
\item 	
l’ubicazione dei servizi a cui ci si può rivolgere per informarsi sulla validità del certificato qualificato;
\item qualora i dati per la creazione di un sigillo elettronico connessi ai dati di convalida del sigillo elettronico siano ubicati in un dispositivo per la creazione di un sigillo elettronico qualificato, un’indicazione appropriata di questo fatto, almeno in una forma adatta al trattamento automatizzato.
\end{enumerate}
\section{Allegato IV}\label{sec:allegatoIVreg9102014}
Requisiti per i certificati qualificati di autenticazione di siti web\par 
I certificati qualificati di autenticazione di siti web contengono:
\begin{enumerate}
	\item un’indicazione, almeno in una forma adatta al trattamento automatizzato, del fatto che il certificato è stato rilasciato quale certificato qualificato di autenticazione di sito web;
	\item un insieme di dati che rappresenta in modo univoco il prestatore di servizi fiduciari qualificato che rilascia i certificati qualificati e include almeno lo Stato membro in cui tale prestatore è stabilito e
	\begin{enumerate}
		\item per una persona giuridica: il nome e, se del caso, il numero di registrazione quali appaiono nei documenti ufficiali,
		\item 	
		per una persona fisica: il nome della persona;
	\end{enumerate}
\item per le persone fisiche: almeno il nome della persona a cui è stato rilasciato il certificato, o uno pseudonimo. Qualora sia usato uno pseudonimo, ciò è chiaramente indicato;

per le persone giuridiche: almeno il nome della persona giuridica cui è stato rilasciato il certificato e, se del caso, il numero di registrazione quali appaiono nei documenti ufficiali;
\item elementi dell’indirizzo, fra cui almeno la città e lo Stato, della persona fisica o giuridica cui è rilasciato il certificato e, se del caso, quali appaiono nei documenti ufficiali;
\item il nome del dominio o dei domini gestiti dalla persona fisica o giuridica cui è rilasciato il certificato;
\item 	
l’indicazione dell’inizio e della fine del periodo di validità del certificato;
\item 	
il codice di identità del certificato che deve essere unico per il prestatore di servizi fiduciari qualificato;
\item la firma elettronica avanzata o il sigillo elettronico avanzato del prestatore di servizi fiduciari qualificato che rilascia il certificato;
\item 	
il luogo in cui il certificato relativo alla firma elettronica avanzata o al sigillo elettronico avanzato di cui al numero 8) è disponibile gratuitamente;
\item 	
l’ubicazione dei servizi competenti per lo status di validità del certificato a cui ci si può rivolgere per informarsi sulla validità dei certificato qualificato.
\end{enumerate}
\section{Articolo 26}\label{sec:articolo26reg9102014}
Requisiti di una firma elettronica avanzata\par 
Una firma elettronica avanzata soddisfa i seguenti requisiti:
\begin{enumerate}
	\item 	
	è connessa unicamente al firmatario;
	\item 	
	è idonea a identificare il firmatario;
	\item è creata mediante dati per la creazione di una firma elettronica che il firmatario può, con un elevato livello di sicurezza, utilizzare sotto il proprio esclusivo controllo; e
	\item è collegata ai dati sottoscritti in modo da consentire l’identificazione di ogni successiva modifica di tali dati.
\end{enumerate}
\section{Articolo 42}\label{sec:articolo42reg9102014}
Requisiti per la validazione temporale elettronica qualificata
\begin{enumerate}
	\item Una validazione temporale elettronica qualificata soddisfa i requisiti seguenti:
	\begin{enumerate}
		\item collega la data e l’ora ai dati in modo da escludere ragionevolmente la possibilità di modifiche non rilevabili dei dati;
		\item si basa su una fonte accurata di misurazione del tempo collegata al tempo universale coordinato; e
		\item è apposta mediante una firma elettronica avanzata o sigillata con un sigillo elettronico avanzato del prestatore di servizi fiduciari qualificato o mediante un metodo equivalente.
	\end{enumerate}
\item La Commissione può, mediante atti di esecuzione, stabilire i numeri di riferimento delle norme applicabili al collegamento della data e dell’ora ai dati e a fonti accurate di misurazione del tempo. Si presume che i requisiti di cui al paragrafo 1 siano stati rispettati ove il collegamento della data e dell’ora ai dati e alla fonte accurata di misurazione del tempo rispondano a dette norme. Tali atti di esecuzione sono adottati secondo la procedura d’esame di cui all’articolo 48, paragrafo 2.
\end{enumerate}
\section{Articolo 44}\label{sec:articolo44reg9102014}
Requisiti per i servizi elettronici di recapito certificato qualificati:
\begin{enumerate}
	\item I servizi elettronici di recapito certificato qualificati soddisfano i requisiti seguenti:
	\begin{enumerate}
		\item sono forniti da uno o più prestatori di servizi fiduciari qualificati;
		\item garantiscono con un elevato livello di sicurezza l’identificazione del mittente;
		\item garantiscono l’identificazione del destinatario prima della trasmissione dei dati;
		\item 	
		l’invio e la ricezione dei dati sono garantiti da una firma elettronica avanzata o da un sigillo elettronico avanzato di un prestatore di servizi fiduciari qualificato in modo da escludere la possibilità di modifiche non rilevabili dei dati;
		\item qualsiasi modifica ai dati necessaria al fine di inviarli o riceverli è chiaramente indicata al mittente e al destinatario dei dati stessi;
		\item la data e l’ora di invio e di ricezione e qualsiasi modifica dei dati sono indicate da una validazione temporale elettronica qualificata.
	\end{enumerate}
Qualora i dati siano trasferiti fra due o più prestatori di servizi fiduciari qualificati, i requisiti di cui alle lettere da a) a f) si applicano a tutti i prestatori di servizi fiduciari qualificati.
\item La Commissione può, mediante atti di esecuzione, stabilire i numeri di riferimento delle norme applicabili ai processi di invio e ricezione dei dati. Si presume che i requisiti di cui al paragrafo 1 siano stati rispettati ove il processo di invio e ricezione dei dati risponda a tali norme. Tali atti di esecuzione sono adottati secondo la procedura d’esame di cui all’articolo 48, paragrafo 2.
6tgtg \end{enumerate}
	
\nocite{*}



\chapter{Bibliografia}
%\addcontentsline{toc}{chapter}{Bibliografia}

 \section{Bibliografia italiana}
 \begin{description}
 	\item[CdS] Parere Consiglio di Stato
 	\item[CIRC] Circolare 
 	\item[DL] Decreto Legge
 	\item[INAIL] Circolare INAIL
 	\item[LEGGE] Legge
 	\item[LG] Linee guida
 	\item[NOTA] Nota ministeriale
 \end{description}
% \addcontentsline{toc}{section}{Bibliografia italiana}
\printbibliography[keyword=LEX,heading=subbibliography]
%\section*{Sentenze}
% \addcontentsline{toc}{section}{Sentenze}
%\printbibliography[keyword=SEN,heading=subbibliography]
 
 \section{Bibliografia europea}
% \addcontentsline{toc}{section}{Bibliografia europea}
 \begin{description}
 	\item[DE] Decisione del Parlamento Europeo e del Consiglio
 	\item[DI] Direttiva del Parlamento Europeo e del Consiglio
 	\item[RA] Raccomandazione Parlamento Europeo e del Consiglio
 \end{description}
\printbibliography[keyword=EU,heading=subbibliography]
\section{Altro}
 %\addcontentsline{toc}{section}{Altro}
\printbibliography[keyword=EXTRA, heading=subbibliography]
\addcontentsline{toc}{section}{Pubblicazioni}
\printbibliography[keyword=BOOK,title={Pubblicazioni}]
\chapter{Sitografia}
% \addcontentsline{toc}{chapter}{Sitografia}
\printbibliography[keyword=WWW,type=online,restoreclassic,annotation=false,heading=subbibliography]

\chapter{Indici}
%\addcontentsline{toc}{chapter}{Indici}
\printindex
\printindex[due]
\backmatter
\appendix

\chapter{Mezzi usati}
\CDMezziUsati

\end{document}
