
%%%\newglossaryentry{Classiaggiornamento}{name={Classi di aggiornamento},description={Nella scuola media e' data facolta' di istituire classi di aggiornamento che si affiancano alla prima e alla terza. Alla prima classe di aggiornamento possono accedere gli alunni bisognosi di particolari cure per frequentare con profitto la prima classe di scuola media. Alla terza classe di aggiornamento possono accedere gli alunni che non abbiano conseguito la licenza di scuola media perche' respinti. Le classi di aggiornamento non possono avere piu' di 15 alunni ciascuna: ad esse vengono destinati insegnanti particolarmente qualificati~\cite[Art.~11]{LEGGE1962}\mioindex{1953}{Classi!aggiornamento}. Abbolite con legge 4 agosto 1977, n. 517~\cite[Art.~10~C.~10]{LEGGE1977}}}
%%\newglossaryentry{Insegnantedisostegno}{name={Insegnante di sostegno},description={Nell'ambito di tali attività la scuola attua forme di integrazione a favore degli alunni portatori di handicaps con la prestazione di insegnanti specializzati assegnati ai sensi dell'articolo 9 del decreto del Presidente della Repubblica 31 ottobre 1975, n. 970, anche se appartenenti a ruoli speciali, o ai sensi del quarto comma dell'articolo 1 della legge 24 settembre 1971, n. 820. Devono inoltre essere assicurati la necessaria integrazione specialistica, il servizio socio-psicopedagogico e forme particolari di sostegno secondo le rispettive, competenze dello Stato e degli enti locali preposti, nei limiti delle relative disponibilità di bilancio e sulla base del programma predisposto dal consiglio scolastico distrettuale.~\cite[Art.~2]{LEGGE1977}. Al fine di agevolare l'attuazione del diritto allo studio e la piena formazione della personalità degli alunni, la programmazione educativa può comprendere attività scolastiche di integrazione anche a carattere interdisciplinare, organizzate per gruppi di alunni della stessa classe o di classi diverse, ed iniziative di sostegno, anche allo scopo di realizzare interventi individualizzati in relazione alle esigenze dei singoli alunni.	Nell'ambito della programmazione di cui al precedente comma sono previste forme di integrazione e di sostegno a favore degli alunni portatori di handicaps da realizzare mediante la utilizzazione dei docenti, di ruolo o incaricati a tempo indeterminato, in servizio nella scuola media e in possesso di particolari titoli di specializzazione, che ne facciano richiesta, entro il limite di una unità per ciascuna classe che accolga alunni portatori di handicaps e nel numero massimo di sei ore settimanali. Le classi che accolgono alunni portatori di handicaps sono costituite con un massimo di 20 alunni. In tali classi devono essere assicurati la necessaria integrazione specialistica, il servizio socio-psico-pedagogico e forme particolari di sostegno secondo le rispettive competenze dello Stato e degli enti locali preposti, nei limiti delle relative disponibilità di bilancio e sulla base del programma predisposto dal consiglio scolastico distrettuale. Le attività di cui al primo comma del presente articolo si svolgono periodicamente in sostituzione delle normali attività didattiche e fino ad un massimo di 160 ore nel corso dell'anno scolastico con particolare riguardo al tempo iniziale e finale del periodo delle lezioni, secondo un programma di iniziative di integrazione e di sostegno che dovrà essere elaborato dal collegio dei docenti sulla base di criteri generali indicati dal consiglio di istituto e delle proposte dei consigli di classe.~\cite[Art.~7]{LEGGE1977}.}}
%%\newglossaryentry{classidifferenziali}{name={Classi differenziali},description={Le classi differenziali non sono istituti scolastici a sé stanti, ma funzionano presso le comuni scuole elementari ed accolgono gli alunni nervosi, tardivi, instabili, i quali rivelano l'inadattabilità alla disciplina comune e ai normali metodi e ritmi d'insegnamento e possono raggiungere un livello migliore solo se l'insegnamento viene ad essi impartito con modi e forme particolari~\cite{CIRC1953}. Abbolite con legge 4 agosto 1977, n. 517~\cite[Art.~10~C.~10]{LEGGE1977}.\mioindex{1953}{Classi!differenziali}}}
%%\newglossaryentry{classispeciali}{name={Classi speciali},description={Le classi speciali per minorati e quelle di differenziazione didattica sono istituti scolastici nei quali viene impartito l'insegnamento elementare ai fanciulli aventi determinate minorazioni fisiche o psichiche ed istituti nei quali vengono adottati speciali metodi didattici per l'insegnamento ai ragazzi anormali, es. scuole Montessori~\cite{CIRC1953}\mioindex{1953}{Classi!speciali}.}}
%%\newglossaryentry{mutilatiinvalidicivili}{name={Mutilati ed invalidi civili},description={Cittadini affetti da minorazioni congenite o acquisite, anche a carattere progressivo, compresi gli irregolari psichici per oligofrenie di carattere organico o dismetabolico, insufficienze mentali derivanti da difetti sensoriali e funzionali che abbiano subito una riduzione permanente della capacità lavorativa non inferiore a un terzo o, se minori di anni 18, che abbiano difficolta' persistenti a svolgere i compiti e le funzioni proprie della loro età. Sono esclusi gli invalidi per cause di guerra, di lavoro, di servizio, nonche' i ciechi e i sordomuti per i quali provvedono altre leggi~\cite[Art.~2]{LEGGE1971}\mioindex{1971}{Mutilati ed invalidi civili}.}}
%\newacronym{AgID}{AgID}{Agenzia per l'Italia digitale}
%\newacronym{BES}{BES}{Bisogni educativi speciali}
%\newacronym{CIE}{CIE}{Carta di identità elettronica}
%\newacronym{DPIA}{DPIA}{Data Protecion Impact Assessment}
%\newacronym{DPO}{DPO}{Data Protection Officer}
%\newacronym{DSA}{DSA}{Disturbi Specifici dell'Apprendimento}
%\newacronym{EDPB}{EDPB}{European Data Protection Board}
%\newacronym{GDPR}{GDPR}{General Data Protection Regulation}
%\newacronym{SPID}{SPID}{Sistema Pubblico di Identità Digitale}
%\newacronym{WP29}{WP29}{Article 29 Working Party}
%\newacronym{eIDAS}{eIDAS}{electronic IDentification Authentication and Signature}
%\newacronym{eID}{eID}{electronic IDentification. identita digitale}
%\newglossaryentry{datipersonali}{name={Dati personali},description={Informazioni riguardanti una persona fisica identificata o identificabile~\cite{DE2018}.}}
%\newglossaryentry{dpog}{name={Data Protection Officer},description={Il DPO è un supervisore indipendente, il quale sarà designato obbligatoriamente, da soggetti apicali di tutte le pubbliche amministrazioni e nello specifico è previsto l'obbligo nel caso in cui “il trattamento è effettuato da un'autorità pubblica o da un organismo pubblico, eccettuate le autorità giurisdizionali quando esercitano le loro funzioni giurisdizionali”}}
%\newglossaryentry{eIDASg}{name={Regolamento eIDAS},description={Regolamento UE n° 910/2014 sull'identità digitale - ha l'obiettivo di fornire una base normativa a livello comunitario per i servizi fiduciari e i mezzi di identificazione elettronica degli stati membri~\cite{RE2014}.}}
%\newglossaryentry{edpbg}{name={Comitato europeo per la protezione dei dati},description={Il comitato europeo per la protezione dei dati è un organo europeo indipendente, che contribuisce all'applicazione coerente delle norme sulla protezione dei dati in tutta l'Unione europea e promuove la cooperazione tra le autorità competenti per la protezione dei dati dell'UE.}}
%\newglossaryentry{gdprg}{name={Regolamento GDPR},description={Regolamento Ue 2016/679, noto come GDPR (General Data Protection Regulation) – relativo alla protezione delle persone fisiche con riguardo al trattamento e alla libera circolazione dei dati personali~\cite{RE2016}.}}
%\newglossaryentry{interoperabilittecnica}{name={Interoperabilità tecnica},description={La capacità dei sistemi di tecnologia dell'informazione e della comunicazione di interagire in modo da consentire la condivisione di informazioni, mediante un accordo fra tutte le parti e i titolari delle informazioni~\cite{DE2018}.}}
%\newglossaryentry{servizidiautenticazione}{name={Servizi di autenticazione},description={processi tecnici, quali le firme elettroniche e l'autenticazione di siti web, che consentono agli utenti di verificare le informazioni, come ad esempio l'identità, attraverso Europass~\cite{DE2018}.}}
%\newglossaryentry{standardaperti}{name={Standard aperti},description={Standard tecnici che sono stati elaborati nell'ambito di un processo collaborativo e sono stati pubblicati per essere utilizzati liberamente da tutti i soggetti interessati; ~\cite{DE2018}.}}
\newacronym{AIR}{AIR}{Analisi di Impatto della Regolazione}
\newacronym{ANPAL}{ANPAL}{Agenzia Nazionale per le Politiche Attive del Lavoro~\cite{DL2015}}
\newacronym{ANVUR}{ANVUR}{Agenzia Nazionale di Valutazione del sistema Universitario e della Ricerca~\cite{LEGGE2006}}
\newacronym{ASR}{ASR}{Accordo Stato Regioni}
\newacronym{ATECO}{ATECO}{ATtività ECOnomica}
\newacronym{ATI}{ATI}{Associazione Temporanea di Imprese}
\newacronym{ATS}{ATS}{Associazione Temporanea di Scopo}
\newacronym{CBT}{CBT}{Computer Based Test}
\newacronym{CCR}{CCR}{Closed-Constructed Response, Aperti con risposta univoca}
\newacronym{CMC}{CMC}{Complex Multiple Choice, Scelta multipla complessa}
\newacronym{CNPI}{CNPI}{Consiglio nazionale della pubblica istruzione}
\newacronym{CR}{CR}{Constructed Response, Aperti con risposta articolata}
\newacronym{CSPI}{CSPI}{Consiglio superiore della pubblica istruzione}
\newacronym{CSR}{CSR}{Conferenza Stato Regioni~\cite{DL1997}}
\newacronym{CS}{CS}{Comitato Scientifico~\cite{LG2019}~\cite{DL2010a}}
\newacronym{CTS}{CTS}{Comitato Tecnico Scientifico~\cite{LG2019}}
\newacronym{CURA}{CURA}{Catalogo Unico Regionale dell'Offerta di Apprendimento~\cite{SIRU}}
\newacronym{CU}{CU}{Conferenza Unificata~\cite{DL1997}}
\newacronym{DGOSV}{DGOSV}{Direzione generale per gli ordinamenti scolastici e la valutazione del sistema nazionale di istruzione~\cite{DL2014}~\cite{DL2014a}}
\newacronym{DM}{DM}{Decreto Ministeriale~\cite{Wikipedia2019a}}
\newacronym{DPCM}{DPCM}{Decreto del presidente del Consiglio dei ministri~\cite{Wikipedia2019a}}
\newacronym{EAfA}{EAfA}{Alleanza europea per l'apprendistato~\cite{LG2019}}
\newacronym{ECTS}{ECTS}{Sistema europeo di trasferimento e accumulo dei crediti~\cite{LG2019}}
\newacronym{ECVET}{ECVET}{Sistema europeo di crediti per la VET~\cite{LG2019}}
\newacronym{EQARF}{EQARF}{Quadro europeo per la garanzia della qualità del sistema di istruzione e formazione professionale~\cite{LG2019}}
\newacronym{EQAVET}{EQAVET}{Quadro europeo di riferimento per la garanzia della qualità dell'istruzione e della formazione professionale~\cite{DL2015a}}
\newacronym{EQF}{EQF}{Quadro europeo delle qualificazioni~\cite{RA2017}}
\newacronym{EntreComp}{EntreComp}{Quadro europeo della competenza imprenditorialità~\cite{LG2019}}
\newacronym{Europass}{Europass}{Passaporto europeo delle competenze~\cite{LG2019},~\cite{DE2004}~\cite{DE2018}}
\newacronym{IFTS}{IFTS}{Istruzione e Formazione Tecnica Superiore~\cite{DL2008}}
\newacronym{ILO}{ILO}{Organizzazione Internazionale del Lavoro}
\newacronym{INAPP}{INAPP}{Istituto Nazionale per l'Analisi delle Politiche Pubbliche~\cite{DL2016}}
\newacronym{INDIRE}{INDIRE}{Istituto nazionale di documentazione per l'innovazione e la ricerca educativa}
\newacronym{INVALSI}{INVALSI}{Istituto nazionale per la valutazione del sistema educativo di istruzione e formazione}
\newacronym{IP}{IP}{Istruzione Professionale}
\newacronym{ISCED}{ISCED}{International Standard Classification of Education~\cite{Wikipedia2020}}
\newacronym{ISCO08}{ISCO08}{International Standard Classification of Occupations, 2008~\cite{ILO2008}}
\newacronym{ISFOL}{ISFOL}{Istituto per lo sviluppo della formazione professionale dei lavoratori~\cite{DL2015}}
\newacronym{ITC}{ITC}{Information and Communications Technology}
\newacronym{ITS}{ITS}{Istituto Tecnico Superiore~\cite{DL2008}}
\newacronym{IVC}{IVC}{Individuazione, validazione e certificazione delle competenze~\cite{TECNOSTRUCTURA1996}}
\newacronym{IeFP}{IeFP}{Istruzione e Formazione Professionale~\cite{LEGGE2003}~\cite{LEGGE2007}}
\newacronym{MLPS}{MLPS}{Ministero del Lavoro e delle Politiche Sociali}
\newacronym{NUP}{NUP}{Nomenclatura e classificazione delle Unità professionali~\cite{DL2018a}}
\newacronym{OCSE}{OCSE}{Organizzazione per la cooperazione e lo sviluppo economico}
\newacronym{OER}{OER}{Open Educational Resources~\cite{RA2012}}
\newacronym{OF}{OF}{Offerta Formativa}
\newacronym{OSA}{OSA}{Obiettivi specifici di apprendimento~\cite{CIRC2005a}.}
\newacronym{PCTO}{PCTO}{Percorsi per le competenze trasversali e per l'orientamento~\cite{LEGGE2018}}
\newacronym{PDP}{PDP}{Piano Didattico Personalizzato}
\newacronym{PECuP}{P.E.Cu.P.}{Profilo educativo, culturale e professionale~\cite{DL2005a}}
\newacronym{PFI}{PFI}{Progetto Formativo Individuale~\cite{DL2017b}~\cite{DL2018a}}
\newacronym{PIAAC}{PIAAC}{Programme for the International Assessment of Adult Competencies~\cite{RA2018}}
\newacronym{PISA}{PISA}{Programme for International Student Assessment}
\newacronym{POF}{POF}{Piano dell'offerta formativa~\cite{CIRC2005}}
\newacronym{PPT}{PPT}{Pen and paper test,Penna e carta test}
\newacronym{PSN}{PSN}{Programma Statistico Nazionale}
\newacronym{QCER}{QCER}{Quadro comune di riferimento per la conoscenza delle lingue in ambito europeo~\cite{LG2019}}
\newacronym{QCGQ}{QCGQ}{Quadro comune di garanzia della qualità~\cite{RA2009}}
\newacronym{QNQR}{QNQR}{Quadro di Riferimento Nazionale delle Qualificazioni Regionali}
\newacronym{QNQ}{QNQ}{Quadro nazionale delle qualificazioni~\cite{DL2018}}
\newacronym{RAV}{RAV}{Rapporto di Auto-Valutazione}
\newacronym{SDG}{SDG}{Sustainable Development Goals}
\newacronym{SEE}{SEE}{Spazio Economico Europeo}
\newacronym{SEP}{SEP}{Settori economico professionali}
\newacronym{SISTAN}{SISTAN}{Sistema statistico nazionale}
\newacronym{SMC}{SMC}{Simple Multiple Choice, Scelta multipla semplice}
\newacronym{STEM}{STEM}{Scienza, Tecnologia, Ingegneria e Matematica~\cite{RA2018}~\cite{Wikipedia2019b}}
\newacronym{STEAM}{STEAM}{Scienza, Tecnologia, Arte, Ingegneria e Matematica}
\newacronym{T-VET}{T-VET}{Technical and Vocational Education and Training~\cite{LG2019}}
\newacronym{TIC}{TIC}{Tecnologie di informazione e comunicazione}
\newacronym{TSI}{TSI}{Tecnologie della società dell'informazione~\cite{RA2006}}
\newacronym{UNESCO}{UNESCO}{United Nations Educational, Scientific and Cultural Organization}
\newacronym{UdA}{UdA}{Unità di Apprendimento~\cite{DL2018a}}
\newacronym{VET}{VET}{Vocational Education and Training}
\newglossaryentry{abilita}{name={Abilità},description={\begin{tabular}{cp{12cm}}\toprule 2005	&Le abilità rappresentano il saper fare che una cultura reputa importante trasmettere alle nuove generazioni, per realizzare opere o conseguire scopi. È abile colui che non solo produce qualcosa o risolve problemi, ma colui che conosce anche le ragioni di questo “fare”, sa perché, operando in un certo modo e rispettando determinate procedure, si ottengono determinati risultati. Come le conoscenze, sono ordinate, nelle Indicazioni nazionali, per “discipline” e per “Educazione alla Convivenza civile” e costituiscono, con esse, gli “obiettivi specifici di apprendimento” che i docenti trasformano in obiettivi formativi~\cite{CIRC2005a}.\\ \midrule 2017-18	&Capacità di applicare le conoscenze e di usare il know-how per portare a termine compiti e risolvere problemi, Nel contesto dell'EQF, le abilità sono descritte come cognitive (comprendenti l'uso del pensiero logico, intuitivo e creativo) e pratiche (comprendenti la manualità e l'uso di metodi, materiali, strumenti e utensili)~\cite{RA2017}~\cite{RA2008}.\\ \bottomrule\end{tabular}\mioindex{2008}{Abilità}\mioindex{2017}{Abilità}\mioindex{2005}{Abilità}}}
\newglossaryentry{accreditamento}{name={Accreditamento},description={La procedura mediante la quale le Regioni e le Province autonome di Trento e Bolzano riconoscono a una istituzione scolastica di IP, l'idoneità a erogare percorsi di IeFP per il rilascio della qualifica e del diploma professionale quadriennale di cui all'art. 17 del decreto legislativo 17 ottobre 2005, n. 226~\cite{DL2018b}.\mioindex{2018}{Accreditamento}}}
\newglossaryentry{alternanza}{name={Alternanza scuola-lavoro},description={\begin{tabular}{cp{12cm}}\toprule 2005&Alternanza scuola-lavoro modalità di realizzazione dei corsi del secondo ciclo, sia nel sistema dei licei, sia nel sistema dell'istruzione e della formazione professionale, per assicurare ai giovani, oltre alle conoscenze di base, l'acquisizione di competenze spendibili nel mercato del lavoro~\cite{DL2005}. \\\midrule 2016 & Per alternanza scuola-lavoro, si intende una metodologia didattica che consente agli studenti che frequentano gli istituti di istruzione superiore di svolgere una parte del proprio percorso formativo presso un'impresa o un ente, l'alternanza scuola lavoro si basa su una concezione integrata del processo educativo in cui il momento formativo, attuato mediante lo studio teorico d'aula, e il momento applicativo, attuato mediante esperienze assistite sul posto di lavoro, si fondono~\cite{INAIL2016}. \\\bottomrule\end{tabular}\mioindex{2005}{Alternanza!scuola-lavoro}\mioindex{2016}{Alternanza!scuola-lavoro}}}
\newglossaryentry{ambientelavoro}{name={Ambiente di lavoro},description={Si intende non solo lo stabilimento aziendale, bensì anche un eventuale cantiere all'aperto o un luogo pubblico, purché in essi si svolga un progetto di alternanza scuola-lavoro e l'attività ivi svolta presenti le caratteristiche oggettive elencate dall'art.1, n. 28 del d.p.r, 1124/65~\cite{INAIL2016}.\mioindex{2016}{Ambiente di lavoro}}}
\newglossaryentry{analisifabbisognocompetenze}{name={Analisi del fabbisogno di competenze},description={L'analisi quantitativa o qualitativa disponibile di dati aggregati sulle competenze da fonti esistenti relative al mercato del lavoro e delle corrispondenti opportunità di apprendimento nel sistema di istruzione e formazione, che può contribuire all'orientamento e alla consulenza, alle procedure di assunzione, alla scelta del percorso di studi, di formazione e di carriera professionale~\cite{DE2018}.\mioindex{2018}{Analisi!del fabbisogno di competenze}\mioindex{2018}{Competenze! analisi del fabbisogno di competenze}}}
\newglossaryentry{apprendimentoformale}{name={Apprendimento formale},description={\begin{tabular}{cp{12cm}}\toprule 2012		&Apprendimento erogato in un contesto organizzato e strutturato, specificamente dedicato all'apprendimento, che di norma porta all'ottenimento di qualifiche, generalmente sotto forma di certificati o diplomi; comprende sistemi di istruzione generale, formazione professionale iniziale e istruzione superiore~\cite{RA2012}\mioindex{2012}{Apprendimento!formale}. \\ \midrule 2013		& Qualsiasi attività intrapresa dalla persona in modo formale, non formale e informale, nelle varie fasi della vita, al fine di migliorare le conoscenze, le capacità e le competenze, in una prospettiva di crescita personale, civica, sociale e occupazionale~\cite{DL2013}\mioindex{2013}{Apprendimento!formale}. \\ \midrule 2018		& Apprendimento che si attua nel sistema di istruzione e formazione e nelle università e istituzioni di alta formazione artistica, musicale e coreutica, e che si conclude con il conseguimento di un titolo di studio o di una qualifica o diploma professionale, conseguiti anche in apprendistato, o di una certificazione riconosciuta, nel rispetto della legislazione vigente in materia di ordinamenti scolastici e universitari, a norma dell'articolo 2, comma 1, lettera b), del decreto legislativo 16 gennaio 2013, n. 13~\cite{DL2018a}\mioindex{2018}{Apprendimento!formale}. \\ \bottomrule \end{tabular}}}
\newglossaryentry{apprendimentoinformale}{name={Apprendimento informale},description={\begin{tabular}{cp{12cm}} \toprule 2012		&Apprendimento risultante dalle attività della vita quotidiana legate al lavoro, alla famiglia o al tempo libero e non strutturato in termini di obiettivi di apprendimento, di tempi o di risorse dell'apprendimento; esso può essere non intenzionale dal punto di vista del discente; esempi di risultati di apprendimento acquisiti mediante l'apprendimento informale sono le abilità acquisite durante le esperienze di vita e lavoro come la capacità di gestire progetti o le abilità ITC acquisite sul lavoro; le lingue e le abilità interculturali acquisite durante il soggiorno in un altro paese; le abilità ITC acquisite al di fuori del lavoro, le abilità acquisite nel volontariato, nelle attività culturali e sportive, nel lavoro, nell'animazione socio educativa e mediante attività svolte in casa (ad esempio l'accudimento dei bambini)~\cite{RA2012}\mioindex{2012}{Apprendimento!informale}. \\ \midrule 2013-18		& Apprendimento che, anche a prescindere da una scelta intenzionale, si realizza nello svolgimento, da parte di ogni persona, di attività nelle situazioni di vita quotidiana e nelle interazioni che in essa hanno luogo, nell'ambito del contesto di lavoro, familiare e del tempo libero~\cite{DL2013}\mioindex{2013}{Apprendimento!informale}~\cite{DL2018a}\mioindex{2018}{Apprendimento!informale}. \\ \bottomrule\end{tabular}}}
\newglossaryentry{apprendimentononformale}{name={Apprendimento non formale},description={\begin{tabular}{cp{12cm}}\toprule 2013		&Apprendimento caratterizzato da una scelta intenzionale della persona, che si realizza al di fuori dei sistemi indicati alla lettera b), in ogni organismo che persegua scopi educativi e formativi, anche del volontariato, del servizio civile nazionale e del privato sociale e nelle imprese~\cite{DL2013}\mioindex{2013}{Apprendimento!non formale}. \\ \midrule 2018		& Apprendimento caratterizzato da una scelta intenzionale della persona, che si realizza al di fuori dei sistemi indicati per l'apprendimento formale, in ogni organismo che persegua scopi educativi e formativi, anche del volontariato, del servizio civile nazionale e del privato sociale e nelle imprese, a norma dell'articolo 2, comma 1, lettera c), del decreto legislativo 16 gennaio 2013, n. 13~\cite{DL2018a}\mioindex{2018}{Apprendimento!non formale}. \\ \bottomrule\end{tabular}}}
\newglossaryentry{apprendimentopermanente}{name={Apprendimento permanente},description={\begin{tabular}{cp{12cm}} \toprule 2013		&Qualsiasi attività intrapresa dalla persona in modo formale, non formale e informale, nelle varie fasi della vita, al fine di migliorare le conoscenze, le capacità e le competenze, in una prospettiva di crescita personale, civica, sociale e occupazionale~\cite{DL2013}\mioindex{2013}{Apprendimento!permanente}. \\ \midrule 2013		& L'intero complesso di istruzione generale, istruzione e formazione professionale, istruzione non formale e apprendimento informale intrapresi nel corso della vita che comporta un miglioramento delle conoscenze, delle abilità e delle competenze, che può includere l'etica professionale~\cite{DI2013}\mioindex{2013}{Apprendimento!permanente}. \\ \bottomrule\end{tabular}}}
\newglossaryentry{apprendistatoduale}{name={Apprendistato duale},description={L'apprendistato duale è una tipologia di contratto a causa mista che prevede la concomitanza di istruzione e formazione professionale, Tutte le tipologie di contratto di apprendistato si possono ricondurre al sistema duale~\cite{DL2015c}.\mioindex{2015}{Apprendistato duale}}}
\newglossaryentry{associazionetemporaneaimprese}{name={Associazione temporanea di imprese},description={Designa un insieme di imprenditori, o fornitori, o prestatori di servizi, costituito, anche mediante scrittura privata, allo scopo di partecipare alla procedura di affidamento di uno specifico contratto pubblico, mediante presentazione di una unica offerta~\cite{DL2006}.\mioindex{2006}{Associazione!temporanea di imprese}}}
\newglossaryentry{associazionetemporanescopo}{name={Associazione temporanea di scopo},description={L'associazione temporanea di scopo è un accordo in base al quale i partecipanti conferiscono ad uno di essi un mandato di rappresentanza nei confronti di un soggetto finanziatore, per la realizzazione di un progetto di interesse comune.}}
\newglossaryentry{atecogg}{name={ATECO},description={Strumento adottato dall'Istituto nazionale di statistica (ISTAT) per classificare e rappresentare le attività economiche~\cite{DL2018a}.\mioindex{2018}{ATECO}}}
\newglossaryentry{atlantelavorodellequalificazioni}{name={Atlante del lavoro e delle qualificazioni},description={Dispositivo classificatorio e informativo realizzato sulla base delle sequenze descrittive della Classificazione dei settori economico-professionali, anche ai sensi dell'art. 8 del decreto legislativo n. 13 del 2013 e dell'art. 3, comma 5, del decreto interministeriale del 30 giugno 2015, e parte integrante dei sistemi informativi di cui agli articoli 13 e 15 del decreto legislativo n. 150 del 2015~\cite{DL2018}.\mioindex{2018}{Atlante del lavoro!e delle qualificazioni}}}
\newglossaryentry{attestazioneparteprima}{name={Attestazione di parte prima},description={Attestazione la cui validità delle informazioni contenute è data dalla autodichiarazione della persona, anche laddove attuata con un percorso accompagnato e realizzata attraverso procedure e modulistiche predefinite~\cite{DL2015a}.\mioindex{2015}{Attestazione!di parte prima}}}
\newglossaryentry{attestazioneparteseconda}{name={Attestazione di parte seconda},description={Attestazione rilasciata su responsabilità dell'ente titolato che eroga servizi di individuazione e validazione e certificazione delle competenze, in rapporto agli elementi di regolamentazione e garanzia del processo in capo all'ente titolare ai sensi del decreto legislativo 16 gennaio 2013, n. 13~\cite{DL2015a}.\mioindex{2015}{Attestazione!di parte seconda}}}
\newglossaryentry{attestazioneparteterza}{name={Attestazione di parte terza},description={Attestazione rilasciata su responsabilità dell'ente titolare, con il supporto dell'ente titolato che eroga i servizi di individuazione e validazione e certificazione delle competenze ai sensi del decreto legislativo 16 gennaio 2013, n. 13~\cite{DL2015a}.\mioindex{2015}{Attestazione!di parte terza}}}
\newglossaryentry{attivitlavororiservata}{name={Attività di lavoro riservata},description={Attività di lavoro riservata a persone iscritte in albi o elenchi ai sensi dell'art. 2229 del codice civile nonchè alle professioni sanitarie e ai mestieri artigianali, commerciali e di pubblico esercizio disciplinati da specifiche normative~\cite{DL2015a}.\mioindex{2015}{Attività!di lavoro riservata}}}
\newglossaryentry{autoritcompetente}{name={Autorità competente},description={Qualsiasi autorità o organismo abilitato da uno Stato membro in particolare a rilasciare o a ricevere titoli di formazione e altri documenti o informazioni, nonché a ricevere le domande e ad adottare le decisioni di cui alla presente direttiva~\cite{DI2005}.\mioindex{2005}{Autorità!competente}}}
\newglossaryentry{autovalutazionedellecompetenze}{name={Autovalutazione delle competenze},description={Il processo di riflessione sistematica della persona sulle proprie competenze tramite il riferimento a descrizioni fisse delle competenze~\cite{DE2018}.\mioindex{2018}{Autovalutazione!delle competenze}\mioindex{2018}{Competenze!autovalutazione}}}
\newglossaryentry{bilanciocompetenze}{name={Bilancio di competenze},description={Processo volto all'individuazione e all'analisi delle conoscenze, abilità e competenze di una persona, comprese attitudini e motivazioni, per definire un progetto professionale e/o pianificare un progetto di riorientamento o formazione professionale; lo scopo di un bilancio di competenze è di aiutare una persona ad analizzare il profilo professionale acquisito, a comprendere la propria posizione nel mondo del lavoro e a progettare una carriera professionale, o in taluni casi, a prepararsi in vista della convalida dei risultati dell'apprendimento non formale o informale~\cite{RA2012}.\mioindex{2012}{Competenze!bilancio}\mioindex{2012}{Bilancio!di competenze}}}
\newglossaryentry{bilanciopersonalegg}{name={Bilancio personale},description={Strumento che evidenzia i saperi e le competenze acquisiti da ciascuna studentessa e da ciascuno studente, anche in modo non formale e informale, idoneo a rilevare le potenzialità e le carenze riscontrate~\cite{DL2018a}.\mioindex{2018}{Bilancio!personale}}}
\newglossaryentry{certificazionecompetenze}{name={Certificazione delle competenze},description={\begin{tabular}{cp{12cm}} \toprule 2005		&La certificazione delle competenze scaturisce dalla somma qualitativa e quantitativa delle rilevazioni e degli accertamenti effettuati nel percorso scolastico, coinvolge nella maniera professionalmente più alta i docenti, perché si assumono la responsabilità di certificarle a livello iniziale, intermedio ed esperto. È prevista anche una certificazione delle competenze degli allievi nel superamento delle prove di esame. Va sottolineato che questa competenza si aggiunge, e non si sostituisce, a quelle identificate nel Profilo~\cite{CIRC2005a}\mioindex{2005}{Competenze!certificazione}. \\ \midrule 2013		&Procedura di formale riconoscimento, da parte dell'ente titolato, in base alle norme generali, ai livelli essenziali delle prestazioni e agli standard minimi di cui al presente decreto, delle competenze acquisite dalla persona in contesti formali, anche in caso di interruzione del percorso formativo, o di quelle validate acquisite in contesti non formali e informali. La procedura di certificazione delle competenze si conclude con il rilascio di un certificato conforme agli standard minimi~\cite{DL2013}\mioindex{2013}{Competenze!certificazione}. \\ \midrule 2018		& Procedura di formale riconoscimento, da parte dell'ente titolato a norma dell'articolo 2, lettera g), del decreto legislativo 16 gennaio 2013, n. 13, in base alle norme generali, ai livelli essenziali delle prestazioni e agli standard minimi di cui al medesimo decreto legislativo, delle competenze acquisite dalla persona in contesti formali, anche in caso di interruzione del percorso formativo, o di quelle validate acquisite in contesti non formali e informali. La procedura di certificazione delle competenze si conclude con il rilascio di un certificato conforme agli standard minimi di cui all'articolo 6 del decreto legislativo n. 13 del 2013~\cite{DL2018a}\mioindex{2018}{Competenze!certificazione}. \\ \bottomrule \end{tabular}}}
\newglossaryentry{classificazione deisettorieconomicoprofessionali}{name={Classificazione dei settori economico professionali},description={\begin{tabular}{cp{12cm}} \toprule 2015-18		&Sistema di classificazione che, a partire dai codici di classificazione statistica ISTAT relativi alle attività economiche (ATECO) e alle professioni (Classificazione delle professioni), consente di aggregare in settori l'insieme delle attività e delle professionalità operanti sul mercato del lavoro, I settori economico-professionali sono articolati secondo una sequenza descrittiva che prevede la definizione di: comparti, processi di lavoro, aree di attività, attività di lavoro e ambiti tipologici di esercizio~\cite{DL2015a}\mioindex{2015}{Classificazione!dei settori economico professionali}~\cite{DL2018a}\mioindex{2018}{Classificazione!dei settori economico professionali}. \\ \midrule 2019		& La classificazione dei settori economico professionali (SEP) è stata ottenuta utilizzando i codici delle classificazioni adottate dall'ISTAT relativamente alle attività economiche (ATECO 2007) e alle professioni (Classificazione delle professioni 2011). La classificazione SEP è composta da 23 settori, più un settore denominato Area COMUNE, che comprende tutte quelle attività lavorative non caratterizzate da un settore di riferimento specifico~\cite{LG2019a}\mioindex{2019}{Classificazione!dei settori economico professionali}. \\ \bottomrule\end{tabular}}}
\newglossaryentry{comitatotecnicoscientifico}{name={Comitato Tecnico Scientifico},description={Composto da docenti e da esperti del mondo del lavoro, delle professioni e della ricerca scientifica e tecnologica, con funzioni consultive e di proposta per l'organizzazione delle aree di indirizzo e l'utilizzazione degli spazi di autonomia e flessibilità; ai componenti del comitato non spettano compensi ad alcun titolo~\cite{DL2010a}~\cite{DL2017a}.\mioindex{2010}{Comitato!Tecnico Scientifico}\mioindex{2017}{Comitato!Tecnico Scientifico}}}
\newglossaryentry{competenzaalfabeticafunzionale}{name={Competenza alfabetica funzionale},description={La competenza alfabetica funzionale indica la capacità di individuare, comprendere, esprimere, creare e interpretare concetti, sentimenti, fatti e opinioni, in forma sia orale sia scritta, utilizzando materiali visivi, sonori e digitali attingendo a varie discipline e contesti. Essa implica l'abilità di comunicare e relazionarsi efficacemente con gli altri in modo opportuno e creativo. Il suo sviluppo costituisce la base per l'apprendimento successivo e l'ulteriore interazione linguistica, a seconda del contesto, la competenza alfabetica funzionale può essere sviluppata nella lingua madre, nella lingua dell'istruzione scolastica e/o nella lingua ufficiale di un paese o di una regione~\cite{RA2018}.\mioindex{2018}{Competenza!alfabetica funzionale}}}
\newglossaryentry{competenzaconsapevolezzaespressioneculturali}{name={Competenza consapevolezza ed espressione culturali},description={Consapevolezza dell'importanza dell'espressione creativa di idee, esperienze ed emozioni in un'ampia varietà di mezzi di comunicazione, compresi la musica, le arti dello spettacolo, la letteratura e le arti visive~\cite{RA2006}.\mioindex{2006}{Competenza!consapevolezza ed espressione culturali}}}
\newglossaryentry{competenzadigitalegg}{name={Competenza digitale},description={\begin{tabular}{cp{12cm}}\toprule 2006&La competenza digitale consiste nel saper utilizzare con dimestichezza e spirito critico le tecnologie della società dell'informazione (TSI) per il lavoro, il tempo libero e la comunicazione. Essa è supportata da abilità di base nelle TIC: l'uso del computer per reperire, valutare, conservare, produrre, presentare e scambiare informazioni nonché per comunicare e partecipare a reti collaborative tramite Internet~\cite{RA2006}\mioindex{2006}{Competenza!digitale}.\\\midrule 2018&La competenza digitale presuppone l'interesse per le tecnologie digitali e il loro utilizzo con dimestichezza e spirito critico e responsabile per apprendere, lavorare e partecipare alla società. Essa comprende l'alfabetizzazione informatica e digitale, la comunicazione e la collaborazione, l'alfabetizzazione mediatica, la creazione di contenuti digitali (inclusa la programmazione), la sicurezza (compreso l'essere a proprio agio nel mondo digitale e possedere competenze relative alla cibersicurezza), le questioni legate alla proprietà intellettuale, la risoluzione di problemi e il pensiero critico~\cite{RA2018}\mioindex{2018}{Competenza!digitale}.\end{tabular}}}
\newglossaryentry{competenzaiingegneria}{name={Competenza tecnologie e ingegneria},description={Le competenze in tecnologie e ingegneria sono applicazioni di tali conoscenze e metodologie per dare risposta ai desideri o ai bisogni avvertiti dagli esseri umani. La competenza in scienze, tecnologie e ingegneria implica la comprensione dei cambiamenti determinati dall'attività umana e della responsabilità individuale del cittadino~\cite{RA2018}\mioindex{2018}{Competenza!tecnologie e ingegneria}.}}
\newglossaryentry{competenzaimparareimparare}{name={Competenza imparare a imparare},description={Imparare a imparare è l'abilità di perseverare nell'apprendimento, di organizzare il proprio apprendimento anche mediante una gestione efficace del tempo e delle informazioni, sia a livello individuale che in gruppo, Questa competenza comprende la consapevolezza del proprio processo di apprendimento e dei propri bisogni, l'identificazione delle opportunità disponibili e la capacità di sormontare gli ostacoli per apprendere in modo efficace, Questa competenza comporta l'acquisizione, l'elaborazione e l'assimilazione di nuove conoscenze e abilità come anche la ricerca e l'uso delle opportunità di orientamento, Il fatto di imparare a imparare fa sì che i discenti prendano le mosse da quanto hanno appreso in precedenza e dalle loro esperienze di vita per usare e applicare conoscenze e abilità in tutta una serie di contesti: a casa, sul lavoro, nell'istruzione e nella formazione. La motivazione e la fiducia sono elementi essenziali perché una persona possa acquisire tale competenza~\cite{RA2006}\mioindex{2006}{Competenza!imparare a imparare}.}}
\newglossaryentry{competenzaimprenditoriale}{name={Competenza imprenditoriale},description={La competenza imprenditoriale si riferisce alla capacità di agire sulla base di idee e opportunità e di trasformarle in valori per gli altri, Si fonda sulla creatività, sul pensiero critico e sulla risoluzione di problemi, sull'iniziativa e sulla perseveranza, nonché sulla capacità di lavorare in modalità collaborativa al fine di programmare e gestire progetti che hanno un valore culturale, sociale o finanziario~\cite{RA2018}\mioindex{2018}{Competenza!imprenditoriale}.}}
\newglossaryentry{competenzainmateriaconsapevolezza}{name={Competenza in materia di consapevolezza ed espressione culturali},description={La competenza in materia di consapevolezza ed espressione culturali implica la comprensione e il rispetto di come le idee e i significati vengono espressi creativamente e comunicati in diverse culture e tramite tutta una serie di arti e altre forme culturali, presuppone l'impegno di capire, sviluppare ed esprimere le proprie idee e il senso della propria funzione o del proprio ruolo nella società in una serie di modi e contesti~\cite{RA2018}\mioindex{2018}{Competenza!in materia di consapevolezza ed espressione culturali}.}}
\newglossaryentry{competenzainmateriadicittadinanza}{name={Competenza in materia di cittadinanza},description={La competenza in materia di cittadinanza si riferisce alla capacità di agire da cittadini responsabili e di partecipare pienamente alla vita civica e sociale, in base alla comprensione delle strutture e dei concetti sociali, economici, giuridici e politici oltre che dell'evoluzione a livello globale e della sostenibilità~\cite{RA2018}\mioindex{2018}{Competenza!in materia di cittadinanza}.}}
\newglossaryentry{competenzainscienze}{name={Competenza in scienze},description={La competenza in scienze si riferisce alla capacità di spiegare il mondo che ci circonda usando l'insieme delle conoscenze e delle metodologie, comprese l'osservazione e la sperimentazione, per identificare le problematiche e trarre conclusioni che siano basate su fatti empirici, e alla disponibilità a farlo~\cite{RA2018}\mioindex{2018}{Competenza!in scienze}.}}
\newglossaryentry{competenzamatematica}{name={Competenza matematica},description={\begin{tabular}{cp{12cm}}\toprule 2006&La competenza matematica è l'abilità di sviluppare e applicare il pensiero matematico per risolvere una serie di problemi in situazioni quotidiane, partendo da una solida padronanza delle competenze aritmetico-matematiche, l'accento è posto sugli aspetti del processo e dell'attività oltre che su quelli della conoscenza. La competenza matematica comporta, in misura variabile, la capacità e la disponibilità a usare modelli matematici di pensiero (pensiero logico e spaziale) e di presentazione (formule, modelli, costrutti, grafici, carte)~\cite{RA2006}\mioindex{2006}{Competenza!matematica}. \\\midrule 2018& La competenza matematica è la capacità di sviluppare e applicare il pensiero e la comprensione matematici per risolvere una serie di problemi in situazioni quotidiane, partendo da una solida padronanza della competenza aritmetico-matematica, l'accento è posto sugli aspetti del processo e dell'attività oltre che sulla conoscenza. La competenza matematica comporta, a differenti livelli, la capacità di usare modelli matematici di pensiero e di presentazione (formule, modelli, costrutti, grafici, diagrammi) e la disponibilità a farlo~\cite{RA2018}\mioindex{2018}{Competenza!matematica}. \\\bottomrule\end{tabular}}}
\newglossaryentry{competenzamultilinguistica}{name={Competenza multilinguistica},description={Tale competenza definisce la capacità di utilizzare diverse lingue in modo appropriato ed efficace allo scopo di comunicare, In linea di massima essa condivide le abilità principali con la competenza alfabetica: si basa sulla capacità di comprendere, esprimere e interpretare concetti, pensieri, sentimenti, fatti e opinioni in forma sia orale sia scritta (comprensione orale, espressione orale, comprensione scritta ed espressione scritta) in una gamma appropriata di contesti sociali e culturali a seconda dei desideri o delle esigenze individuali. Le competenze linguistiche comprendono una dimensione storica e competenze interculturali, Tale competenza si basa sulla capacità di mediare tra diverse lingue e mezzi di comunicazione, come indicato nel quadro comune europeo di riferimento, Secondo le circostanze, essa può comprendere il mantenimento e l'ulteriore sviluppo delle competenze relative alla lingua madre, nonché l'acquisizione della lingua ufficiale o delle lingue ufficiali di un paese~\cite{RA2018}\mioindex{2018}{Competenza!multilinguistica}.}}
\newglossaryentry{competenzansoiniziativaimprenditorialit}{name={Competenza senso di iniziativa e di imprenditorialità},description={Il senso di iniziativa e l'imprenditorialità concernono la capacità di una persona di tradurre le idee in azione, In ciò rientrano la creatività, l'innovazione e l'assunzione di rischi, come anche la capacità di pianificare e di gestire progetti per raggiungere obiettivi, È una competenza che aiuta gli individui, non solo nella loro vita quotidiana, nella sfera domestica e nella società, ma anche nel posto di lavoro, ad avere consapevolezza del contesto in cui operano e a poter cogliere le opportunità che si offrono ed è un punto di partenza per le abilità e le conoscenze più specifiche di cui hanno bisogno coloro che avviano o contribuiscono ad un'attività sociale o commerciale. Essa dovrebbe includere la consapevolezza dei valori etici e promuovere il buon governo~\cite{RA2006}\mioindex{2006}{Competenza!senso di iniziativa e di imprenditorialità}.}}
\newglossaryentry{competenzapersonalesociale}{name={Competenza personale, sociale e capacità di imparare a imparare},description={La competenza personale, sociale e la capacità di imparare a imparare consiste nella capacità di riflettere su sé stessi, di gestire efficacemente il tempo e le informazioni, di lavorare con gli altri in maniera costruttiva, di mantenersi resilienti e di gestire il proprio apprendimento e la propria carriera. Comprende la capacità di far fronte all'incertezza e alla complessità, di imparare a imparare, di favorire il proprio benessere fisico ed emotivo, di mantenere la salute fisica e mentale, nonché di essere in grado di condurre una vita attenta alla salute e orientata al futuro, di empatizzare e di gestire il conflitto in un contesto favorevole e inclusivo~\cite{RA2018}.\mioindex{2018}{Competenza!personale, sociale e capacità di imparare a imparare}}}
\newglossaryentry{competenzasocialiciviche}{name={Competenza sociali e civiche},description={Queste includono competenze personali, interpersonali e interculturali e riguardano tutte le forme di comportamento che consentono alle persone di partecipare in modo efficace e costruttivo alla vita sociale e lavorativa, in particolare alla vita in società sempre più diversificate, come anche a risolvere i conflitti ove ciò sia necessario. La competenza civica dota le persone degli strumenti per partecipare appieno alla vita civile grazie alla conoscenza dei concetti e delle strutture sociopolitici e all'impegno a una partecipazione attiva e democratica~\cite{RA2006}.\mioindex{2006}{Competenza!sociali e civiche}}}
\newglossaryentry{competenza}{name={Competenza},description={\begin{tabular}{cp{12cm}}\toprule 2005&La competenza è l’agire personale di ciascuno, basato sulle conoscenze e abilità acquisite, adeguato, in un determinato contesto, in modo soddisfacente e socialmente riconosciuto, a rispondere ad un bisogno, a risolvere un problema, a eseguire un compito, a realizzare un progetto. Non è mai un agire semplice, atomizzato, astratto, ma è sempre un agire complesso che coinvolge tutta la persona e che connette in maniera unitaria e inseparabile i saperi (conoscenze) e i saper fare (abilità), i comportamenti individuali e relazionali, gli atteggiamenti emotivi, le scelte valoriali, le motivazioni e i fini. Per questo, nasce da una continua interazione tra persona, ambiente e società, e tra significati personali e sociali, impliciti ed espliciti~\cite{CIRC2005a}\mioindex{2005}{Competenza}. \\ \midrule 2008& Comprovata capacità di utilizzare conoscenze, abilità e capacità personali, sociali e/o metodologiche, in situazioni di lavoro o di studio e nello sviluppo professionale e personale, Nel contesto del Quadro europeo delle qualifiche le competenze sono descritte in termini di responsabilità e autonomia~\cite{RA2008}\mioindex{2008}{Competenza}. \\ \midrule 2013-18&Comprovata capacità di utilizzare, in situazioni di lavoro, di studio o nello sviluppo professionale e personale, un insieme strutturato di conoscenze e di abilità acquisite nei contesti di apprendimento formale, non formale o informale~\cite{DL2013}~\cite{DL2018a}\mioindex{2013}{Competenza}\mioindex{2018}{Competenza}. \\\midrule 2017& Comprovata capacità di utilizzare conoscenze, abilità e capacità personali, sociali e/o metodologiche in situazioni di lavoro o di studio e nello sviluppo professionale e personale~\cite{RA2017}\mioindex{2017}{Competenza}. \\ \midrule 2018& Ciò che una persona sa, capisce ed è capace di fare~\cite{DE2018}\mioindex{2018}{Competenza}.\\\bottomrule \end{tabular}}}
\newglossaryentry{competenzeComunicazione nella madrelingua}{name={Competenza comunicazione nella madrelingua},description={La comunicazione nella madrelingua è la capacità di esprimere e interpretare concetti, pensieri, sentimenti, fatti e opinioni in forma sia orale sia scritta (comprensione orale, espressione orale, comprensione scritta ed espressione scritta) e di interagire adeguatamente e in modo creativo sul piano linguistico in un'intera gamma di contesti culturali e sociali, quali istruzione e formazione, lavoro, vita domestica e tempo libero~\cite{RA2006}\mioindex{2006}{Competenza!comunicazione nella madrelingua}.}}
\newglossaryentry{competenzeComunicazionelinguestraniere}{name={Competenza comunicazione in lingue straniere},description={La comunicazione nelle lingue straniere condivide essenzialmente le principali abilità richieste per la comunicazione nella madrelingua: essa si basa sulla capacità di comprendere, esprimere e interpretare concetti, pensieri, sentimenti, fatti e opinioni in forma sia orale sia scritta — comprensione orale, espressione orale, comprensione scritta ed espressione scritta — in una gamma appropriata di contesti sociali e culturali — istruzione e formazione, lavoro, casa, tempo libero — a seconda dei desideri o delle esigenze individuali. La comunicazione nelle lingue straniere richiede anche abilità quali la mediazione e la comprensione interculturale, Il livello di padronanza di un individuo varia inevitabilmente tra le quattro dimensioni (comprensione orale, espressione orale, comprensione scritta ed espressione scritta) e tra le diverse lingue e a seconda del suo background sociale e culturale, del suo ambiente e delle sue esigenze e/o dei suoi interessi~\cite{RA2006}\mioindex{2006}{Competenza!comunicazione in lingue straniere}.}}
\newglossaryentry{competenzechiave}{name={Competenze chiave},description={Le competenze sono definite come una combinazione di conoscenze, abilità e atteggiamenti, in cui: d) la conoscenza si compone di fatti e cifre, concetti, idee e teorie che sono già stabiliti e che forniscono le basi per comprendere un certo settore o argomento; e) per abilità si intende sapere ed essere capaci di eseguire processi ed applicare le conoscenze esistenti al fine di ottenere risultati; f) gli atteggiamenti descrivono la disposizione e la mentalità per agire o reagire a idee, persone o situazioni. Le competenze chiave sono quelle di cui tutti hanno bisogno per la realizzazione e lo sviluppo personali, l'occupabilità, l'inclusione sociale, uno stile di vita sostenibile, una vita fruttuosa in società pacifiche, una gestione della vita attenta alla salute e la cittadinanza attiva. Esse si sviluppano in una prospettiva di apprendimento permanente, dalla prima infanzia a tutta la vita adulta, mediante l'apprendimento formale, non formale e informale in tutti i contesti, compresi la famiglia, la scuola, il luogo di lavoro, il vicinato e altre comunità~\cite{RA2018}\mioindex{2018}{Competenze!chiave}.}}
\newglossaryentry{competenzecompetenzabasecamposcientificotecnologico}{name={Competenza di base in campo scientifico e tecnologico},description={La competenza in campo scientifico si riferisce alla capacità e alla disponibilità a usare l'insieme delle conoscenze e delle metodologie possedute per spiegare il mondo che ci circonda sapendo identificare le problematiche e traendo le conclusioni che siano basate su fatti comprovati. La competenza in campo tecnologico è considerata l'applicazione di tale conoscenza e metodologia per dare risposta ai desideri o bisogni avvertiti dagli esseri umani. La competenza in campo scientifico e tecnologico comporta la comprensione dei cambiamenti determinati dall'attività umana e la consapevolezza della responsabilità di ciascun cittadino~\cite{RA2006}\mioindex{2006}{Competenza!di base in campo scientifico e tecnologico}.}}
\newglossaryentry{compitidirealt}{name={Compiti di realtà},description={I compiti di realtà si identificano nella richiesta rivolta allo studente di risolvere una situazione problematica, complessa e nuova, quanto più possibile vicina al mondo reale, utilizzando conoscenze e abilità già acquisite e trasferendo procedure e condotte cognitive in contesti e ambiti di riferimento moderatamente diversi da quelli resi familiari dalla pratica didattica, pur non escludendo prove che chiamino in causa una sola disciplina, si ritiene opportuno privilegiare prove per la cui risoluzione l'alunno debba richiamare in forma integrata, componendoli autonomamente, più apprendimenti acquisiti~\cite{LG2018}.\mioindex{2018}{Compiti!di realtà}}}
\newglossaryentry{conoscenze}{name={Conoscenze},description={\begin{tabular}{cp{12cm}}\toprule 2005 & Le conoscenze rappresentano il sapere che costituisce il patrimonio di una cultura; sono un insieme di informazioni, nozioni, dati, principi, regole di comportamento, teorie, concetti codificati e conservati perché ritenuti degni di essere trasmessi alle nuove generazioni. Le conoscenze sono ordinate, nelle Indicazioni nazionali, per “discipline” e per “Educazione alla Convivenza civile” e costituiscono, unitamente alle abilità, gli “obiettivi specifici di apprendimento”~\cite{CIRC2005a}\mioindex{2005}{Conoscenze}. \\\midrule 2008&Risultato dell'assimilazione di informazioni attraverso l'apprendimento. Le conoscenze sono l'insieme di fatti, principi, teorie e pratiche che riguardano un ambito di lavoro o di studio, Nel contesto dell'EQF, le conoscenze sono descritte come teoriche e/o pratiche~\cite{RA2017}~\cite{RA2008}\mioindex{2017}{Conoscenze}\mioindex{2008}{Conoscenze}. \\\bottomrule\end{tabular}}}
\newglossaryentry{consigliosuperiorepubblicaistruzione}{name={Consiglio Superiore della Pubblica Istruzione},description={Il Consiglio superiore della pubblica istruzione è organo di garanzia dell'unitarietà del sistema nazionale dell'istruzione, Ha compiti di supporto tecnico-scientifico per l'esercizio delle funzioni di governo nelle materie di “istruzione universitaria, ordinamenti scolastici, programmi scolastici, organizzazione generale dell'istruzione scolastica e stato giuridico del personale” (articolo 1, comma 3, lettera q), della legge 59 del 15 marzo 1997)~\cite{LEGGE1997}\mioindex{1997}{Consiglio!Superiore della Pubblica Istruzione}.}}
\newglossaryentry{convalida}{name={Convalida},description={\begin{tabular}{cp{12cm}}\toprule 2012&Il processo mediante il quale un'autorità o un organismo competente conferma che un individuo ha acquisito, anche in un contesto di apprendimento non formale e informale, risultati dell'apprendimento misurati in relazione a uno standard appropriato e che si articola in quattro fasi distinte, vale a dire individuazione, documentazione, valutazione e certificazione dei risultati della valutazione sotto forma di qualifica piena, crediti o qualifica parziale, ove opportuno e in funzione delle circostanze nazionali~\cite{DE2018}~\cite{RA2012}\mioindex{2018}{Convalida}\mioindex{2012}{Convalida}.\\\midrule 2017&Processo in base al quale un'autorità competente conferma l'acquisizione, in un contesto di apprendimento non formale e informale, di risultati dell'apprendimento misurati in relazione a uno standard appropriato; si articola nelle seguenti quattro fasi distinte: individuazione, mediante un colloquio, delle esperienze specifiche dell'interessato; documentazione per rendere visibili le esperienze dell'interessato; valutazione formale di tali esperienze e certificazione dei risultati della valutazione, che può portare a una qualifica parziale o completa~\cite{RA2017}\mioindex{2017}{Convalida}. \\\bottomrule\end{tabular}}}
\newglossaryentry{cp2011}{name={CP2011},description={La classificazione CP2011 fornisce uno strumento per ricondurre tutte le professioni esistenti nel mercato del lavoro all’interno di un numero limitato di raggruppamenti professionali, da utilizzare per comunicare, diffondere e scambiare dati statistici e amministrativi sulle professioni, comparabili a livello internazionale; non deve invece essere inteso come strumento di regolamentazione delle professioni.~\cite{ISTAT2020}\mioindex{2011}{Classificazione!CP2011}.}}
\newglossaryentry{crediti}{name={Crediti},description={Unità che confermano che una parte della qualifica, costituita da un insieme coerente di risultati dell'apprendimento, è stata valutata e convalidata da un'autorità competente, secondo una norma concordata; i crediti sono concessi da autorità competenti quando il soggetto ha conseguito i risultati dell'apprendimento definiti, comprovati da opportune valutazioni, e possono essere espressi con un valore quantitativo (ad esempio crediti o unità di credito), che indica il carico di lavoro ritenuto solitamente necessario affinché una persona consegua i risultati dell'apprendimento corrispondenti~\cite{RA2017}\mioindex{2017}{Crediti}.}}
\newglossaryentry{creditoformativo}{name={Credito formativo},description={Per credito formativo si intende il valore attribuibile alle competenze, abilità e conoscenze acquisite nel percorso di apprendimento, certificate, validate e comunque riconoscibili ai fini dell'inserimento nel percorso di IP o di IeFP per il quale è stata presentata domanda di passaggio, anche in seguito ad eventuali verifiche in ingresso~\cite{LG2019a}\mioindex{2019}{Credito!formativo}.}}
\newglossaryentry{dimensioneeuropeaorientamento}{name={Dimensione europea dell'orientamento},description={La cooperazione e il sostegno a livello di Unione volti a rafforzare politiche, sistemi e pratiche di orientamento all'interno dell'Unione~\cite{DE2018}\mioindex{2018}{Dimensione!europea dell'orientamento}.}}
\newglossaryentry{dirigenteazienda}{name={Dirigente d'azienda},description={Qualsiasi persona che abbia svolto in un'impresa del settore professionale corrispondente:i) la funzione di direttore d'azienda o di filiale, o ii) la funzione di institore o vice direttore d'azienda, se tale funzione implica una responsabilità corrispondente a quella dell'imprenditore o del direttore d'azienda rappresentato, o iii) la funzione di dirigente con mansioni commerciali e/o tecniche e responsabile di uno o più reparti dell'azienda~\cite{DI2005}\mioindex{2018}{Dirigente!d'azienda}.}}
\newglossaryentry{entepubblicotitolareg}{name={Ente pubblico titolare},description={Amministrazione pubblica, centrale, regionale e delle province autonome titolare, a norma di legge, della regolamentazione di servizi di individuazione e validazione e certificazione delle competenze, Nello specifico sono da intendersi enti pubblici titolari: 1) il Ministero dell'istruzione, dell'università e della ricerca, in materia di individuazione e validazione e certificazione delle competenze riferite ai titoli di studio del sistema scolastico e universitario; 2) le regioni e le province autonome di Trento e Bolzano, in materia di individuazione e validazione e certificazione di competenze riferite a qualificazioni rilasciate nell'ambito delle rispettive competenze; 3) il Ministero del lavoro e delle politiche sociali, in materia di individuazione e validazione e certificazione di competenze riferite a qualificazioni delle professioni non organizzate in ordini o collegi, salvo quelle comunque afferenti alle autorità competenti di cui al successivo punto 4; 4) il Ministero dello sviluppo economico e le altre autorità competenti ai sensi dell'articolo 5 del decreto legislativo 9 novembre 2007, n. 206, in materia di individuazione e validazione e certificazione di competenze riferite a qualificazioni delle professioni regolamentate a norma del medesimo decreto~\cite{DL2013}\mioindex{2013}{Ente!pubblico titolare}.}}
\newglossaryentry{entetitolato}{name={Ente titolato},description={Soggetto, pubblico o privato, ivi comprese le camere di commercio, industria, artigianato e agricoltura, autorizzato o accreditato dall'ente pubblico titolare, ovvero deputato a norma di legge statale o regionale, ivi comprese le istituzioni scolastiche, le università e le istituzioni dell'alta formazione artistica, musicale e coreutica, a erogare in tutto o in parte servizi di individuazione e validazione e certificazione delle competenze, in relazione agli ambiti di titolarità enti pubblico titolare~\cite{DL2013}\mioindex{2013}{Ente!titolato}.}}
\newglossaryentry{esperienzaprofessionale}{name={Esperienza professionale},description={L'esercizio effettivo e legittimo della professione in questione in uno Stato membro, a tempo pieno o a tempo parziale per un periodo equivalente~\cite{DI2013}\mioindex{2013}{Esperienza!professionale}.}}
\newglossaryentry{formazioneregolamentata}{name={Formazione regolamentata},description={Qualsiasi formazione specificamente orientata all'esercizio di una professione determinata e consistente in un ciclo di studi completato, eventualmente, da una formazione professionale, un tirocinio professionale o una pratica professionale. La struttura e il livello della formazione professionale, del tirocinio professionale o della pratica professionale sono stabiliti dalle disposizioni legislative, regolamentari o amministrative dello Stato membro in questione e sono soggetti a controllo o autorizzazione dell'autorità designata a tal fine~\cite{DI2005}\mioindex{2005}{Formazione!regolamentata}.}}
\newglossaryentry{indirizzistudio}{name={Indirizzi di studio},description={Gli indirizzi di studio sono strutturati in: a) attività e insegnamenti di istruzione generale, comuni a tutti gli indirizzi, riferiti all'asse culturale dei linguaggi, all'asse matematico e all'asse storico sociale; b) attività e insegnamenti di indirizzo riferiti all'asse scientifico, tecnologico e professionale e, nel caso di presenza di una seconda lingua straniera, all'asse dei linguaggi~\cite{LG2019a}\mioindex{2019}{Indirizzi!di studio}.}}
\newglossaryentry{individuazionevalidazionecompetenze}{name={Individuazione e validazione delle competenze},description={Processo che conduce al riconoscimento, da parte dell'ente titolato in base alle norme generali, ai livelli essenziali delle prestazioni e agli standard minimi di cui al presente decreto, delle competenze acquisite dalla persona in un contesto non formale o informale. Ai fini della individuazione delle competenze sono considerate anche quelle acquisite in contesti formali. La validazione delle competenze puo' essere seguita dalla certificazione delle competenze ovvero si conclude con il rilascio di un documento di validazione conforme agli standard minimi~\cite{DL2013}\mioindex{2013}{Competenze!individuazione e validazione}.}}
\newglossaryentry{istituzionescolasticaaccreditata}{name={Istituzione scolastica accreditata},description={l'istituzione scolastica di I.P, cui é riconosciuta l'idoneità ad erogare percorsi di IeFP~\cite{DL2018b}\mioindex{2018}{Istituzione scolastica!accreditata}.}}
\newglossaryentry{istituzioniscolasticheIPgg}{name={Istituzioni scolastiche di I.P.},description={\begin{tabular}{cp{12cm}} \toprule 2018		&Le istituzioni scolastiche di IP, sono scuole territoriali dell'innovazione, aperte al territorio e concepite come laboratori di ricerca, sperimentazione e innovazione didattica. Esse definiscono i Piani triennali dell'offerta formativa secondo i principi e le finalità indicati all'articolo 1 del decreto legislativo, tenuto conto delle richieste degli studenti e delle famiglie per realizzare attività finalizzate al raggiungimento degli obiettivi formativi considerati prioritari a norma dell'articolo 1, comma 7, della legge n. 107 del 2015~\cite{DL2018a}\mioindex{2018}{Istituzione scolastica!di IP}.\\ \midrule 2018		& Istituzioni scolastiche che offrono percorsi di istruzione professionale a norma del decreto legislativo decreto legislativo 13 aprile 2017, n. 61~\cite{DL2018a}\mioindex{2018}{Istituzione scolastica!di IP}.\\\bottomrule\end{tabular}}}
\newglossaryentry{livelloavanzato}{name={Livello avanzato},description={Lo studente svolge compiti e problemi complessi in situazioni anche non note, mostrando padronanza nell'uso delle conoscenze e delle abilità, Sa proporre e sostenere le proprie opinioni e assumere autonomamente decisioni consapevoli~\cite{DL2010}\mioindex{2010}{Livello!avanzato}.}}
\newglossaryentry{livellobase}{name={Livello base},description={Lo studente svolge compiti semplici in situazioni note, mostrando di possedere conoscenze ed abilità essenziali e di saper applicare regole e procedure fondamentali~\cite{DL2010}\mioindex{2010}{Livello!base}.}}
\newglossaryentry{livellointermedio}{name={Livello intermedio},description={Lo studente svolge compiti e risolve problemi complessi in situazioni note, compie scelte consapevoli, mostrando di saper utilizzare le conoscenze e le abilita acquisite~\cite{DL2010}\mioindex{2010}{Livello!intermedio}.}}
\newglossaryentry{maternage}{name={Maternage},description={Una Regione o PA sprovvista di alcune qualificazioni può attingere al Repertorio di un'altra Regione e trasferire singole qualificazioni professionali nel proprio. Il Maternage è possibile anche attingendo al bacino informatico comune (il QNQR), che raccoglie tutte le qualificazioni professionali regionali esistenti.}}
\newglossaryentry{motiviimperativiinteressegenerale}{name={Motivi imperativi di interesse generale},description={Motivi riconosciuti tali dalla giurisprudenza della Corte di giustizia dell'Unione europea~\cite{DI2013}\mioindex{2010}{Motivi imperativi!di interesse generale}.}}
\newglossaryentry{nomenclaturaclassificazioneUnitprofessionaligg}{name={Nomenclatura e classificazione delle Unità professionali (NUP)},description={Strumento, adottato dall'ISTAT, per classificare e rappresentare le professioni; costituisce, a norma dell'articolo 3, comma 5, del decreto legislativo decreto legislativo 13 aprile 2017, n. 61 l'ulteriore riferimento, oltre al codice ATECO, per la declinazione degli indirizzi di studio da parte delle istituzioni scolastiche che offrono percorsi di istruzione professionale, in coerenza con le richieste del territorio secondo le priorità indicate dalle regioni nella propria programmazione e nei limiti degli spazi di flessibilità di cui all' articolo 6, comma 1, lettera b) del medesimo decreto legislativo~\cite{DL2018a}\mioindex{2018}{Nomenclatura!classificazione delle Unità professionali}.}}
\newglossaryentry{nomenclaturaunitprofessionali}{name={Nomenclatura delle unità professionali},description={Si tratta di uno strumento per ricondurre tutte le professioni esistenti nel mercato del lavoro all'interno di un numero limitato di raggruppamenti professionali. Ogni professione è definita dall'insieme di attività lavorative concretamente svolte. La logica utilizzata per aggregare professioni diverse all'interno di un medesimo raggruppamento si basa sul concetto di competenza, visto nella sua duplice dimensione del livello e del campo delle competenze richieste per l'esercizio della professione. Il livello di competenza è definito in funzione della complessità, dell'estensione dei compiti svolti, del livello di responsabilità e di autonomia decisionale che caratterizza la professione; il campo di competenza coglie, invece, le differenze nei domini settoriali, negli ambiti disciplinari delle conoscenze applicate, nelle attrezzature utilizzate, nei materiali lavorati, nel tipo di bene prodotto o servizio erogato nell'ambito della professione. La Nomenclatura delle Unità Professionali si inserisce tra gli interventi finalizzati alla definizione e alla messa a punto di un sistema nazionale di osservazione dei fabbisogni professionali~\cite{LG2019a}\mioindex{2019}{Nomenclatura!delle Unità professionali}.}}
\newglossaryentry{obiettivispecificiapprendimento}{name={Obiettivi specifici di apprendimento},description={Gli obiettivi specifici di apprendimento (OSA) indicano le conoscenze (il sapere) e le abilità (il saper fare) che tutte le scuole del territorio nazionale sono tenute ad utilizzare per progettare e organizzare autonomamente i piani di studio personalizzati che aiutino a trasformare le capacità di ciascun alunno in competenze~\cite{CIRC2005a}.}}
\newglossaryentry{organismonazionaleitalianodiaccreditamento}{name={Organismo nazionale italiano di accreditamento},description={Organismo nazionale di accreditamento designato dall'Italia in attuazione del regolamento (CE) n. 765/2008 del Parlamento europeo e del Consiglio del 9 luglio 2008~\cite{DL2013}.\mioindex{2008}{Organismo!nazionale italiano di accreditamento}}}
\newglossaryentry{organizzazionesettorialeinternazionale}{name={Organizzazione settoriale internazionale},description={Associazione di organizzazioni nazionali, anche, ad esempio, di datori di lavoro e organismi professionali, che rappresenta gli interessi di settori nazionali~\cite{RA2008}.\mioindex{2008}{Organizzazione!settoriale internazionale}}}
\newglossaryentry{orientamento}{name={Orientamento},description={Un processo continuativo che consente alle persone di identificare le proprie capacità, competenze e interessi attraverso una serie di attività individuali e collettive che servono a prendere decisioni in materia di istruzione, formazione e occupazione e a gestire i propri percorsi personali nell'ambito dell'istruzione, del lavoro e in altri contesti in cui è possibile acquisire o sfruttare tali capacità e competenze~\cite{DE2018}.\mioindex{2018}{Orientamento}}}
\newglossaryentry{percorsiIeFPigg}{name={Percorsi di IeFP},description={I percorsi di istruzione e formazione professionale per il conseguimento di qualifiche e diplomi professionali di cui al Capo III del decreto legislativo 17 ottobre 2005, n. 226~\cite{DL2018a}.\mioindex{2018}{Percorsi!IeFP}}}
\newglossaryentry{pianistudiopersonalizzati}{name={Piani di studio personalizzati},description={Il Piano di Studi personalizzato è l’insieme delle Unità di apprendimento concretamente realizzate, sia nel tempo scuola obbligatorio sia in quello opzionale facoltativo, e rappresenta il progetto realizzato dall’équipe pedagogica, in cooperazione con le famiglie e gli stessi alunni, per l’educazione di ciascuno. Ha come punto di riferimento obbligato le competenze espresse nel Profilo educativo, culturale e professionale dello studente alla fine del primo ciclo, che vengono promosse a partire dalle capacità di quegli alunni, in quel determinato contesto, modellando in obiettivi formativi gli obiettivi specifici di apprendimento elencati nelle Indicazioni nazionali. Il piano di studio è impostato nelle sue linee generali all’inizio dell’anno scolastico, tenendo conto anche di tutti gli apprendimenti non formali e informali acquisiti dagli alunni, ma si definisce riflessivamente e compiutamente solo durante e al termine delle attività realizzate~\cite{CIRC2005a}.\mioindex{2005}{Piani!studio personalizzati}}}
\newglossaryentry{piattaformaonline}{name={Piattaforma online},description={Un'applicazione basata sul web che fornisce informazioni e strumenti agli utenti finali e permette loro di portare a termine compiti specifici online~\cite{DE2018}.\mioindex{2005}{Piattaforma!online}}}
\newglossaryentry{professionalitdellavoro}{name={Professionalità del lavoro},description={Risiede nell'assumere responsabilità in riferimento ad uno scopo definito e nella capacità di apprendere anche dall'esperienza, ovvero di trovare soluzioni creative ai problemi sempre nuovi che si pongono~\cite{DL2017a}.\mioindex{2017}{Professionalità!lavoro}}}
\newglossaryentry{professioneregolamentata}{name={Professione regolamentata},description={Attività, o insieme di attività professionali, l'accesso alle quali e il cui esercizio, o una delle cui modalità di esercizio, sono subordinati direttamente o indirettamente, in forza di norme legislative, regolamentari o amministrative, al possesso di determinate qualifiche professionali; in particolare costituisce una modalità di esercizio l'impiego di un titolo professionale riservato da disposizioni legislative, regolamentari o amministrative a chi possiede una specifica qualifica professionale, Quando non si applica la prima frase, è assimilata ad una professione regolamentata una professione di cui al paragrafo 2~\cite{DI2005}.\mioindex{2005}{Professione!regolamentata}}}
\newglossaryentry{profiloprofessionalegg}{name={Profilo professionale},description={Insieme dei contenuti «tipici» delle funzioni/mansioni di una specifica categoria di professioni omogenee rispetto a competenze, abilità, conoscenze ed attività lavorative svolte~\cite{DL2018a}.\mioindex{2018}{Profilo!professionale}}}
\newglossaryentry{profilouscitaciascunindirizzoigg}{name={Profilo di uscita di ciascun indirizzo},description={Profilo formativo inteso come standard formativo in uscita dagli indirizzi di studio, quale insieme compiuto e riconoscibile di competenze descritte secondo una prospettiva di validità e spendibilità in molteplici contesti lavorativi del settore economico-professionale correlato~\cite{DL2018a}.\mioindex{2018}{Profilo!uscita di ciascun indirizzo}}}
\newglossaryentry{progettoformativoindividuale}{name={Progetto formativo individuale},description={\begin{tabular}{cp{12cm}}\toprule 2017 & Redatto dal Consiglio di classe entro il 31 gennaio del primo anno di frequenza, In esso sono evidenziati i saperi e le competenze acquisiti dallo studente anche in modo non formale e informale, ai fini di un apprendimento personalizzato, idoneo a consentirgli di proseguire con successo, anche attraverso l'esplicitazione delle sue motivazioni allo studio, le aspettative per le scelte future, le difficoltà incontrate e le potenzialità rilevate~\cite{DL2017a}\mioindex{2017}{Progetto!formativo individuale}.\\\midrule 2018&Progetto che ha il fine di motivare e orientare la studentessa e lo studente nella progressiva costruzione del proprio percorso formativo e lavorativo, di supportarli per migliorare il successo formativo e di accompagnarli negli eventuali passaggi tra i sistemi formativi di cui all'articolo 8 del decreto legislativo 13 aprile 2017, n. 61, con l'assistenza di un tutor individuato all'interno del consiglio di classe, Il progetto formativo individuale si basa sul bilancio personale, è effettuato nel primo anno di frequenza del percorso di istruzione professionale ed è aggiornato per tutta la sua durata\mioindex{2018}{Progetto!formativo individuale}~\cite{DL2018a}. \\\bottomrule\end{tabular}}}
\newglossaryentry{provaattitudinale}{name={Prova attitudinale},description={Una verifica riguardante le conoscenze, le abilità e le competenze professionali del richiedente, effettuata o riconosciuta dalle autorità competenti dello Stato membro ospitante allo scopo di valutare l'idoneità del richiedente a esercitare in tale Stato membro una professione regolamentata. Per consentire che la verifica sia effettuata, le autorità competenti predispongono un elenco delle materie che, in base a un confronto tra la formazione e l'istruzione richiesta nello Stato membro ospitante e quella ricevuta dal richiedente, non sono coperte dal diploma o dai titoli di formazione del richiedente. La prova attitudinale deve tener conto del fatto che il richiedente è un professionista qualificato nello Stato membro d'origine o di provenienza. Essa verte su materie da scegliere tra quelle che figurano nell'elenco e la cui conoscenza è essenziale per poter esercitare la professione in questione nello Stato membro ospitante, Tale prova può altresì comprendere la conoscenza delle regole professionali applicabili alle attività in questione nello Stato membro ospitante. Le modalità dettagliate della prova attitudinale nonché lo status di cui gode, nello Stato membro ospitante, il richiedente che desidera prepararsi alla prova attitudinale in detto Stato membro sono determinate dalle autorità competenti di detto Stato membro\mioindex{2013}{Prova!attitudinale}~\cite{DI2013}.}}
\newglossaryentry{quadronazionalequalifiche}{name={Quadro nazionale delle qualifiche},description={Strumento di classificazione delle qualifiche in funzione di una serie di criteri basati sul raggiungimento di livelli di apprendimento specifici; esso mira a integrare e coordinare i sottosistemi nazionali delle qualifiche e a migliorare la trasparenza, l'accessibilità, la progressione e la qualità delle qualifiche rispetto al mercato del lavoro e alla società civile\mioindex{2008}{Quadro!nazionale delle qualifiche}\mioindex{2012}{Quadro!nazionale delle qualifiche}\mioindex{2017}{Quadro!nazionale delle qualifiche}~\cite{RA2017}~\cite{RA2012}~\cite{RA2008}.}}
\newglossaryentry{qualificainternazionale}{name={Qualifica internazionale},description={Qualifica, rilasciata da un organismo internazionale legalmente costituito (associazione, organizzazione, settore o impresa) o da un organismo nazionale che agisce a nome di un organismo internazionale, che è utilizzata in più di un paese e include i risultati dell'apprendimento, valutati facendo riferimento alle norme stabilite da un organismo internazionale\mioindex{2017}{Qualifica!internazionale}~\cite{RA2017}.}}
\newglossaryentry{qualificaprofessionale}{name={Qualifica professionale},description={La qualifica attestata da un titolo di formazione, un attestato di competenza e/o un'esperienza professionale\mioindex{2005}{Qualifica!professionale}~\cite{DI2005}.}}
\newglossaryentry{qualificazioneinternazionale}{name={Qualificazione internazionale},description={Qualificazione rilasciata da un organismo internazionale legalmente costituito o da un organismo nazionale che agisce a nome di un organismo internazionale, che è utilizzata in più di un Paese e include i risultati di apprendimento, valutati facendo riferimento alle norme stabilite da un organismo internazionale~\cite{DL2018}\mioindex{2018}{Qualificazione!internazionale}.}}
\newglossaryentry{qualificazione}{name={Qualificazione},description={Titolo di istruzione e di formazione, ivi compreso quello di istruzione e formazione professionale, o di qualificazione professionale rilasciato da un ente pubblico titolato nel rispetto delle norme generali, dei livelli essenziali delle prestazioni e degli standard minimi~\cite{DL2013}~\cite{DL2018a}.\mioindex{2013}{Qualificazione}\mioindex{2018}{Qualificazione}}}
\newglossaryentry{qualifica}{name={Qualifica},description={Risultato formale di un processo di valutazione e convalida, acquisito quando un'autorità competente stabilisce che una persona ha conseguito i risultati dell'apprendimento rispetto a standard predefiniti\mioindex{2017}{Qualifica}\mioindex{2018}{Qualifica}\mioindex{2012}{Qualifica}\mioindex{2018}{Qualifica}~\cite{RA2017}~\cite{DE2018}~\cite{RA2012}~\cite{RA2008}.}}
\newglossaryentry{readingliteracy}{name={Reading literacy},description={Competenza in lettura}}
\newglossaryentry{referenziazione}{name={Referenziazione},description={Il processo istituzionale e tecnico che associa le qualificazioni rilasciate nell'ambito del Sistema nazionale di certificazione delle competenze a uno degli otto livelli del QNQ. La referenziazione delle qualificazioni italiane al garantisce la referenziazione delle stesse al Quadro europeo delle qualifiche\mioindex{2018}{Referenziazione}~\cite{DL2018}.}}
\newglossaryentry{responsabilitaeautonomia}{name={Responsabilità e autonomia},description={Capacità del discente di applicare le conoscenze e le abilità in modo autonomo e responsabile\mioindex{2017}{Responsabilità e autonomia}~\cite{RA2017}.}}
\newglossaryentry{riconoscimentoformazioneprecedente}{name={Riconoscimento della formazione precedente},description={Convalida dei risultati di apprendimento, nel quadro dell'istruzione formale o dell'apprendimento non formale o informale, acquisiti prima della richiesta di convalida\mioindex{2012}{Riconoscimento!della formazione precedente}~\cite{RA2012}.}}
\newglossaryentry{riconoscimento}{name={Riconoscimento formale dei risultati dell'apprendimento},description={Processo in base al quale un'autorità competente dà valore ufficiale ai risultati dell'apprendimento acquisiti a fini di studi ulteriori o di occupazione, mediante i) il rilascio di qualifiche (certificati, diplomi o titoli), ii) la convalida dell'apprendimento non formale e informale, iii) il riconoscimento di equivalenze, il rilascio di crediti o la concessione di deroghe\mioindex{2012}{Riconoscimento!formale dei risultati dell'apprendimento}~\cite{RA2017}.}}
\newglossaryentry{rilevazioneaccertamentocompetenze}{name={Rilevazione e accertamento delle competenze},description={Accertare e certificare la competenza di una persona richiede strumenti caratterizzati da accuratezza e attendibilità che, a differenza di quelli utilizzati per valutare soltanto la padronanza delle conoscenze e delle abilità, eccedono, senza escluderle, le consuete modalità valutative scolastiche disciplinari (test, prove oggettive, interrogazioni, saggi brevi, ecc.), ma richiedono anche osservazioni sistematiche prolungate nel tempo, valutazioni collegiali dei docenti che coinvolgano anche attori esterni alla scuola, a partire dalla famiglia, autovalutazioni dell’allievo, diari, storie fotografiche e filmati, coinvolgimento di esperti e simili. Il livello di accettabilità della competenza manifestata in situazione scaturisce dalla somma di queste condivisioni e coinvolge nella maniera professionalmente più alta i docenti che si assumono la responsabilità di certificarla\mioindex{2005}{Rilevazione e accertamento!delle competenze}\mioindex{2005}{Competenze!rilevazione e accertamento}~\cite{CIRC2005a}.}}
\newglossaryentry{risorseeducativeaperte}{name={Risorse educative aperte (OER)},description={Materiale digitalizzato messo gratuitamente e liberamente a disposizione di docenti, studenti, e chiunque studi in maniera autonoma, per l'uso e il riuso nell'insegnamento, l'apprendimento e la ricerca; esse comprendono materiale didattico, strumenti informatici per lo sviluppo, l'uso e la diffusione dei contenuti, e risorse per l'applicazione come le licenze aperte; le OER fanno anche riferimento a una somma di beni digitali che possono essere modificati e che offrono vantaggi senza che ne sia limitata la possibilità di utilizzo da parte di altri\mioindex{2012}{Risorse!educative aperte}~\cite{RA2012}.}}
\newglossaryentry{risultatidellapprendimento}{name={Risultati dell'apprendimento},description={\begin{tabular}{cp{12cm}}\toprule 2008-2017 & Descrizione di ciò che un discente conosce, capisce ed è in grado di realizzare al termine di un processo di apprendimento; sono definiti in termini di conoscenze, abilità e responsabilità e autonomia\mioindex{2008}{Risultati dell'apprendimento}\mioindex{2017}{Risultati dell'apprendimento}~\cite{RA2017}~\cite{RA2008}.\\\midrule 2012&Descrizione di ciò che un discente conosce, capisce ed è in grado di realizzare al termine di un processo di apprendimento definito in termini di conoscenze, abilità e competenze\mioindex{2012}{Risultati dell'apprendimento}~\cite{RA2012}. \\\bottomrule\end{tabular}}}
\newglossaryentry{settore}{name={Settore},description={Raggruppamento di attività professionali in base a funzione economica, prodotto, servizio o tecnologia principali\mioindex{2008}{Settore}~\cite{RA2008}.}}
\newglossaryentry{sistemaeuropeoaccumulazione}{name={Sistema europeo di accumulazione e trasferimento dei crediti o crediti ECTS},description={Il sistema di crediti per l'istruzione superiore utilizzato nello Spazio europeo dell'istruzione superiore\mioindex{2013}{Sistema!europeo di accumulazione e trasferimento dei crediti o crediti ECTS}~\cite{DI2013}.}}
\newglossaryentry{sistemanazionalecertificazionecompetenze}{name={Sistema nazionale di certificazione delle competenze},description={l'insieme dei servizi di individuazione e validazione e certificazione delle competenze erogati nel rispetto delle norme generali, dei livelli essenziali delle prestazioni e degli standard minimi\mioindex{2013}{Sistema!nazionale di certificazione delle competenze}\mioindex{2018}{Sistema!nazionale di certificazione delle competenze}~\cite{DL2013}~\cite{DL2018a}.}}
\newglossaryentry{sistemanazionaledellequalifiche}{name={Sistema nazionale delle qualifiche},description={Complesso delle attività di uno Stato membro connesse con il riconoscimento dell'apprendimento e altri meccanismi che mettono in relazione istruzione e formazione al mercato del lavoro e alla società civile. Ciò comprende l'elaborazione e l'attuazione di disposizioni e processi istituzionali in materia di garanzia della qualità, valutazione e rilascio delle qualifiche, Un sistema nazionale delle qualifiche può essere composto da vari sottosistemi e può comprendere un quadro nazionale delle qualifiche\mioindex{2017}{Sistema!nazionale delle qualifiche}\mioindex{2008}{Sistema!nazionale delle qualifiche}~\cite{RA2017}~\cite{RA2008}.}}
\newglossaryentry{sistemidicrediti}{name={Sistemi di crediti},description={Strumenti di trasparenza volti ad agevolare il riconoscimento dei crediti, Tali sistemi possono comprendere tra l'altro equivalenze, esenzioni, possibilità di accumulare e trasferire unità/moduli, autonomia degli erogatori che possono personalizzare i percorsi nonché convalida dell'apprendimento non formale e informale\mioindex{2017}{Sistemi!crediti}~\cite{RA2017}.}}
\newglossaryentry{softskill}{name={Softskill},description={Competenze trasversali e trasferibili attraverso la dimensione operativa del fare: Capacità di interagire e lavorare con gli altri, capacità di risoluzione di problemi, creatività, pensiero critico, consapevolezza, resilienza e capacità di individuare le forme di orientamento e sostegno disponibili per affrontare la complessità e l'incertezza dei cambiamenti, preparandosi alla natura mutante delle economie moderne e delle società complesse\mioindex{2019}{Softskill}~\cite{LG2019}}}
\newglossaryentry{stakeholder}{name={Stakeholder},description={Tutti i soggetti, individui od organizzazioni, attivamente coinvolti in un’iniziativa economica (progetto, azienda), il cui interesse è negativamente o positivamente influenzato dal risultato dell’esecuzione, o dall’andamento, dell’iniziativa e la cui azione o reazione a sua volta influenza le fasi o il completamento di un progetto o il destino di un’organizzazione\mioindex{2019}{Stakeholder}~\cite{Wikipedia2019}}}
\newglossaryentry{standardformativoregionale}{name={Standard formativo regionale},description={Regolamentazione regionale in materia di IeFP che, nel rispetto dei livelli essenziali delle prestazioni di cui al capo III del decreto legislativo n. 226 del 2005, definisce in particolare: a) la durata, l'articolazione e gli obiettivi dei percorsi di IeFP; b) le modalità per l'effettuazione delle prove finali di accertamento degli allievi e di certificazione finale e intermedia delle competenze acquisite anche in contesti non formali e informali, nonchè di riconoscimento dei crediti, spendibili nel sistema di istruzione, formazione e lavoro; c) la modulazione temporale tra attività formativa e alternanza scuola lavoro nonchè dell'apprendistato ai sensi dell'art. 43 del decreto legislativo n. 81 del 2015\mioindex{2018}{Standard!formativo regionale}~\cite{DL2018b}.}}
\newglossaryentry{standardminimiprocesso}{name={Standard minimi di processo },description={1, Con riferimento al processo di individuazione e validazione e alla procedura di certificazione, l'ente pubblico titolare assicura quali standard minimi: a) l'articolazione nelle seguenti fasi: 1) identificazione: fase finalizzata a individuare e mettere in trasparenza le competenze della persona riconducibili a una o più qualificazioni; in caso di apprendimenti non formali e informali questa fase implica un supporto alla persona nell'analisi e documentazione dell'esperienza di apprendimento e nel correlarne gli esiti a una o più qualificazioni; 2) valutazione: fase finalizzata all'accertamento del possesso delle competenze riconducibili a una o più qualificazioni; nel caso di apprendimenti non formali e informali questa fase implica l'adozione di specifiche metodologie valutative e di riscontri e prove idonei a comprovare le competenze effettivamente possedute; 3) attestazione: fase finalizzata al rilascio di documenti di validazione o certificati, standardizzati ai sensi del presente decreto, che documentano le competenze individuate e validate o certificate riconducibili a una o più qualificazioni; b) l'adozione di misure personalizzate di informazione e orientamento in favore dei destinatari dei servizi di individuazione e validazione e certificazione delle competenze\mioindex{2013}{Standard!minimi di processo}~\cite{DL2013}.}}
\newglossaryentry{standardminimiservizio}{name={Standard minimi di servizio},description={Gli standard minimi di servizio costituiscono livelli essenziali delle prestazioni da garantirsi su tutto il territorio nazionale, anche in riferimento alla individuazione e validazione degli apprendimenti non formali e informali e al riconoscimento dei crediti formativi, Gli standard minimi di servizio costituiscono riferimento per gli enti pubblici titolari nella definizione di standard minimi di erogazione dei servizi da parte degli enti titolati\mioindex{2013}{Standard!minimi di servizio}~\cite{DL2013}.}}
\newglossaryentry{supplementi Europass}{name={Supplementi Europass},description={Una serie di documenti, come ad esempio i supplementi al diploma e i supplementi al certificato, rilasciati dalle autorità od organismi competenti~\cite{DE2018}.}}
\newglossaryentry{supplementoaldiploma}{name={Supplemento al diploma},description={Un documento allegato a un diploma di istruzione superiore rilasciato dalle autorità od organismi competenti allo scopo di facilitare la comprensione da parte di terzi — soprattutto in un altro paese — dei risultati di apprendimento ottenuti dal titolare della qualifica come pure della natura, del livello, del contesto, del contenuto e dello status dell'istruzione e della formazione completata e delle competenze acquisite\mioindex{2018}{Supplemento!al diploma}~\cite{DE2018}.}}
\newglossaryentry{supplementocertificato}{name={Supplemento al certificato},description={Un documento accluso a un certificato di istruzione e formazione professionale o a un certificato professionale rilasciato dalle autorità od organismi competenti allo scopo di facilitare la comprensione da parte di terzi — soprattutto in un altro paese — dei risultati di apprendimento ottenuti dal titolare della qualifica come pure della natura, del livello, del contesto, del contenuto e dello status dell'istruzione e della formazione completata e delle competenze acquisite\mioindex{2018}{Supplemento!al certificato}~\cite{DE2018}.}}
\newglossaryentry{tesseraprofessionaleeuropea}{name={Tessera professionale europea},description={Un certificato elettronico attestante o che il professionista ha soddisfatto tutte le condizioni necessarie per fornire servizi, su base temporanea e occasionale, in uno Stato membro ospitante o il riconoscimento delle qualifiche professionali ai fini dello stabilimento in uno Stato membro ospitante\mioindex{2013}{Tessera!professionale europea}~\cite{DI2013}.}}
\newglossaryentry{tirociniodiadattamento}{name={Tirocinio di adattamento},description={l'esercizio di una professione regolamentata nello Stato membro ospitante sotto la responsabilità di un professionista qualificato, accompagnato eventualmente da una formazione complementare, Il tirocinio è oggetto di una valutazione. Le modalità del tirocinio di adattamento e della sua valutazione nonché lo status di tirocinante migrante sono determinati dalle autorità competenti dello Stato membro ospitante. Lo status di cui il tirocinante gode nello Stato membro ospitante, soprattutto in materia di diritto di soggiorno nonché di obblighi, diritti e benefici sociali, indennità e retribuzione, è stabilito dalle autorità competenti di detto Stato membro conformemente al diritto comunitario applicabile\mioindex{2005}{Tirocinio!adattamento}~\cite{DI2005}.}}
\newglossaryentry{tirocinioprofessionale}{name={Tirocinio professionale},description={Fatto salvo l'articolo 46, paragrafo 4, un periodo di pratica professionale effettuato sotto supervisione, purché costituisca una condizione per l'accesso a una professione regolamentata e che può svolgersi durante o dopo il completamento di un'istruzione che conduce a un diploma\mioindex{2013}{Tirocinio!professionale}~\cite{DI2013}.}}
\newglossaryentry{titoloformazione}{name={Titolo di formazione},description={Diplomi, certificati e altri titoli rilasciati da un'autorità di uno Stato membro designata ai sensi delle disposizioni legislative, regolamentari e amministrative di tale Stato membro e che sanciscono una formazione professionale acquisita in maniera preponderante nella Comunità~\cite{DI2005}.}}
\newglossaryentry{traghettamento}{name={Traghettamento},description={Con apposito Protocollo una Regione o PA può trasferire un intero Repertorio di qualificazioni professionali e di standard di certificazione di un’altra Regione e traghettarlo nella propria\mioindex{}{Traghettamento}.}}
\newglossaryentry{trasferimentodicrediti}{name={Trasferimento di crediti},description={Processo che consente ai soggetti che hanno accumulato crediti in un contesto di farli valutare e riconoscere in un altro contesto\mioindex{2017}{Trasferimento!crediti}~\cite{RA2017}.}}
\newglossaryentry{unitapprendimento}{name={Unità di apprendimento},description={\begin{tabular}{cp{12cm}}\toprule 2005 & Dopo aver identificato l’apprendimento unitario da promuovere (ad esempio, un campo unitario e significativo di esperienze e di possibile competenza, problemi da risolvere, compiti da eseguire o progetti da realizzare, ecc.), l’unità di apprendimento precisa gli obiettivi formativi coinvolti, gli itinerari educativi e didattici ritenuti necessari per raggiungerli e i compiti unitari in situazione che, osservati e analizzati, possono alla fine documentare il perseguimento degli obiettivi formativi posti. L’unità di apprendimento sottintende il principio che l’unico insegnamento efficace è quello che si trasforma in apprendimento degli allievi, e che ogni apprendimento significativo non è mai parziale o segmentato, ma sempre unitario, nel senso che sollecita tutte le dimensioni della persona e coinvolge più prospettive disciplinari\mioindex{2005}{Unità!di apprendimento}~\cite{CIRC2005a}.\\\midrule 2018&Insieme autonomamente significativo di competenze, abilità e conoscenze in cui è organizzato il percorso formativo della studentessa e dello studente; costituisce il necessario riferimento per la valutazione, la certificazione e il riconoscimento dei crediti, soprattutto nel caso di passaggi ad altri percorsi di istruzione e formazione. Le UdA partono da obiettivi formativi adatti e significativi, sviluppano appositi percorsi di metodo e di contenuto, tramite i quali si valuta il livello delle conoscenze e delle abilità acquisite e la misura in cui la studentessa e lo studente hanno maturato le competenze attese\mioindex{2018}{Unità!di apprendimento}~\cite{DL2018a}. \\\bottomrule\end{tabular}}}
\newglossaryentry{valutazionecomportamento}{name={Valutazione del comportamento},description={La valutazione del comportamento si riferisce allo sviluppo delle competenze di cittadinanza. Lo Statuto delle studentesse e degli studenti, il Patto educativo di corresponsabilità e i regolamenti approvati dalle istituzioni scolastiche ne costituiscono i riferimenti essenziali~\cite{DL2017}\mioindex{2017}{Valutazione!del comportamento}.}}
\newglossaryentry{valutazionedellecompetenze}{name={Valutazione delle competenze},description={Il processo o il metodo utilizzato per valutare, misurare e infine descrivere, mediante l'autovalutazione o la valutazione certificata da terzi o entrambe, le competenze individuali acquisite in contesti formali, non formali o informali\mioindex{2018}{Valutazione!delle competenze}~\cite{DE2018}.}}
\newglossaryentry{valutazioneg}{name={Valutazione},description={La valutazione ha per oggetto il processo formativo e i risultati di apprendimento delle alunne e degli alunni, delle studentesse e degli studenti delle istituzioni scolastiche del sistema nazionale di istruzione e formazione, ha finalità formativa ed educativa e concorre al miglioramento degli apprendimenti e al successo formativo degli stessi, documenta lo sviluppo dell'identità personale e promuove la autovalutazione di ciascuno in relazione alle acquisizioni di conoscenze, abilità e competenze~\cite{DL2017}\mioindex{2017}{Valutazione}.}}
\newglossaryentry{isfolg}{name={Isfol},description={Istituto per lo sviluppo della formazione professionale dei lavoratori  -  è un ente nazionale di ricerca sottoposto alla vigilanza del Ministero del Lavoro e delle politiche sociali. Dal 1//12/2016 è diventato INAPP~\cite{Isfol1972}}}