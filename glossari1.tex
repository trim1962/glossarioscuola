%\newglossaryentry{fattoreng}{name={Fattore},description={unità di misura 
%dell'angolo che non fa parte del SI. Il gon è pari a 1/400 parte dell'angolo 
%giro.}}
%\newglossaryentry{gong}{name={Gon},description={}}

%\newglossaryentry{Ag}{type=symbols,name={A},description={Il numero dieci in 
%base 16}}
\newglossaryentry{readingliteracy}{name={Reading literacy},description={Competenza in lettura}}
\newacronym{PISA}{PISA}{Programme for International Student Assessment}
\newacronym{POF}{POF}{Piano 
dell'offerta formativa~\cite{MIUR2005}}
\newacronym{PCTO}{PCTO}{Percorsi per le competenze trasversali e per l’orientamento~\cite{MIUR2019}}
\newacronym{QCER}{QCER}{Quadro comune di riferimento per la conoscenza delle lingue in ambito europeo~\cite{MIUR2019}}
\newacronym{EQF}{EQF}{Quadro europeo delle qualificazioni~\cite{MIUR2019}}
\newacronym{Europass}{Europass}{Passaporto europeo delle competenze~\cite{MIUR2019}}
\newacronym{T-VET}{T-VET}{Technical and Vocational Education and Training~\cite{MIUR2019}}
\newacronym{EQARF}{EQARF}{Quadro europeo per la garanzia della qualità del sistema di istruzione e formazione professionale~\cite{MIUR2019}}
\newacronym{ECVET}{ECVET}{Sistema europeo di crediti per la VET~\cite{MIUR2019}}
\newacronym{ECTS}{ECTS}{Sistema europeo di
trasferimento e accumulo dei crediti~\cite{MIUR2019}}
\newacronym{EAfA}{EAfA}{Alleanza europea per l’apprendistato~\cite{MIUR2019}}
\newacronym{BES}{BES}{Bisogni educativi speciali}
\newglossaryentry{Soft
	skill}{name={Softskill},description={Competenze trasversali e trasferibili attraverso la dimensione operativa del fare:
		Capacità di interagire e lavorare con gli altri, capacità di risoluzione di problemi, creatività, pensiero critico, consapevolezza, resilienza e capacità di individuare le forme di orientamento e sostegno disponibili per affrontare la complessità e l’incertezza dei cambiamenti, preparandosi alla natura mutante delle economie moderne e delle società complesse~\cite{MIUR2019}}}
\newacronym{EntreComp}{EntreComp}{Quadro europeo della competenza imprenditorialità~\cite{MIUR2019}}	
\newacronym{OCSE}{OCSE}{Organizzazione per la cooperazione e lo sviluppo economico}
\newacronym{CTS}{CTS}{Comitato Tecnico Scientifico~\cite{MIUR2019}}
\newacronym{CS}{CS}{Comitato Scientifico~\cite{MIUR2019}}
\newglossaryentry{Compitidirealt}{name={Compiti di realtà},description={I compiti di realtà si identificano nella richiesta rivolta allo studente di risolvere una situazione problematica, complessa e nuova, quanto più possibile vicina al mondo reale, utilizzando conoscenze e abilità già acquisite e trasferendo procedure e condotte cognitive in contesti e ambiti di
		riferimento moderatamente diversi da quelli resi familiari dalla pratica didattica. Pur non escludendo prove che chiamino in causa una sola disciplina, si ritiene opportuno privilegiare prove per la cui
		risoluzione l’alunno debba richiamare in forma integrata, componendoli autonomamente, più apprendimenti acquisiti~\cite{MIUR2018}}}