%\newglossaryentry{Ag}{type=symbols,name={A},description={Il numero dieci in 
%\newglossaryentry{fattoreng}{name={Fattore},description={unità di misura 
%\newglossaryentry{gong}{name={Gon},description={}}
%base 16}}
%dell'angolo che non fa parte del SI. Il gon è pari a 1/400 parte dell'angolo 
%giro.}}
\newacronym{ANPAL}{ANPAL}{Agenzia Nazionale per le Politiche Attive del Lavoro~\cite{DL2015}}
\newacronym{ANVUR}{ANVUR}{Agenzia nazionale di valutazione del sistema  universitario  e della  ricerca}
\newacronym{AgID}{AgID}{Agenzia per l'Italia digitale}
\newacronym{BES}{BES}{Bisogni educativi speciali}
\newacronym{CBT}{CBT}{Computer Based Test}
\newacronym{CCR}{CCR}{Closed-Constructed Response, Aperti con risposta univoca}
\newacronym{CIE}{CIE}{Carta di identità elettronica}
\newacronym{CMC}{CMC}{Complex Multiple Choice, Scelta multipla complessa}
\newacronym{CR}{CR}{Constructed Response, Aperti con risposta articolata}
\newacronym{CS}{CS}{Comitato Scientifico~\cite{LG2019}}
\newacronym{CTS}{CTS}{Comitato Tecnico Scientifico~\cite{LG2019}}
\newacronym{DPIA}{DPIA}{Data Protecion Impact Assessment}
\newacronym{DPO}{DPO}{Data Protection Officer}
\newacronym{DSA}{DSA}{Disturbi Specifici dell’Apprendimento}
\newacronym{EAfA}{EAfA}{Alleanza europea per l’apprendistato~\cite{LG2019}}
\newacronym{ECTS}{ECTS}{Sistema europeo di trasferimento e accumulo dei crediti~\cite{LG2019}}
\newacronym{ECVET}{ECVET}{Sistema europeo di crediti per la VET~\cite{LG2019}}
\newacronym{EDPB}{EDPB}{European Data Protection Board}
\newacronym{EQARF}{EQARF}{Quadro europeo per la garanzia della qualità del sistema di istruzione e formazione professionale~\cite{LG2019}}
\newacronym{EQF}{EQF}{Quadro europeo delle qualificazioni~\cite{RA2017}}
\newacronym{EntreComp}{EntreComp}{Quadro europeo della competenza imprenditorialità~\cite{LG2019}}	
\newacronym{Europass}{Europass}{Passaporto europeo delle competenze~\cite{LG2019},~\cite{DE2004}~\cite{DE2018}}
\newacronym{GDPR}{GDPR}{General Data Protection Regulation}
\newacronym{INAPP}{INAPP}{Istituto Nazionale per l’Analisi delle Politiche Pubbliche~\cite{DL2016}}
\newacronym{INVALSI}{INVALSI}{Istituto nazionale per  la  valutazione  del sistema educativo di istruzione e formazione}
\newacronym{ISFOL}{ISFOL}{Istituto per lo sviluppo della formazione professionale dei lavoratori~\cite{DL2015}}
\newacronym{ITC}{ITC}{Information and Communications Technology}
\newacronym{ITS}{ITS}{Istituto tecnico superiore}
\newacronym{OCSE}{OCSE}{Organizzazione per la cooperazione e lo sviluppo economico}
\newacronym{OER}{OER}{Open Educational Resources~\cite{RA2012}}
\newacronym{PCTO}{PCTO}{Percorsi per le competenze trasversali e per l’orientamento~\cite{LG2019}}
\newacronym{PDP}{PDP}{Piano Didattico Personalizzato}
\newacronym{PIAAC}{PIAAC}{Programme for the International Assessment of Adult Competencies~\cite{RA2018}}
\newacronym{PISA}{PISA}{Programme for International Student Assessment}
\newacronym{POF}{POF}{Piano dell'offerta formativa~\cite{CIRC2005}}
\newacronym{PPT}{PPT}{Pen and paper test,Penna e carta test}
\newacronym{PSN}{PSN}{Programma Statistico Nazionale}
\newacronym{QCER}{QCER}{Quadro comune di riferimento per la conoscenza delle lingue in ambito europeo~\cite{LG2019}}
\newacronym{QCGQ}{QCGQ}{Quadro comune di garanzia della qualità~\cite{RA2009}}
\newacronym{QNQ}{QNQ}{Quadro nazionale delle qualificazioni~\cite{DL2018}}
\newacronym{SDG}{SDG}{Sustainable Development Goals}
\newacronym{SISTAN}{SISTAN}{Sistema statistico nazionale}
\newacronym{SMC}{SMC}{Simple Multiple Choice, Scelta multipla semplice}
\newacronym{SPID}{SPID}{Sistema Pubblico di Identità Digitale}
\newacronym{STEM}{STEM}{Science, Technology, Engineering and Mathematics~\cite{RA2018}}
\newacronym{T-VET}{T-VET}{Technical and Vocational Education and Training~\cite{LG2019}}
\newacronym{TIC}{TIC}{Tecnologie di informazione e comunicazione}
\newacronym{UNESCO}{UNESCO}{United Nations Educational, Scientific and Cultural Organization}
\newacronym{WP29}{WP29}{Article 29 Working Party}
\newacronym{eIDAS}{eIDAS}{electronic IDentification Authentication and Signature}
\newacronym{eID}{eID}{electronic IDentification, identita digitale}
\newglossaryentry{Alternanzag}{name={Alternanza},description={Alternanza scuola-lavoro modalità di realizzazione dei corsi del secondo ciclo, sia nel sistema dei licei, sia nel sistema dell'istruzione e della formazione professionale, per assicurare ai giovani, oltre alle conoscenze di base, l'acquisizione di competenze spendibili nel mercato del lavoro.~\cite{DL2005}.}}
\newglossaryentry{AtlanteLavorodelleQualificazioni}{name={Atlante del lavoro e delle qualificazioni},description={Dispositivo classificatorio e informativo realizzato sulla base delle sequenze descrittive della Classificazione dei settori economico-professionali, anche ai sensi dell'art. 8 del decreto legislativo n. 13 del 2013 e dell'art. 3, comma 5, del decreto interministeriale del 30 giugno 2015, e parte integrante dei sistemi informativi di cui agli articoli 13 e 15 del decreto legislativo n. 150 del 2015~\cite{DL2018}.}}
\newglossaryentry{Competenzaalfabeticafunzionale}{name={Competenza alfabetica funzionale},description={La competenza alfabetica funzionale indica la capacità di individuare, comprendere, esprimere, creare e interpretare concetti, sentimenti, fatti e opinioni, in forma sia orale sia scritta, utilizzando materiali visivi, sonori e digitali attingendo a varie discipline e contesti. Essa implica l’abilità di comunicare e relazionarsi efficacemente con gli altri in modo opportuno e creativo.	Il suo sviluppo costituisce la base per l’apprendimento successivo e l’ulteriore interazione linguistica. A seconda del contesto, la competenza alfabetica funzionale può essere sviluppata nella lingua madre, nella lingua dell’istruzione scolastica e/o nella lingua ufficiale di un paese o di una regione~\cite{RA2018}.}}
\newglossaryentry{Competenzadigitale}{name={Competenza digitale},description={La competenza digitale presuppone l’interesse per le tecnologie digitali e il loro utilizzo con dimestichezza e spirito critico e responsabile per apprendere, lavorare e partecipare alla società. Essa comprende l’alfabetizzazione informatica e digitale, la comunicazione e la collaborazione, l’alfabetizzazione mediatica, la creazione di contenuti digitali (inclusa la programmazione), la sicurezza (compreso l’essere a proprio agio nel mondo digitale e possedere competenze relative alla cibersicurezza), le questioni legate alla proprietà intellettuale, la risoluzione di problemi e il pensiero critico~\cite{RA2018}.}}
\newglossaryentry{Competenzaimprenditoriale}{name={Competenza imprenditoriale},description={La competenza imprenditoriale si riferisce alla capacità di agire sulla base di idee e opportunità e di trasformarle in valori per gli altri. Si fonda sulla creatività, sul pensiero critico e sulla risoluzione di problemi, sull’iniziativa e sulla perseveranza, nonché sulla capacità di lavorare in modalità collaborativa al fine di programmare e gestire progetti che hanno un valore culturale, sociale o finanziario~\cite{RA2018}.}}
\newglossaryentry{Competenzainmateriaconsapevolezza}{name={Competenza in materia di consapevolezza\linebreak ed espressione culturali},description={La competenza in materia di consapevolezza ed espressione culturali implica la comprensione e il rispetto di come le idee e i significati vengono espressi creativamente e comunicati in diverse culture e tramite tutta una serie di arti e altre forme culturali. Presuppone l’impegno di capire, sviluppare ed esprimere le proprie idee e il senso della propria funzione o del proprio ruolo nella società in una serie di modi e contesti~\cite{RA2018}.}}
\newglossaryentry{Competenzainmateriadicittadinanza}{name={Competenza in materia di cittadinanza},description={La competenza in materia di cittadinanza si riferisce alla capacità di agire da cittadini responsabili e di partecipare pienamente alla vita civica e sociale, in base alla comprensione delle strutture e dei concetti sociali, economici, giuridici e politici oltre che dell’evoluzione a livello globale e della sostenibilità~\cite{RA2018}.}}
\newglossaryentry{Competenzamatematica}{name={Competenza matematica},description={La competenza matematica è la capacità di sviluppare e applicare il pensiero e la comprensione matematici per risolvere una serie di problemi in situazioni quotidiane. Partendo da una solida padronanza della competenza aritmetico-matematica, l’accento è posto sugli aspetti del processo e dell’attività oltre che sulla conoscenza. La competenza matematica comporta, a differenti livelli, la capacità di usare modelli matematici di pensiero e di presentazione (formule, modelli, costrutti, grafici, diagrammi) e la disponibilità a farlo.~\cite{RA2018}.}}
\newglossaryentry{Competenzamultilinguistica}{name={Competenza multilinguistica},description={Tale competenza definisce la capacità di utilizzare diverse lingue in modo appropriato ed efficace allo scopo di comunicare. In linea di massima essa condivide le abilità principali con la competenza alfabetica: si basa sulla capacità di comprendere, esprimere e interpretare concetti, pensieri, sentimenti, fatti e opinioni in forma sia orale sia scritta (comprensione orale, espressione orale, comprensione scritta ed espressione scritta) in una gamma appropriata di contesti sociali e culturali a seconda dei desideri o delle esigenze individuali. Le competenze linguistiche comprendono una dimensione storica e competenze interculturali. Tale competenza si basa sulla capacità di mediare tra diverse lingue e mezzi di comunicazione, come indicato nel quadro comune europeo di riferimento. Secondo le circostanze, essa può comprendere il mantenimento e l’ulteriore sviluppo delle competenze relative alla lingua madre, nonché l’acquisizione della lingua ufficiale o delle lingue ufficiali di un paese~\cite{RA2018}.}}
\newglossaryentry{Competenzapersonalesociale}{name={Competenza personale, sociale e capacità di imparare a imparare},description={La competenza personale, sociale e la capacità di imparare a imparare consiste nella capacità di riflettere su sé stessi, di gestire efficacemente il tempo e le informazioni, di lavorare con gli altri in maniera costruttiva, di mantenersi resilienti e di gestire il proprio apprendimento e la propria carriera. Comprende la capacità di far fronte all’incertezza e alla complessità, di imparare a imparare, di favorire il proprio benessere fisico ed emotivo, di mantenere la salute fisica e mentale, nonché di essere in grado di condurre una vita attenta alla salute e orientata al futuro, di empatizzare e di gestire il conflitto in un contesto favorevole e inclusivo~\cite{RA2018}.}}
\newglossaryentry{Compitidirealt}{name={Compiti di realtà},description={I compiti di realtà si identificano nella richiesta rivolta allo studente di risolvere una situazione problematica, complessa e nuova, quanto più possibile vicina al mondo reale, utilizzando conoscenze e abilità già acquisite e trasferendo procedure e condotte cognitive in contesti e ambiti di riferimento moderatamente diversi da quelli resi familiari dalla pratica didattica. Pur non escludendo prove che chiamino in causa una sola disciplina, si ritiene opportuno privilegiare prove per la cui risoluzione l’alunno debba richiamare in forma integrata, componendoli autonomamente, più apprendimenti acquisiti~\cite{LG2018}}}
\newglossaryentry{DPOg}{name={Data Protection Officer},description={Il DPO è un supervisore indipendente, il quale sarà designato obbligatoriamente, da soggetti apicali di tutte le pubbliche amministrazioni e nello specifico è previsto l’obbligo nel caso in cui “il trattamento è effettuato da un’autorità pubblica o da un organismo pubblico, eccettuate le autorità giurisdizionali quando esercitano le loro funzioni giurisdizionali”}}
\newglossaryentry{EDPBg}{name={Comitato europeo per la protezione dei dati},description={Il comitato europeo per la protezione dei dati è un organo europeo indipendente, che contribuisce all’applicazione coerente delle norme sulla protezione dei dati in tutta l’Unione europea e promuove la cooperazione tra le autorità competenti per la protezione dei dati dell’UE.}}
\newglossaryentry{GDPRg}{name={Regolamento GDPR},description={Regolamento Ue 2016/679, noto come GDPR (General Data Protection Regulation) – relativo alla protezione delle persone fisiche con riguardo al trattamento e alla libera circolazione dei dati personali~\cite{RE2016}.}}
\newglossaryentry{Lavalutazionecomportamento}{name={La valutazione del comportamento},description={La valutazione del comportamento si riferisce allo sviluppo delle competenze di cittadinanza. Lo Statuto delle studentesse e degli studenti, il Patto educativo di corresponsabilità e i regolamenti approvati dalle istituzioni scolastiche ne costituiscono i riferimenti essenziali~\cite{DL2017}.}}
\newglossaryentry{Lavalutazione}{name={La valutazione},description={La valutazione ha per oggetto il processo formativo e i risultati di apprendimento delle alunne e degli alunni, delle studentesse e degli studenti delle istituzioni scolastiche del sistema nazionale di istruzione e formazione, ha finalità formativa ed educativa e concorre al miglioramento degli apprendimenti e al successo formativo degli stessi, documenta lo sviluppo dell'identità personale e promuove la autovalutazione di ciascuno in relazione alle acquisizioni di conoscenze, abilità e competenze~\cite{DL2017}.}}
\newglossaryentry{QCGQg}{name={QCGQ},description={Il QCGQ per l'istruzione e la formazione professionale (nell'ambito del follow-up della dichiarazione di Copenaghen) e l'elaborazione di una serie concordata di standard, procedure e orientamenti per la garanzia di qualità (2) (in connessione con il processo di Bologna e nel contesto del programma di lavoro sugli obiettivi dei sistemi di istruzione e di formazione) dovrebbero essere priorità di primissimo piano per l'Europa~\cite{RA2009}.}}
\newglossaryentry{Sistemaeuropeoaccumulazione}{name={Sistema europeo di accumulazione e trasferimento dei crediti o crediti ECTS},description={Il sistema di crediti per l’istruzione superiore utilizzato nello Spazio europeo dell’istruzione superiore~\cite{DI2013}.}}
\newglossaryentry{Softskill}{name={Softskill},description={Competenze trasversali e trasferibili attraverso la dimensione operativa del fare: Capacità di interagire e lavorare con gli altri, capacità di risoluzione di problemi, creatività, pensiero critico, consapevolezza, resilienza e capacità di individuare le forme di orientamento e sostegno disponibili per affrontare la complessità e l’incertezza dei cambiamenti, preparandosi alla natura mutante delle economie moderne e delle società complesse~\cite{LG2019}}}
\newglossaryentry{Standardminimiprocesso }{name={Standard minimi di processo },description={1. Con riferimento al processo di individuazione  e  validazione  e alla procedura di certificazione, l'ente pubblico  titolare  assicura quali standard minimi: a) l'articolazione nelle seguenti fasi: 1) identificazione: fase finalizzata a individuare e mettere in trasparenza le competenze della persona riconducibili a  una  o  piu' qualificazioni; in caso di  apprendimenti  non  formali  e  informali questa  fase  implica  un  supporto  alla  persona   nell'analisi   e documentazione dell'esperienza di apprendimento e nel correlarne  gli esiti a una o piu' qualificazioni; 2) valutazione: fase finalizzata all'accertamento del  possesso delle competenze riconducibili a una o piu' qualificazioni; nel  caso di  apprendimenti  non  formali  e  informali  questa  fase   implica l'adozione di specifiche metodologie  valutative  e  di  riscontri  e prove idonei a comprovare le competenze effettivamente possedute; 3) attestazione: fase finalizzata al rilascio di  documenti  di validazione o  certificati,  standardizzati  ai  sensi  del  presente decreto, che documentano  le  competenze  individuate  e  validate  o certificate riconducibili a una o piu' qualificazioni; b)  l'adozione  di  misure  personalizzate  di   informazione   e orientamento in favore dei destinatari dei servizi di  individuazione e validazione e certificazione delle competenze~\cite{DL2013}.}}
\newglossaryentry{abilita}{name={Abilità},description={Capacità di applicare le conoscenze e di usare il know-how per portare a termine compiti e risolvere problemi. Nel contesto dell’EQF, le abilità sono descritte come cognitive (comprendenti l’uso del pensiero logico, intuitivo e creativo) e pratiche (comprendenti la manualità e l’uso di metodi, materiali, strumenti e utensili)~\cite{RA2017}.}}
\newglossaryentry{analisifabbisognocompetenze}{name={Analisi del fabbisogno di competenze},description={L'analisi quantitativa o qualitativa disponibile di dati aggregati sulle competenze da fonti esistenti relative al mercato del lavoro e delle corrispondenti opportunità di apprendimento nel sistema di istruzione e formazione, che può contribuire all'orientamento e alla consulenza, alle procedure di assunzione, alla scelta del percorso di studi, di formazione e di carriera professionale~\cite{DE2018}.}}
\newglossaryentry{apprendimentoformaleg}{name={Apprendimento formale},description={Qualsiasi attività intrapresa dalla persona in modo formale, non formale e informale, nelle varie fasi della vita, al fine di migliorare le conoscenze, le capacità e le competenze, in una prospettiva di crescita personale, civica, sociale e occupazionale~\cite{DL2013}.}}
\newglossaryentry{apprendimentoformale}{name={Apprendimento formale},description={Apprendimento erogato in un contesto organizzato e strutturato, specificamente dedicato all'apprendimento, che di norma porta all'ottenimento di qualifiche, generalmente sotto forma di certificati o diplomi; comprende sistemi di istruzione generale, formazione professionale iniziale e istruzione superiore~\cite{RA2012}.}}
\newglossaryentry{apprendimentoinformaleg}{name={Apprendimento informale},description={Apprendimento che, anche a prescindere da una scelta intenzionale, si realizza nello svolgimento, da parte di ogni persona, di attività nelle situazioni di vita quotidiana e nelle interazioni che in essa hanno luogo, nell'ambito del contesto di lavoro, familiare e del tempo libero~\cite{DL2013}.}}
\newglossaryentry{apprendimentoinformale}{name={Apprendimento informale},description={Apprendimento risultante dalle attività della vita quotidiana legate al lavoro, alla famiglia o al tempo libero e non strutturato in termini di obiettivi di apprendimento, di tempi o di risorse dell'apprendimento; esso può essere non intenzionale dal punto di vista del discente; esempi di risultati di apprendimento acquisiti mediante l'apprendimento informale sono le abilità acquisite durante le esperienze di vita e lavoro come la capacità di gestire progetti o le abilità ITC acquisite sul lavoro; le lingue e le abilità interculturali acquisite durante il soggiorno in un altro paese; le abilità ITC acquisite al di fuori del lavoro, le abilità acquisite nel volontariato, nelle attività culturali e sportive, nel lavoro, nell'animazione socio educativa e mediante attività svolte in casa (ad esempio l'accudimento dei bambini)~\cite{RA2012}.}}
\newglossaryentry{apprendimentononformaleg}{name={Apprendimento non formale},description={Apprendimento caratterizzato da una scelta intenzionale della persona, che si realizza al di fuori dei sistemi indicati alla lettera b), in ogni organismo che persegua scopi educativi e formativi, anche del volontariato, del servizio civile nazionale e del privato sociale e nelle imprese~\cite{DL2013}.}}
\newglossaryentry{apprendimentononformale}{name={Apprendimento non formale},description={Apprendimento erogato mediante attività pianificate (in termini di obiettivi e tempi di apprendimento) con una qualche forma di sostegno all'apprendimento (ad esempio la relazione studente-docente); può comprendere programmi per il conseguimento di abilità professionali, alfabetizzazione degli adulti e istruzione di base per chi ha abbandonato la scuola prematuramente; sono esempi tipici di apprendimento non formale la formazione impartita sul lavoro, mediante la quale le aziende aggiornano e migliorano le abilità dei propri dipendenti, come ad esempio le abilità relative alle tecnologie per l'informazione e la comunicazione (ITC), l'apprendimento strutturato online (ad esempio con l'uso di risorse educative aperte) e i corsi organizzati dalle organizzazioni della società civile per i loro aderenti, i gruppi interessati o il pubblico generale~\cite{RA2012}.}}
\newglossaryentry{apprendimentopermanenteg}{name={Apprendimento permanente},description={Qualsiasi attività intrapresa dalla persona in modo formale, non formale e informale, nelle varie fasi della vita, al fine di migliorare le conoscenze, le capacità e le competenze, in una prospettiva di crescita personale, civica, sociale e occupazionale~\cite{DL2013}.}}
\newglossaryentry{apprendimentopermanente}{name={Apprendimento permanente},description={L’intero complesso di istruzione generale, istruzione e formazione professionale, istruzione non formale e apprendimento informale intrapresi nel corso della vita che comporta un miglioramento delle conoscenze, delle abilità e delle competenze, che può includere l’etica professionale~\cite{DI2013}.}}
\newglossaryentry{autoritcompetente}{name={Autorità competente},description={Qualsiasi autorità o organismo abilitato da uno Stato membro in particolare a rilasciare o a ricevere titoli di formazione e altri documenti o informazioni, nonché a ricevere le domande e ad adottare le decisioni di cui alla presente direttiva~\cite{DI2005}.}}
\newglossaryentry{autovalutazionedellecompetenze}{name={Autovalutazione delle competenze},description={Il processo di riflessione sistematica della persona sulle proprie competenze tramite il riferimento a descrizioni fisse delle competenze~\cite{DE2018}.}}
\newglossaryentry{bilanciocompetenze}{name={Bilancio di competenze},description={Processo volto all'individuazione e all'analisi delle conoscenze, abilità e competenze di una persona, comprese attitudini e motivazioni, per definire un progetto professionale e/o pianificare un progetto di riorientamento o formazione professionale; lo scopo di un bilancio di competenze è di aiutare una persona ad analizzare il profilo professionale acquisito, a comprendere la propria posizione nel mondo del lavoro e a progettare una carriera professionale, o in taluni casi, a prepararsi in vista della convalida dei risultati dell'apprendimento non formale o informale~\cite{RA2012}.}}
\newglossaryentry{certificazionedellecompetenze}{name={Certificazione delle competenze},description={Procedura di formale riconoscimento, da parte dell'ente titolato, in base alle norme generali, ai livelli essenziali delle prestazioni e agli standard minimi di cui al presente decreto, delle competenze acquisite dalla persona in contesti formali, anche in caso di interruzione del percorso formativo, o di quelle validate acquisite in contesti non formali e informali. La procedura di certificazione delle competenze si conclude con il rilascio di un certificato conforme agli standard minimi~\cite{DL2013}.}}
\newglossaryentry{competenzag}{name={Competenza},description={Comprovata capacità di utilizzare, in situazioni di lavoro, di studio o nello sviluppo professionale e personale, un insieme strutturato di conoscenze e di abilità acquisite nei contesti di apprendimento formale, non formale o informale~\cite{DL2013}.}}
\newglossaryentry{competenzainscienzetecnologieeingegneria}{name={Competenza in scienze, tecnologie e ingegneria},description={La competenza in scienze si riferisce alla capacità di spiegare il mondo che ci circonda usando l’insieme delle conoscenze e delle metodologie, comprese l’osservazione e la sperimentazione, per identificare le problematiche e trarre conclusioni che siano basate su fatti empirici, e alla disponibilità a farlo. Le competenze in tecnologie e ingegneria sono applicazioni di tali conoscenze e metodologie per dare risposta ai desideri o ai bisogni avvertiti dagli esseri umani. La competenza in scienze, tecnologie e ingegneria implica la comprensione dei cambiamenti determinati dall’attività umana e della responsabilità individuale del cittadino~\cite{RA2018}.}}
\newglossaryentry{competenzaq}{name={Competenza},description={Ciò che una persona sa, capisce ed è capace di fare~\cite{DE2018}.}}
\newglossaryentry{competenza}{name={Competenza},description={Comprovata capacità di utilizzare conoscenze, abilità e capacità personali, sociali e/o metodologiche in situazioni di lavoro o di studio e nello sviluppo professionale e personale~\cite{RA2017}.}}
\newglossaryentry{competenzechiave}{name={Competenze chiave},description={Ai fini della presente raccomandazione le competenze sono definite come una combinazione di conoscenze, abilità e atteggiamenti, in cui: d) la conoscenza si compone di fatti e cifre, concetti, idee e teorie che sono già stabiliti e che forniscono le basi per comprendere un certo settore o argomento; e) per abilità si intende sapere ed essere capaci di eseguire processi ed applicare le conoscenze esistenti al fine di ottenere risultati; f) gli atteggiamenti descrivono la disposizione e la mentalità per agire o reagire a idee, persone o situazioni. Le competenze chiave sono quelle di cui tutti hanno bisogno per la realizzazione e lo sviluppo personali, l’occupabilità, l’inclusione sociale, uno stile di vita sostenibile, una vita fruttuosa in società pacifiche, una gestione della vita attenta alla salute e la cittadinanza attiva. Esse si sviluppano in una prospettiva di apprendimento permanente, dalla prima infanzia a tutta la vita adulta, mediante l’apprendimento formale, non formale e informale in tutti i contesti, compresi la famiglia, la scuola, il luogo di lavoro, il vicinato e altre comunità~\cite{RA2018}.}}
\newglossaryentry{conoscenze}{name={Conoscenze},description={Risultato dell’assimilazione di informazioni attraverso l’apprendimento. Le conoscenze sono l’insieme di fatti, principi, teorie e pratiche che riguardano un ambito di lavoro o di studio. Nel contesto dell’EQF, le conoscenze sono descritte come teoriche e/o pratiche~\cite{RA2017}.}}
\newglossaryentry{convalidag}{name={Convalida},description={Il processo mediante il quale un'autorità o un organismo competente conferma che un individuo ha acquisito, anche in un contesto di apprendimento non formale e informale, risultati dell'apprendimento misurati in relazione a uno standard appropriato e che si articola in quattro fasi distinte, vale a dire individuazione, documentazione, valutazione e certificazione dei risultati della valutazione sotto forma di qualifica piena, crediti o qualifica parziale, ove opportuno e in funzione delle circostanze nazionali~\cite{DE2018}~\cite{RA2012}.}}
\newglossaryentry{convalida}{name={Convalida dell’apprendimento non formale e informale},description={Processo in base al quale un’autorità competente conferma l’acquisizione, in un contesto di apprendimento non formale e informale, di risultati dell’apprendimento misurati in relazione a uno standard appropriato; si articola nelle seguenti quattro fasi distinte: individuazione, mediante un colloquio, delle esperienze specifiche dell’interessato; documentazione per rendere visibili le esperienze dell’interessato; valutazione formale di tali esperienze e certificazione dei risultati della valutazione, che può portare a una qualifica parziale o completa~\cite{RA2017}.}}
\newglossaryentry{crediti}{name={Crediti},description={Unità che confermano che una parte della qualifica, costituita da un insieme coerente di risultati dell’apprendimento, è stata valutata e convalidata da un’autorità competente, secondo una norma concordata; i crediti sono concessi da autorità competenti quando il soggetto ha conseguito i risultati dell’apprendimento definiti, comprovati da opportune valutazioni, e possono essere espressi con un valore quantitativo (ad esempio crediti o unità di credito), che indica il carico di lavoro ritenuto solitamente necessario affinché una persona consegua i risultati dell’apprendimento corrispondenti~\cite{RA2017}.}}
\newglossaryentry{datipersonali}{name={Dati personali},description={Informazioni riguardanti una persona fisica identificata o identificabile~\cite{DE2018}.}}
\newglossaryentry{dimensioneeuropeaorientamento}{name={Dimensione europea dell'orientamento},description={La cooperazione e il sostegno a livello di Unione volti a rafforzare politiche, sistemi e pratiche di orientamento all'interno dell'Unione~\cite{DE2018}.}}
\newglossaryentry{dirigenteazienda}{name={Dirigente d'azienda},description={Qualsiasi persona che abbia svolto in un'impresa del settore professionale corrispondente:i) la funzione di direttore d'azienda o di filiale, o ii) la funzione di institore o vice direttore d'azienda, se tale funzione implica una responsabilità corrispondente a quella dell'imprenditore o del direttore d'azienda rappresentato, o iii) la funzione di dirigente con mansioni commerciali e/o tecniche e responsabile di uno o più reparti dell'azienda.~\cite{DI2005}.}}
\newglossaryentry{eIDASg}{name={Regolamento eIDAS},description={Regolamento UE n° 910/2014 sull’identità digitale - ha l’obiettivo di fornire una base normativa a livello comunitario per i servizi fiduciari e i mezzi di identificazione elettronica degli stati membri~\cite{RE2014}.}}
\newglossaryentry{entepubblicotitolareg}{name={Ente pubblico titolare},description={Amministrazione pubblica, centrale, regionale e delle province autonome titolare, a norma di legge, della regolamentazione di servizi di individuazione e validazione e certificazione delle competenze. Nello specifico sono da intendersi enti pubblici titolari: 1) il Ministero dell'istruzione, dell'università e della ricerca, in materia di individuazione e validazione e certificazione delle competenze riferite ai titoli di studio del sistema scolastico e universitario; 2) le regioni e le province autonome di Trento e Bolzano, in materia di individuazione e validazione e certificazione di competenze riferite a qualificazioni rilasciate nell'ambito delle rispettive competenze; 3) il Ministero del lavoro e delle politiche sociali, in materia di individuazione e validazione e certificazione di competenze riferite a qualificazioni delle professioni non organizzate in ordini o collegi, salvo quelle comunque afferenti alle autorità competenti di cui al successivo punto 4; 4) il Ministero dello sviluppo economico e le altre autorità competenti ai sensi dell'articolo 5 del decreto legislativo 9 novembre 2007, n. 206, in materia di individuazione e validazione e certificazione di competenze riferite a qualificazioni delle professioni regolamentate a norma del medesimo decreto~\cite{DL2013}.}}
\newglossaryentry{entetitolato}{name={Ente titolato},description={Soggetto, pubblico o privato, ivi comprese le camere di commercio, industria, artigianato e agricoltura, autorizzato o accreditato dall'ente pubblico titolare, ovvero deputato a norma di legge statale o regionale, ivi comprese le istituzioni scolastiche, le università e le istituzioni dell'alta formazione artistica, musicale e coreutica, a erogare in tutto o in parte servizi di individuazione e validazione e certificazione delle competenze, in relazione agli ambiti di titolarità enti pubblico titolare~\cite{DL2013}.}}
\newglossaryentry{esperienzaprofessionale}{name={Esperienza professionale},description={L’esercizio effettivo e legittimo della professione in questione in uno Stato membro, a tempo pieno o a tempo parziale per un periodo equivalente~\cite{DI2013}.}}
\newglossaryentry{formazioneregolamentata}{name={Formazione regolamentata},description={qualsiasi formazione specificamente orientata all'esercizio di una professione determinata e consistente in un ciclo di studi completato, eventualmente, da una formazione professionale, un tirocinio professionale o una pratica professionale. La struttura e il livello della formazione professionale, del tirocinio professionale o della pratica professionale sono stabiliti dalle disposizioni legislative, regolamentari o amministrative dello Stato membro in questione e sono soggetti a controllo o autorizzazione dell'autorità designata a tal fine~\cite{DI2005}.}}
\newglossaryentry{individuazionevalidazionecompetenze}{name={Individuazione e validazione delle competenze},description={processo che conduce al riconoscimento, da parte dell'ente titolato in base alle norme generali, ai livelli essenziali delle prestazioni e agli standard minimi di cui al presente decreto, delle competenze acquisite dalla persona in un contesto non formale o informale. Ai fini della individuazione delle competenze sono considerate anche quelle acquisite in contesti formali. La validazione delle competenze puo' essere seguita dalla certificazione delle competenze ovvero si conclude con il rilascio di un documento di validazione conforme agli standard minimi~\cite{DL2013}.}}
\newglossaryentry{interoperabilittecnica}{name={Interoperabilità tecnica},description={La capacità dei sistemi di tecnologia dell'informazione e della comunicazione di interagire in modo da consentire la condivisione di informazioni, mediante un accordo fra tutte le parti e i titolari delle informazioni~\cite{DE2018}.}}
\newglossaryentry{motiviimperativiinteressegenerale}{name={Motivi imperativi di interesse generale},description={Motivi riconosciuti tali dalla giurisprudenza della Corte di giustizia dell’Unione europea~\cite{DI2013}.}}
\newglossaryentry{organismonazionaleitalianodiaccreditamento}{name={Organismo nazionale italiano di accreditamento},description={Organismo nazionale di accreditamento designato dall'Italia in attuazione del regolamento (CE) n. 765/2008 del Parlamento europeo e del Consiglio del 9 luglio 2008~\cite{DL2013}.}}
\newglossaryentry{orientamento}{name={Orientamento},description={Un processo continuativo che consente alle persone di identificare le proprie capacità, competenze e interessi attraverso una serie di attività individuali e collettive che servono a prendere decisioni in materia di istruzione, formazione e occupazione e a gestire i propri percorsi personali nell'ambito dell'istruzione, del lavoro e in altri contesti in cui è possibile acquisire o sfruttare tali capacità e competenze~\cite{DE2018}.}}
\newglossaryentry{piattaformaonline}{name={Piattaforma online},description={Un'applicazione basata sul web che fornisce informazioni e strumenti agli utenti finali e permette loro di portare a termine compiti specifici online~\cite{DE2018}.}}
\newglossaryentry{professioneregolamentata}{name={Professione regolamentata},description={Attività, o insieme di attività professionali, l'accesso alle quali e il cui esercizio, o una delle cui modalità di esercizio, sono subordinati direttamente o indirettamente, in forza di norme legislative, regolamentari o amministrative, al possesso di determinate qualifiche professionali; in particolare costituisce una modalità di esercizio l'impiego di un titolo professionale riservato da disposizioni legislative, regolamentari o amministrative a chi possiede una specifica qualifica professionale. Quando non si applica la prima frase, è assimilata ad una professione regolamentata una professione di cui al paragrafo 2~\cite{DI2005}.}}
\newglossaryentry{provaattitudinale}{name={Prova attitudinale},description={Una verifica riguardante le conoscenze, le abilità e le competenze professionali del richiedente, effettuata o riconosciuta dalle autorità competenti dello Stato membro ospitante allo scopo di valutare l’idoneità del richiedente a esercitare in tale Stato membro una professione regolamentata. Per consentire che la verifica sia effettuata, le autorità competenti predispongono un elenco delle materie che, in base a un confronto tra la formazione e l’istruzione richiesta nello Stato membro ospitante e quella ricevuta dal richiedente, non sono coperte dal diploma o dai titoli di formazione del richiedente. La prova attitudinale deve tener conto del fatto che il richiedente è un professionista qualificato nello Stato membro d’origine o di provenienza. Essa verte su materie da scegliere tra quelle che figurano nell’elenco e la cui conoscenza è essenziale per poter esercitare la professione in questione nello Stato membro ospitante. Tale prova può altresì comprendere la conoscenza delle regole professionali applicabili alle attività in questione nello Stato membro ospitante. Le modalità dettagliate della prova attitudinale nonché lo status di cui gode, nello Stato membro ospitante, il richiedente che desidera prepararsi alla prova attitudinale in detto Stato membro sono determinate dalle autorità competenti di detto Stato membro»~\cite{DI2013}.}}
\newglossaryentry{quadro nazionale delle qualifiche}{name={Quadro nazionale delle qualifiche},description={Strumento di classificazione delle qualifiche in funzione di una serie di criteri basati sul raggiungimento di livelli di apprendimento specifici; esso mira a integrare e coordinare i sottosistemi nazionali delle qualifiche e a migliorare la trasparenza, l’accessibilità, la progressione e la qualità delle qualifiche rispetto al mercato del lavoro e alla società civile~\cite{RA2017}.}}
\newglossaryentry{quadronazionalequalifiche}{name={Quadro nazionale di qualifiche},description={Strumento di classificazione delle qualifiche in funzione di una serie di criteri basati sul raggiungimento di livelli di apprendimento specifici; esso mira a integrare e coordinare i sottosistemi nazionali delle qualifiche e a migliorare la trasparenza, l’accessibilità, la progressione e la qualità delle qualifiche rispetto al mercato del lavoro e alla società civile~\cite{RA2012}.}}
\newglossaryentry{qualificainternazionale}{name={Qualifica internazionale},description={Qualifica, rilasciata da un organismo internazionale legalmente costituito (associazione, organizzazione, settore o impresa) o da un organismo nazionale che agisce a nome di un organismo internazionale, che è utilizzata in più di un paese e include i risultati dell’apprendimento, valutati facendo riferimento alle norme stabilite da un organismo internazionale~\cite{RA2017}.}}
\newglossaryentry{qualificazioneinternazionale}{name={Qualificazione internazionale},description={Qualificazione rilasciata da un organismo internazionale legalmente costituito o da un organismo nazionale che agisce a nome di un organismo internazionale, che è utilizzata in più di un Paese e include i risultati di apprendimento, valutati facendo riferimento alle norme stabilite da un organismo internazionale~\cite{DL2018}.}}
\newglossaryentry{qualificazione}{name={Qualificazione},description={Titolo di istruzione e di formazione, ivi compreso quello di istruzione e formazione professionale, o di qualificazione professionale rilasciato da un ente pubblico titolato nel rispetto delle norme generali, dei livelli essenziali delle prestazioni e degli standard minimi~\cite{DL2013}.}}
\newglossaryentry{qualifica}{name={Qualifica},description={Risultato formale di un processo di valutazione e convalida, acquisito quando un’autorità competente stabilisce che una persona ha conseguito i risultati dell’apprendimento rispetto a standard predefiniti~\cite{RA2017}~\cite{DE2018}~\cite{RA2012}.}}
\newglossaryentry{qualificheprofessionali}{name={Qualifiche professionali},description={Le qualifiche attestate da un titolo di formazione, un attestato di competenza - di cui all'articolo 11, lettera a), punto i) - e/o un'esperienza professionale~\cite{DI2005}.}}
\newglossaryentry{readingliteracy}{name={Reading literacy},description={Competenza in lettura}}
\newglossaryentry{referenziazione}{name={Referenziazione},description={Il processo istituzionale e tecnico che associa le qualificazioni rilasciate nell'ambito del Sistema nazionale di certificazione delle competenze a uno degli otto livelli del QNQ. La referenziazione delle qualificazioni italiane al garantisce la referenziazione delle stesse al Quadro europeo delle qualifiche~\cite{DL2018}.}}
\newglossaryentry{responsabilitaeautonomia}{name={Responsabilità e autonomia},description={Capacità del discente di applicare le conoscenze e le abilità in modo autonomo e responsabile~\cite{RA2017}.}}
\newglossaryentry{riconoscimentoformazioneprecedente}{name={Riconoscimento della formazione precedente},description={Convalida dei risultati di apprendimento, nel quadro dell'istruzione formale o dell'apprendimento non formale o informale, acquisiti prima della richiesta di convalida~\cite{RA2012}.}}
\newglossaryentry{riconoscimento}{name={Riconoscimento formale dei risultati dell’apprendimento},description={Processo in base al quale un’autorità competente dà valore ufficiale ai risultati dell’apprendimento acquisiti a fini di studi ulteriori o di occupazione, mediante i) il rilascio di qualifiche (certificati, diplomi o titoli), ii) la convalida dell’apprendimento non formale e informale, iii) il riconoscimento di equivalenze, il rilascio di crediti o la concessione di deroghe~\cite{RA2017}.}}
\newglossaryentry{risorseeducativeaperte}{name={Risorse educative aperte (OER)},description={Materiale digitalizzato messo gratuitamente e liberamente a disposizione di docenti, studenti, e chiunque studi in maniera autonoma, per l'uso e il riuso nell'insegnamento, l'apprendimento e la ricerca; esse comprendono materiale didattico, strumenti informatici per lo sviluppo, l'uso e la diffusione dei contenuti, e risorse per l'applicazione come le licenze aperte; le OER fanno anche riferimento a una somma di beni digitali che possono essere modificati e che offrono vantaggi senza che ne sia limitata la possibilità di utilizzo da parte di altri~\cite{RA2012}.}}
\newglossaryentry{risultatiapprendimentoq}{name={Risultati di apprendimento},description={Descrizione di ciò che un discente conosce, capisce ed è in grado di realizzare al termine di un processo di apprendimento definito in termini di conoscenze, abilità e competenze~\cite{RA2012}.}}
\newglossaryentry{risultatidellapprendimento}{name={Risultati dell’apprendimento},description={Descrizione di ciò che un discente conosce, capisce ed è in grado di realizzare al termine di un processo di apprendimento; sono definiti in termini di conoscenze, abilità e responsabilità e autonomia~\cite{RA2017}.}}
\newglossaryentry{servizidiautenticazione}{name={Servizi di autenticazione},description={processi tecnici, quali le firme elettroniche e l'autenticazione di siti web, che consentono agli utenti di verificare le informazioni, come ad esempio l'identità, attraverso Europass~\cite{DE2018}.}}
\newglossaryentry{sistemanazionalecertificazionecompetenze}{name={Sistema nazionale di certificazione delle competenze},description={L'insieme dei servizi di individuazione e validazione e certificazione delle competenze erogati nel rispetto delle norme generali, dei livelli essenziali delle prestazioni e degli standard minimi~\cite{DL2013}.}}
\newglossaryentry{sistemanazionaledellequalifiche}{name={Sistema nazionale delle qualifiche},description={Complesso delle attività di uno Stato membro connesse con il riconoscimento dell’apprendimento e altri meccanismi che mettono in relazione istruzione e formazione al mercato del lavoro e alla società civile. Ciò comprende l’elaborazione e l’attuazione di disposizioni e processi istituzionali in materia di garanzia della qualità, valutazione e rilascio delle qualifiche. Un sistema nazionale delle qualifiche può essere composto da vari sottosistemi e può comprendere un quadro nazionale delle qualifiche~\cite{RA2017}.}}
\newglossaryentry{sistemidicrediti}{name={Sistemi di crediti},description={Strumenti di trasparenza volti ad agevolare il riconoscimento dei crediti. Tali sistemi possono comprendere tra l’altro equivalenze, esenzioni, possibilità di accumulare e trasferire unità/moduli, autonomia degli erogatori che possono personalizzare i percorsi nonché convalida dell’apprendimento non formale e informale~\cite{RA2017}.}}
\newglossaryentry{standardaperti}{name={Standard aperti},description={Standard tecnici che sono stati elaborati nell'ambito di un processo collaborativo e sono stati pubblicati per essere utilizzati liberamente da tutti i soggetti interessati; ~\cite{DE2018}.}}
\newglossaryentry{standardminimiservizio}{name={Standard minimi di servizio},description={Gli standard minimi di servizio costituiscono livelli essenziali delle prestazioni da garantirsi su tutto il territorio nazionale, anche in riferimento alla individuazione e validazione degli apprendimenti non formali e informali e al riconoscimento dei crediti formativi. Gli standard minimi di servizio costituiscono riferimento per gli enti pubblici titolari nella definizione di standard minimi di erogazione dei servizi da parte degli enti titolati~\cite{DL2013}.}}
\newglossaryentry{supplementi Europass}{name={Supplementi Europass},description={Una serie di documenti, come ad esempio i supplementi al diploma e i supplementi al certificato, rilasciati dalle autorità od organismi competenti~\cite{DE2018}.}}
\newglossaryentry{supplementoaldiploma}{name={Supplemento al diploma},description={Un documento allegato a un diploma di istruzione superiore rilasciato dalle autorità od organismi competenti allo scopo di facilitare la comprensione da parte di terzi — soprattutto in un altro paese — dei risultati di apprendimento ottenuti dal titolare della qualifica come pure della natura, del livello, del contesto, del contenuto e dello status dell'istruzione e della formazione completata e delle competenze acquisite~\cite{DE2018}.}}
\newglossaryentry{supplementocertificato}{name={Supplemento al certificato},description={Un documento accluso a un certificato di istruzione e formazione professionale o a un certificato professionale rilasciato dalle autorità od organismi competenti allo scopo di facilitare la comprensione da parte di terzi — soprattutto in un altro paese — dei risultati di apprendimento ottenuti dal titolare della qualifica come pure della natura, del livello, del contesto, del contenuto e dello status dell'istruzione e della formazione completata e delle competenze acquisite~\cite{DE2018}.}}
\newglossaryentry{tesseraprofessionaleeuropea}{name={Tessera professionale europea},description={Un certificato elettronico attestante o che il professionista ha soddisfatto tutte le condizioni necessarie per fornire servizi, su base temporanea e occasionale, in uno Stato membro ospitante o il riconoscimento delle qualifiche professionali ai fini dello stabilimento in uno Stato membro ospitante~\cite{DI2013}.}}
\newglossaryentry{tirociniodiadattamento}{name={Tirocinio di adattamento},description={l'esercizio di una professione regolamentata nello Stato membro ospitante sotto la responsabilità di un professionista qualificato, accompagnato eventualmente da una formazione complementare. Il tirocinio è oggetto di una valutazione. Le modalità del tirocinio di adattamento e della sua valutazione nonché lo status di tirocinante migrante sono determinati dalle autorità competenti dello Stato membro ospitante. Lo status di cui il tirocinante gode nello Stato membro ospitante, soprattutto in materia di diritto di soggiorno nonché di obblighi, diritti e benefici sociali, indennità e retribuzione, è stabilito dalle autorità competenti di detto Stato membro conformemente al diritto comunitario applicabile~\cite{DI2005}.}}
\newglossaryentry{tirocinioprofessionale}{name={Tirocinio professionale},description={Fatto salvo l’articolo 46, paragrafo 4, un periodo di pratica professionale effettuato sotto supervisione, purché costituisca una condizione per l’accesso a una professione regolamentata e che può svolgersi durante o dopo il completamento di un’istruzione che conduce a un diploma~\cite{DI2013}.}}
\newglossaryentry{titoloformazione}{name={Titolo di formazione},description={Diplomi, certificati e altri titoli rilasciati da un'autorità di uno Stato membro designata ai sensi delle disposizioni legislative, regolamentari e amministrative di tale Stato membro e che sanciscono una formazione professionale acquisita in maniera preponderante nella Comunità~\cite{DI2005}.}}
\newglossaryentry{trasferimentodicrediti}{name={Trasferimento di crediti},description={Processo che consente ai soggetti che hanno accumulato crediti in un contesto di farli valutare e riconoscere in un altro contesto~\cite{RA2017}.}}
\newglossaryentry{valutazionedellecompetenze}{name={Valutazione delle competenze},description={Il processo o il metodo utilizzato per valutare, misurare e infine descrivere, mediante l'autovalutazione o la valutazione certificata da terzi o entrambe, le competenze individuali acquisite in contesti formali, non formali o informali~\cite{DE2018}.}}