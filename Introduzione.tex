\chapter{Introduzione}
Quando ho cominciato a scrivere queste note non pensavo alla necessità di un'introduzione, poi andando avanti,con il pensiero, mi sono visto nel 1968 che varcavo la porta della scuola dove avrei fatto la mia prima elementare. Tante cose sono cambiate, era il 31 ottobre, san Remigio e nel pomeriggio, avrei visto in TV la festa dei Remigini.\par  Oggi non troverei più la mia scuola che ha ormai cambiato nome in scuola primaria, non esiste più la festa dei Remigini perché ora la scuola inizia a settembre con un calendario deciso dalla Regione in cui vivo.\par  Tante cose sono cambiate, in bene o in peggio non so, molti di questi cambiamenti sono strutturali e difficilmente  etichettatili a questa o a quella forza politica.\par  Sono cambiamenti imposti dal mercato, brutta parola, o dalla Comunità Europea che per omogenizzare ha imposto le sue direttive e raccomandazioni, o dalla necessità d'inseguire  miglioramenti statistici a volte discutibili o anche dalla necessità di un adeguamento ai tempi che cambiano.\par  Quello che noto, è la nascita un linguaggio tecnico, una neo-lingua che con le nuove parole crea concetti e ne toglie altri creando così nuove idee.\par  Mi tornano in mente i rosari che sentivo in gioventù, con cui, in un improbabile latino, le vecchie della parrocchia pregavano, non comprendendo probabilmente, le parole che usavano.\par  Questo va evitato. Bisogna combattere la formazione di neo-lingue, che diventando  cesure nette con la società, isolano la scuola e  chi ci vive.\par  Quello che segue è un glossario in cui ho riportato le definizioni trovate nelle leggi, decreti, raccomandazioni prodotte negli anni e che sono riuscito a reperire. 