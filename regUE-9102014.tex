\section*{Regolamento UE n. 910/2014}
\section*{Allegato I}\label{sec:allegatoIreg9102014}
Requisiti per i certificati qualificati di firma elettronica\par I certificati qualificati di firma elettronica contengono:
\begin{enumerate}
	\item 	un’indicazione, almeno in una forma adatta al trattamento automatizzato, del fatto che il certificato è stato rilasciato quale certificato qualificato di firma elettronica;
	\item un insieme di dati che rappresenta in modo univoco il prestatore di servizi fiduciari qualificato che rilascia i certificati qualificati e include almeno lo Stato membro in cui tale prestatore è stabilito e
	\begin{enumerate}
		\item per una persona giuridica: il nome e, se del caso, il numero di registrazione quali figurano nei documenti ufficiali,
		\item per una persona fisica: il nome della persona;
	\end{enumerate}
\item è chiaramente indicato almeno il nome del firmatario, o uno pseudonimo, qualora sia usato uno pseudonimo;
\item i dati di convalida della firma elettronica che corrispondono ai dati per la creazione di una firma elettronica;
\item l’indicazione dell’inizio e della fine del periodo di validità del certificato;
\item il codice di identità del certificato che deve essere unico per il prestatore di servizi fiduciari qualificato;
\item 	
la firma elettronica avanzata o il sigillo elettronico avanzato del prestatore di servizi fiduciari qualificato che rilascia il certificato;
\item il luogo in cui il certificato relativo alla firma elettronica avanzata o al sigillo elettronico avanzato di cui al numero 7) è disponibile gratuitamente;
\item 	
l’ubicazione dei servizi a cui ci si può rivolgere per informarsi sulla validità del certificato qualificato;
\item qualora i dati per la creazione di una firma elettronica connessi ai dati di convalida della firma elettronica siano ubicati in un dispositivo per la creazione di una firma elettronica qualificata, un’indicazione appropriata di questo fatto, almeno in una forma adatta al trattamento automatizzato.
\end{enumerate}
\subsection*{Allegato II}\label{sec:allegatoIIreg9102014}
Requisiti per i dispositivi per la creazione di una firma elettronica qualificata
\begin{enumerate}
	\item I dispositivi per la creazione di una firma elettronica qualificata garantiscono, mediante mezzi tecnici e procedurali appropriati, almeno quanto segue:
	\begin{enumerate}
		\item 	
		è ragionevolmente assicurata la riservatezza dei dati per la creazione di una firma elettronica utilizzati per creare una firma elettronica;
		\item 	
		i dati per la creazione di una firma elettronica utilizzati per creare una firma elettronica possono comparire in pratica una sola volta;
		\item 	
		i dati per la creazione di una firma elettronica utilizzati per creare una firma elettronica non possono, con un grado ragionevole di sicurezza, essere derivati e la firma elettronica è attendibilmente protetta da contraffazioni compiute con l’impiego di tecnologie attualmente disponibili;
		\item i dati per la creazione di una firma elettronica utilizzati nella creazione della stessa possono essere attendibilmente protetti dal firmatario legittimo contro l’uso da parte di terzi.
	\end{enumerate}
\item I dispositivi per la creazione di una firma elettronica qualificata non alterano i dati da firmare né impediscono che tali dati siano presentati al firmatario prima della firma.
\item La generazione o la gestione dei dati per la creazione di una firma elettronica per conto del firmatario può essere effettuata solo da un prestatore di servizi fiduciari qualificato.
\item Fatto salvo il punto 1, lettera d), i prestatori di servizi fiduciari qualificati che gestiscono dati per la creazione di una firma elettronica per conto del firmatario possono duplicare i dati per la creazione di una firma elettronica solo a fini di back-up, purché rispettino i seguenti requisiti:
\begin{enumerate}
	\item 	
	la sicurezza degli insiemi di dati duplicati deve essere dello stesso livello della sicurezza degli insiemi di dati originali;
	\item 	
	il numero di insiemi di dati duplicati non eccede il minimo necessario per garantire la continuità del servizio.
\end{enumerate}
\end{enumerate}
\subsection*{Allegato III}\label{sec:allegatoIIIreg9102014}
Requisiti per i certificati qualificati dei sigilli elettronici\par 
I certificati qualificati dei sigilli elettronici contengono:
\begin{enumerate}
	\item un’indicazione, almeno in una forma adatta al trattamento automatizzato, del fatto che il certificato è stato rilasciato quale certificato qualificato di sigillo elettronico;
	\item 	
	un insieme di dati che rappresenta in modo univoco il prestatore di servizi fiduciari qualificato che rilascia i certificati qualificati e include almeno lo Stato membro in cui tale prestatore è stabilito e
	\begin{enumerate}
		\item per una persona giuridica: il nome e, se del caso, il numero di registrazione quali appaiono nei documenti ufficiali,
		\item per una persona fisica: il nome della persona;
	\end{enumerate}
\item almeno il nome del creatore del sigillo e, se del caso, il numero di registrazione quali appaiono nei documenti ufficiali;
\item 	
i dati di convalida del sigillo elettronico che corrispondono ai dati per la creazione di un sigillo elettronico;
\item l’indicazione dell’inizio e della fine del periodo di validità del certificato;
\item il codice di identità del certificato che deve essere unico per il prestatore di servizi fiduciari qualificato;
\item la firma elettronica avanzata o il sigillo elettronico avanzato del prestatore di servizi fiduciari qualificato che rilascia il certificato;
\item 	
il luogo in cui il certificato relativo alla firma elettronica avanzata o al sigillo elettronico avanzato di cui alla lettera g) è disponibile gratuitamente;
\item 	
l’ubicazione dei servizi a cui ci si può rivolgere per informarsi sulla validità del certificato qualificato;
\item qualora i dati per la creazione di un sigillo elettronico connessi ai dati di convalida del sigillo elettronico siano ubicati in un dispositivo per la creazione di un sigillo elettronico qualificato, un’indicazione appropriata di questo fatto, almeno in una forma adatta al trattamento automatizzato.
\end{enumerate}
\subsection*{Allegato IV}\label{sec:allegatoIVreg9102014}
Requisiti per i certificati qualificati di autenticazione di siti web\par 
I certificati qualificati di autenticazione di siti web contengono:
\begin{enumerate}
	\item un’indicazione, almeno in una forma adatta al trattamento automatizzato, del fatto che il certificato è stato rilasciato quale certificato qualificato di autenticazione di sito web;
	\item un insieme di dati che rappresenta in modo univoco il prestatore di servizi fiduciari qualificato che rilascia i certificati qualificati e include almeno lo Stato membro in cui tale prestatore è stabilito e
	\begin{enumerate}
		\item per una persona giuridica: il nome e, se del caso, il numero di registrazione quali appaiono nei documenti ufficiali,
		\item 	
		per una persona fisica: il nome della persona;
	\end{enumerate}
\item per le persone fisiche: almeno il nome della persona a cui è stato rilasciato il certificato, o uno pseudonimo. Qualora sia usato uno pseudonimo, ciò è chiaramente indicato;

per le persone giuridiche: almeno il nome della persona giuridica cui è stato rilasciato il certificato e, se del caso, il numero di registrazione quali appaiono nei documenti ufficiali;
\item elementi dell’indirizzo, fra cui almeno la città e lo Stato, della persona fisica o giuridica cui è rilasciato il certificato e, se del caso, quali appaiono nei documenti ufficiali;
\item il nome del dominio o dei domini gestiti dalla persona fisica o giuridica cui è rilasciato il certificato;
\item 	
l’indicazione dell’inizio e della fine del periodo di validità del certificato;
\item 	
il codice di identità del certificato che deve essere unico per il prestatore di servizi fiduciari qualificato;
\item la firma elettronica avanzata o il sigillo elettronico avanzato del prestatore di servizi fiduciari qualificato che rilascia il certificato;
\item 	
il luogo in cui il certificato relativo alla firma elettronica avanzata o al sigillo elettronico avanzato di cui al numero 8) è disponibile gratuitamente;
\item 	
l’ubicazione dei servizi competenti per lo status di validità del certificato a cui ci si può rivolgere per informarsi sulla validità dei certificato qualificato.
\end{enumerate}
\subsection*{Articolo 26}\label{sec:articolo26reg9102014}
Requisiti di una firma elettronica avanzata\par 
Una firma elettronica avanzata soddisfa i seguenti requisiti:
\begin{enumerate}
	\item 	
	è connessa unicamente al firmatario;
	\item 	
	è idonea a identificare il firmatario;
	\item è creata mediante dati per la creazione di una firma elettronica che il firmatario può, con un elevato livello di sicurezza, utilizzare sotto il proprio esclusivo controllo; e
	\item è collegata ai dati sottoscritti in modo da consentire l’identificazione di ogni successiva modifica di tali dati.
\end{enumerate}
\subsection*{Articolo 42}\label{sec:articolo42reg9102014}
Requisiti per la validazione temporale elettronica qualificata
\begin{enumerate}
	\item Una validazione temporale elettronica qualificata soddisfa i requisiti seguenti:
	\begin{enumerate}
		\item collega la data e l’ora ai dati in modo da escludere ragionevolmente la possibilità di modifiche non rilevabili dei dati;
		\item si basa su una fonte accurata di misurazione del tempo collegata al tempo universale coordinato; e
		\item è apposta mediante una firma elettronica avanzata o sigillata con un sigillo elettronico avanzato del prestatore di servizi fiduciari qualificato o mediante un metodo equivalente.
	\end{enumerate}
\item La Commissione può, mediante atti di esecuzione, stabilire i numeri di riferimento delle norme applicabili al collegamento della data e dell’ora ai dati e a fonti accurate di misurazione del tempo. Si presume che i requisiti di cui al paragrafo 1 siano stati rispettati ove il collegamento della data e dell’ora ai dati e alla fonte accurata di misurazione del tempo rispondano a dette norme. Tali atti di esecuzione sono adottati secondo la procedura d’esame di cui all’articolo 48, paragrafo 2.
\end{enumerate}
\subsection*{Articolo 44}\label{sec:articolo44reg9102014}
Requisiti per i servizi elettronici di recapito certificato qualificati:
\begin{enumerate}
	\item I servizi elettronici di recapito certificato qualificati soddisfano i requisiti seguenti:
	\begin{enumerate}
		\item sono forniti da uno o più prestatori di servizi fiduciari qualificati;
		\item garantiscono con un elevato livello di sicurezza l’identificazione del mittente;
		\item garantiscono l’identificazione del destinatario prima della trasmissione dei dati;
		\item 	
		l’invio e la ricezione dei dati sono garantiti da una firma elettronica avanzata o da un sigillo elettronico avanzato di un prestatore di servizi fiduciari qualificato in modo da escludere la possibilità di modifiche non rilevabili dei dati;
		\item qualsiasi modifica ai dati necessaria al fine di inviarli o riceverli è chiaramente indicata al mittente e al destinatario dei dati stessi;
		\item la data e l’ora di invio e di ricezione e qualsiasi modifica dei dati sono indicate da una validazione temporale elettronica qualificata.
	\end{enumerate}
Qualora i dati siano trasferiti fra due o più prestatori di servizi fiduciari qualificati, i requisiti di cui alle lettere da a) a f) si applicano a tutti i prestatori di servizi fiduciari qualificati.
\item La Commissione può, mediante atti di esecuzione, stabilire i numeri di riferimento delle norme applicabili ai processi di invio e ricezione dei dati. Si presume che i requisiti di cui al paragrafo 1 siano stati rispettati ove il processo di invio e ricezione dei dati risponda a tali norme. Tali atti di esecuzione sono adottati secondo la procedura d’esame di cui all’articolo 48, paragrafo 2.
 \end{enumerate}
