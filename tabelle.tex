\section*{Scuola}
\begin{adjustbox}{max width=\textwidth}
\begin{tabular}{m{4cm}p{12.0cm}}
\toprule
% \multicolumn{2}{c}{Raccomandazione del Consiglio Europeo relativa alle competenze chiave per
% 	l'apprendimento permanente}\\
% \midrule
\multirowcell{14}{Competenza personale,\\sociale e capacità\\ di imparare\\ a imparare}& Capacità di riflettere su se stessi e individuare le proprie attitudini\\
	&Capacità di gestire efficacemente il tempo e le informazioni  \\
	& Capacità di imparare e di lavorare sia in modalità collaborativa sia in maniera
	autonoma\\
	&  Capacità di lavorare con gli altri in maniera costruttiva\\
	&  Capacità di comunicare costruttivamente in ambienti diversi\\
	&  Capacità di creare fiducia e provare empatia\\
	&  Capacità di esprimere e comprendere punti di vista diversi\\
	&  Capacità di negoziare\\
	&  Capacità di concentrarsi, di riflettere criticamente e di prendere decisioni\\
	&  Capacità di gestire il proprio apprendimento e la propria carriera\\
	&  Capacità di gestire l'incertezza, la complessità e lo stress\\
	&  Capacità di mantenersi resilienti\\
	&  Capacità di favorire il proprio benessere fisico ed emotivo\\
\midrule
\multirowcell{1}{Competenze in materia\\di cittadinanza
}	& Capacità di impegnarsi efficacemente con gli altri per un interesse comune o
Pubblico \\
	& Capacità di pensiero critico e abilità integrate nella soluzione dei problemi \\
\midrule
	\multirowcell{12}{Competenza\\imprenditoriale}& Creatività e immaginazione\\
&Capacità di pensiero strategico e risoluzione dei problemi\\
& Capacità di trasformare le idee in azioni\\
& Capacità di riflessione critica e costruttiva \\
	&Capacità di assumere l'iniziativa\\
	& Capacità di lavorare sia in modalità collaborativa in gruppo sia in maniera autonoma\\
	& Capacità di mantenere il ritmo dell'attività\\
	& Capacità di comunicare e negoziare efficacemente con gli altri\\
	& Capacità di gestire l'incertezza, l'ambiguità e il rischio\\
	& Capacità di possedere spirito di iniziativa e autoconsapevolezza\\
	& Capacità di essere proattivi e lungimiranti\\
	& Capacità di coraggio e perseveranza nel raggiungimento degli obiettivi\\
	& Capacità di motivare gli altri e valorizzare le loro idee, di provare empatia\\
	& Capacità di accettare la responsabilità  \\
	\midrule
	\multirowcell{3}{Competenza in materia\\ di
		consapevolezza ed\\ espressione culturali
}& Capacità di esprimere esperienze ed emozioni con empatia\\
	& Capacità di riconoscere e realizzare le opportunità di valorizzazione personale,
sociale o commerciale mediante le arti e le atre forme culturali\\
& Capacità di impegnarsi in processi creativi sia individualmente che collettivamente\\
& Curiosità nei confronti del mondo, apertura per immaginare nuove possibilità\\
\bottomrule
\end{tabular}
\end{adjustbox}
\captionof{table}{Raccomandazione del Consiglio Europeo relativa alle competenze chiave per l'apprendimento permanente}
\begin{sidewaystable}
\centering
\begin{adjustbox}{max width=\textwidth}
\begin{tabular}{@{}>{\centering\arraybackslash}p{1.0cm}p{6cm}p{12cm}p{12cm}}
	\toprule
\multicolumn{1}{c}{Livello}	&\multicolumn{1}{c}{Conoscenze}  & \multicolumn{1}{c}{Abilità} & \multicolumn{1}{c}{Autonomia e Responsabilità} \\
	\midrule
1& Conoscenze concrete, di base, di limitata ampiezza, finalizzate ad eseguire un compito semplice in contesti noti e struttura &Applicare saperi, materiali e strumenti per svolgere un compito semplice, coinvolgendo abilità cognitive, relazionali e sociali di base.Tipicamente: CONCENTRAZIONE e INTERAZIONE   & Svolgere il compito assegnato nel rispetto dei parametri previsti, sotto diretta supervisione nello svolgimento delle attività, in un contesto strutturato. \\
\midrule
2	&Conoscenze concrete,di base, di moderata ampiezza,finalizzate ad eseguire compiti semplici in sequenze diversificate. & Applicare saperi, materiali e strumenti per svolgere compiti semplici insequenze diversificate, coinvolgendo abilità cognitive,relazionali e sociali necessarie per svolgere compiti semplici all'interno di una gamma definita di variabili di contesto.Tipicamente: MEMORIA e PARTECIPAZIONE & Eseguire i compiti assegnati secondo criteri prestabiliti, assicurando la conformità delle attività svolte, sotto supervisione per il conseguimento del risultato, in un contesto strutturato, con un numero limitato di situazioni diversificate \\
	\midrule
3	& Gamma di conoscenze, prevalentemente concrete, con elementi concettuali finalizzati a creare collegamenti logici.Capacità interpretativa. & Utilizzare anche attraverso adattamenti, riformulazioni e rielaborazioni una gamma di saperi, metodi, materiali e strumenti per raggiungere i risultati previsti, attivando un set di abilità cognitive, relazionali, sociali e di attivazione che facilitano l'adattamento nelle situazioni mutevoli.Tipicamente: COGNIZIONE, COLLABORAZIONE e ORIENTAMENTO AL RISULTATO  & Raggiungere i risultati previsti assicurandone la conformità e individuando le modalità di realizzazione più adeguate, in un contesto strutturato, con situazioni mutevoli che richiedono una modifica del proprio operato. \\
	\midrule
4	& Ampia gamma di conoscenze, integrate dal punto di vista della dimensione fattuale e/o concettuale, approfondite in alcune aree.Capacità interpretativa. & Utilizzare anche attraverso adattamenti, riformulazioni e rielaborazioni una gamma di saperi, metodi, prassi e protocolli, materiali e strumenti, per risolvere problemi, attivando un set di abilità cognitive, relazionali, sociali e di attivazione necessarie per superare difficoltà crescenti.Tipicamente: PROBLEM SOLVING, COOPERAZIONE e MULTITASKING & Provvedere al conseguimento degli obiettivi, coordinando e integrando le attività e i risultati anche di altri, partecipando al processo decisionale e attuativo, in un contesto di norma prevedibile, soggetto a cambiamenti imprevisti. \\
	\midrule
5	&Conoscenze integrate, complete, approfondite e specializzate.Consapevolezza degli ambiti di conoscenza.&Utilizzare anche attraverso adattamenti, riformulazioni e rielaborazioni un'ampia gamma di metodi, prassi, protocolli e strumenti, in modo consapevole e selettivo anche al fine di modificarli, attivando un set esauriente di abilità cognitive, relazionali, sociali e di attivazione che consentono di trovare soluzioni tecniche anche non convenzionali.Tipicamente: ANALISI E VALUTAZIONE, COMUNICAZIONE EFFICACE RISPETTO ALL'AMBITO TECNICO e GESTIONE DI CRITICITÀ&Garantire la conformità degli obiettivi conseguiti in proprio e da altre risorse, identificando e programmando interventi di revisione e sviluppo, identificando le decisioni e concorrendo al processo attuativo, in un contesto determinato, complesso ed esposto a cambiamenti ricorrenti e imprevisti.\\
\midrule
6&Conoscenze integrate, avanzate in un ambito, trasferibili da un contesto ad un altro.Consapevolezza critica di teorie e principi in un ambito&Trasferire in contesti diversi i metodi, le prassi e i protocolli necessari per risolvere problemi complessi e imprevedibili, mobilitando abilità cognitive, relazionali, sociali e di attivazione avanzate, necessarie per portare a sintesi operativa le istanze di revisione e quelle di indirizzo, attraverso soluzioni innovative e originali.Tipicamente: VISIONE DI SINTESI, CAPACITA' DI NEGOZIARE E MOTIVARE e PROGETTAZIONE &Presidiare gli obiettivi e i processi di persone e gruppi, favorendo la gestione corrente e la stabilità delle condizioni, decidendo in modo autonomo e negoziando obiettivi e modalità di attuazione, in un contesto non determinato, esposto a cambiamenti imprevedibili.\\
\midrule
7&Conoscenze integrate, altamente specializzate, alcune delle quali all'avanguardia in un ambito.Consapevolezza critica di teorie e principi in più ambiti di conoscenza&Integrare e trasformare saperi, metodi, prassi e protocolli, mobilitando abilità cognitive, relazionali, sociali e di attivazione specializzate, necessarie per indirizzare scenari di sviluppo, ideare e attuare nuove attività e procedure.Tipicamente: VISIONE SISTEMICA, LEADERSHIP, GESTIONE DI RETI RELAZIONALI E INTERAZIONI SOCIALI COMPLESSE e PIANIFICAZIONE &Governare i processi di integrazione e trasformazione, elaborando le strategie di attuazione e indirizzando lo sviluppo dei risultati e delle risorse, decidendo in modo indipendente e indirizzando obiettivi e modalità di attuazione, in un contesto non determinato, esposto a cambiamenti continui, di norma confrontabili rispetto a variabili note, soggetto ad innovazione\\
\midrule
8&Conoscenze integrate, esperte e all'avanguardia in un ambito e nelle aree comuni ad ambiti diversi.Consapevolezza critica di teorie e principi in più ambiti di conoscenza&Concepire nuovi saperi, metodi, prassi e protocolli, mobilitando abilità cognitive, relazionali, sociali e di attivazione esperte, necessarie a intercettare e rispondere alla domanda di innovazione.Tipicamente: VISIONE STRATEGICA, CREATIVITÀ e CAPACITÀ DI PROIEZIONE ED EVOLUZIONE&Promuovere processi di innovazione e sviluppo strategico, prefigurando scenari e soluzioni e valutandone i possibili effetti, in un contesto di avanguardia non confrontabile con situazioni e contesti precedenti.\\
\bottomrule
\end{tabular}
\end{adjustbox}
\caption[Quadro Nazionale delle Qualificazioni NQF Italia]{Quadro Nazionale delle Qualificazioni NQF Italia~\cite{DL2018}}
\end{sidewaystable}

\begin{sidewaystable}
	\centering
	\begin{adjustbox}{max width=\textwidth}
		\begin{tabular}{@{}>{\centering\arraybackslash}p{1.0cm}p{6cm}p{12cm}p{12cm}}
			\toprule
			\multicolumn{1}{c}{Livello}	&\multicolumn{1}{c}{Conoscenze}  & \multicolumn{1}{c}{Abilità} & \multicolumn{1}{c}{Responsabilità e autonomia} \\
			\midrule
		1&Nel contesto dell'EQF, le conoscenze sono descritte come teoriche e/o pratiche.&Nel contesto dell'EQF, le abilità sono descritte come cognitive (comprendenti l'uso del pensiero logico, intuitivo e creativo) e pratiche (comprendenti la manualità e l'uso di metodi, materiali, strumenti e utensili).&Nel contesto dell'EQF, la responsabilità e l'autonomia sono descritte come la capacità del discente di applicare le conoscenze e le abilità in modo autonomo e responsabile.\\
		\midrule
		2&Conoscenze generali di base&Abilità di base necessarie a svolgere compiti semplici&Lavoro o studio, sotto supervisione diretta, in un contesto strutturato\\
		\midrule
		3&Conoscenze pratiche di base in un ambito di lavoro o di studio&Abilità cognitive e pratiche di base necessarie all'uso di informazioni pertinenti per svolgere compiti e risolvere problemi ricorrenti usando strumenti e regole semplici&Lavoro o studio, sotto supervisione, con un certo grado di autonomia\\
		\midrule
		3&Conoscenza di fatti, principi, processi e concetti generali, in un ambito di lavoro o di studio&Una gamma di abilità cognitive e pratiche necessarie a svolgere compiti e risolvere problemi scegliendo e applicando metodi di base, strumenti, materiali ed informazioni&Assumere la responsabilità di portare a termine compiti nell'ambito del lavoro o dello studio
		
		Adeguare il proprio comportamento alle circostanze nella soluzione dei problemi\\
		\midrule
		4&Conoscenze pratiche e teoriche in ampi contesti in un ambito di lavoro o di studio&Una gamma di abilità cognitive e pratiche necessarie a risolvere problemi specifici in un ambito di lavoro o di studio&Sapersi gestire autonomamente, nel quadro di istruzioni in un contesto di lavoro o di studio, di solito prevedibili ma soggetti a cambiamenti
		
		Sorvegliare il lavoro di routine di altri, assumendo una certa responsabilità per la valutazione e il miglioramento di attività lavorative o di studio\\
		\midrule
		5&Conoscenze pratiche e teoriche esaurienti e specializzate, in un ambito di lavoro o di studio, e consapevolezza dei limiti di tali conoscenze&Una gamma esauriente di abilità cognitive e pratiche necessarie a dare soluzioni creative a problemi astratti&Saper gestire e sorvegliare attività nel contesto di attività lavorative o di studio esposte a cambiamenti imprevedibili
		
		Esaminare e sviluppare le prestazioni proprie e di altri\\
		\midrule
		6&Conoscenze avanzate in un ambito di lavoro o di studio, che presuppongono una comprensione critica di teorie e principi&Abilità avanzate, che dimostrino padronanza e innovazione necessarie a risolvere problemi complessi ed imprevedibili in un ambito specializzato di lavoro o di studio&Gestire attività o progetti tecnico/professionali complessi assumendo la responsabilità di decisioni in contesti di lavoro o di studio imprevedibili.

		Assumere la responsabilità di gestire lo sviluppo professionale di persone e gruppi\\
		\midrule
		7&Conoscenze altamente specializzate, parte delle quali all'avanguardia in un ambito di lavoro o di studio, come base del pensiero originale e/o della ricerca
		
		Consapevolezza critica di questioni legate alla conoscenza in un ambito e all'intersezione tra ambiti diversi&Abilità specializzate, orientate alla soluzione di problemi, necessarie nella ricerca e/o nell'innovazione al fine di sviluppare conoscenze e procedure nuove e integrare le conoscenze ottenute in ambiti diversi& Gestire e trasformare contesti di lavoro o di studio complessi, imprevedibili e che richiedono nuovi approcci strategici
		
		Assumere la responsabilità di contribuire alla conoscenza e alla pratica professionale e/o di verificare le prestazioni strategiche dei gruppi\\
		\midrule
		8&Le conoscenze più all'avanguardia in un ambito di lavoro o di studio e all'intersezione tra ambiti diversi&Le abilità e le tecniche più avanzate e specializzate, comprese le capacità di sintesi e di valutazione, necessarie a risolvere problemi complessi della ricerca e/o dell'innovazione e ad estendere e ridefinire le conoscenze o le pratiche professionali esistenti.&Dimostrare effettiva autorità, capacità di innovazione, autonomia, integrità tipica dello studioso e del professionista e impegno continuo nello sviluppo di nuove idee o processi all'avanguardia in contesti di lavoro, di studio e di ricerca.\\
		
			\bottomrule
		\end{tabular}
	\end{adjustbox}
	\caption[Descrittori che definiscono i livelli del quadro europeo delle qualifiche (EQF)]{Descrittori che definiscono i livelli del quadro europeo delle qualifiche (EQF)~\cite{RA2017}}
\end{sidewaystable}
\begin{center}
	\begin{tabular}{cl}
\toprule
Codice&Decodifica\\
\midrule
AA&Scuola Materna\\
EE&Scuola Elementare\\
IC&Istituto Comprensivo\\
IS&Istituto di Istruzione Secondaria Superiore\\
MM&Scuola Media\\
PC& Liceo Classico\\
PL&Liceo Linguistico\\
PM&Istituto Magistrale\\
PQ&Scuola Magistrale\\
PS& Liceo Scientifico\\
RA&Ist.Prof. per l'agricoltura\\
RB&Scuola Tecnica per l'arte Bianca\\
RC&Ist.Prof.Commerciale\\
RE&Ist.Prof.Ind. e Art.per Ciechi\\
RF&Ist.Prof.Femminile\\
RH&Ist.Prof.Alberghiero\\
RI&Ist.Prof.Industria e Artigianato\\
RM&Ist.Prof.Ind.e Attività Marinare\\
RN&Ist.Prof.per l'alimentazione\\
RS&Ist.Prof.Ind.e Art.per Sordomuti\\
RT&Ist.Prof.per l'industria Edile\\
RV&Ist.Prof.Cinematografia e Televisione\\
SD&Istituto d'arte\\
SL&Liceo Artistico\\
SM&Accademia di Belle Arti\\
SN&Accademia Nazionale di Danza\\
SR&Accademia Nazionale d'arte Drammatica\\
ST&Conservatorio di Musica\\
TA&Istituto Tecnico Agrario\\
TB&Istituto Tecnico Aeronautico\\
TD&Istituto Tecnico Commerciale e per Geometri\\
TD&Istituto Tecnico Commerciale\\
TE&Istituto Tecnico Femminile\\
TF&Istituto Tecnico Industriale\\
TH&Istituto Tecnico Nautico\\
TL&Istituto Tecnico per Geometri\\
TN&Istituto Tecnico per Il Turismo\\
VC&Convitto Nazionale\\
VE&Educandato\\
\bottomrule
\end{tabular}
\captionof{table}{Codifica scuole}
\end{center}