
\newglossaryentry{abbonato}{name={Abbonato},description={Qualunque persona fisica, persona giuridica, ente o associazione parte di un contratto con un fornitore di servizi di comunicazione elettronica accessibili al pubblico per la fornitura di tali servizi, o comunque destinatario di tali servizi tramite schede prepagate.\mioindex{2003}{Abbonato}~\cite{DL2003}}}
\newglossaryentry{abilita}{name={Abilità},description={\begin{tabular}{cp{12cm}}\toprule 2005&Le abilità rappresentano il saper fare che una cultura reputa importante trasmettere alle nuove generazioni, per realizzare opere o conseguire scopi. È abile colui che non solo produce qualcosa o risolve problemi, ma colui che conosce anche le ragioni di questo “fare”, sa perché, operando in un certo modo e rispettando determinate procedure, si ottengono determinati risultati. Come le conoscenze, sono ordinate, nelle Indicazioni nazionali, per “discipline” e per “Educazione alla Convivenza civile” e costituiscono, con esse, gli “obiettivi specifici di apprendimento” che i docenti trasformano in obiettivi formativi~\cite{CIRC2005a}.\\ \midrule 2017-18	&Capacità di applicare le conoscenze e di usare il know-how per portare a termine compiti e risolvere problemi, Nel contesto dell'EQF, le abilità sono descritte come cognitive (comprendenti l'uso del pensiero logico, intuitivo e creativo) e pratiche (comprendenti la manualità e l'uso di metodi, materiali, strumenti e utensili)~\cite{RA2017}~\cite{RA2008}.\\ \bottomrule\end{tabular}\mioindex{2008}{Abilità}\mioindex{2017}{Abilità}\mioindex{2005}{Abilità}}}
\newglossaryentry{accreditamento}{name={Accreditamento},description={La procedura mediante la quale le Regioni e le Province autonome di Trento e Bolzano riconoscono a una istituzione scolastica di IP, l'idoneità a erogare percorsi di IeFP per il rilascio della qualifica e del diploma professionale quadriennale di cui all'art. 17 del decreto legislativo 17 ottobre 2005, n. 226~\cite{DL2018b}~\cite{DL2005a}.\mioindex{2018}{Accreditamento}}}
\newglossaryentry{addestramento}{name={Addestramento},description={Complesso delle attività dirette a fare apprendere ai lavoratori l'uso corretto di attrezzature, macchine, impianti, sostanze, dispositivi, anche di protezione individuale, e le procedure di lavoro\mioindex{2008}{Addestramento}~\cite{DL2008a}.}}
\newglossaryentry{addettoservizioprevenzioneprotezione}{name={Addetto al servizio di prevenzione e protezione},description={Persona in possesso delle capacità e dei requisiti professionali di cui all'articolo 32, facente parte del servizio di cui alla lettera l)\mioindex{2008}{Addetto!al servizio di prevenzione e protezione}~\cite{DL2008a}.}}
\newglossaryentry{agidg}{name={Agenzia per l'Italia Digitale},description={ha personalità giuridica di diritto pubblico ed è dotata di autonomia regolamentare, amministrativa, patrimoniale, organizzativa, contabile e finanziaria e persegue, nella sua attività, gli obiettivi di efficacia, efficienza, imparzialità, semplificazione e partecipazione dei cittadini e delle imprese. L'Agenzia è sottoposta ai poteri di indirizzo e vigilanza del Presidente del Consiglio dei Ministri o del Ministro da lui delegato ed al controllo della Corte dei conti. L'Agenzia svolge le funzioni ed i compiti ad essa attribuiti dalla legge al fine di perseguire il massimo livello di utilizzo delle tecnologie digitali nell'organizzazione della Pubblica Amministrazione e nel rapporto tra questa, i cittadini e le imprese, nel rispetto dei principi di legalità, imparzialità e trasparenza e secondo criteri di efficienza, economicità ed efficacia.\mioindex{2014}{Agenzia!per l'Italia Digitale}~\cite{DL2014b}}}
\newglossaryentry{allineamentodati}{name={Allineamento dei dati},description={Il processo di coordinamento dei dati presenti in più archivi finalizzato alla verifica della corrispondenza delle informazioni in essi contenute.\mioindex{2005}{Allineamento!dei dati}~\cite{DL2005c}}}
\newglossaryentry{alternanza}{name={Alternanza scuola-lavoro},description={\begin{tabular}{cp{12cm}}\toprule 2005&Alternanza scuola-lavoro modalità di realizzazione dei corsi del secondo ciclo, sia nel sistema dei licei, sia nel sistema dell'istruzione e della formazione professionale, per assicurare ai giovani, oltre alle conoscenze di base, l'acquisizione di competenze spendibili nel mercato del lavoro~\cite{DL2005}. \\\midrule 2016 & Per alternanza scuola-lavoro, si intende una metodologia didattica che consente agli studenti che frequentano gli istituti di istruzione superiore di svolgere una parte del proprio percorso formativo presso un'impresa o un ente, l'alternanza scuola lavoro si basa su una concezione integrata del processo educativo in cui il momento formativo, attuato mediante lo studio teorico d'aula, e il momento applicativo, attuato mediante esperienze assistite sul posto di lavoro, si fondono~\cite{INAIL2016}. \\\bottomrule\end{tabular}\mioindex{2005}{Alternanza!scuola-lavoro}\mioindex{2016}{Alternanza!scuola-lavoro}}}
\newglossaryentry{alunniconbes}{name={Alunni con BES},description={\begin{enumerate}\item Alunni che richiedono il diritto dell’attivazione della L.104/92 (Disabilità) o della L.170/2010 (DSA), per i quali è necessaria la certificazione;\\ \item alunni che presentano altri disturbi clinici che non danno diritto all’attivazione né della L.104/92 né alla L.170/2010, ma che sono ricompresi tra le classificazioni diagnostiche dei manuali nosografici di riferimento ICD e DSM-5 (a titolo di esempio, gli alunni con DCM, DSL, Spettro autistico ad alto funzionamento, ecc.) e per i quali è auspicabile presentare una diagnosi con profilo funzionale;\\\item Alunni che non necessitano di alcuna certificazione né diagnosi che rientrano nell’area dello svantaggio socioeconomico, linguistico e culturale individuati dalla scuola come indicato nella CM n. 8 del 06/03/2013. Indipendentemente dalle diagnosi o certificazioni, la scuola deve farsi carico delle difficoltà mostrate dall’alunno, in linea con le finalità di tutta la normativa sui BES\\.\end{enumerate}~\cite{CNOP2016}\mioindex{2016}{Alunni!BES}.}}
\newglossaryentry{ambientelavoro}{name={Ambiente di lavoro},description={Si intende non solo lo stabilimento aziendale, bensì anche un eventuale cantiere all'aperto o un luogo pubblico, purché in essi si svolga un progetto di alternanza scuola-lavoro e l'attività ivi svolta presenti le caratteristiche oggettive elencate dall'art.1, n. 28 del d.p.r, 1124/65~\cite{INAIL2016}.\mioindex{2016}{Ambiente di lavoro}}}
\newglossaryentry{analisifabbisognocompetenze}{name={Analisi del fabbisogno di competenze},description={L'analisi quantitativa o qualitativa disponibile di dati aggregati sulle competenze da fonti esistenti relative al mercato del lavoro e delle corrispondenti opportunità di apprendimento nel sistema di istruzione e formazione, che può contribuire all'orientamento e alla consulenza, alle procedure di assunzione, alla scelta del percorso di studi, di formazione e di carriera professionale~\cite{DE2018}.\mioindex{2018}{Analisi!del fabbisogno di competenze}\mioindex{2018}{Competenze! analisi del fabbisogno di competenze}}}
\newglossaryentry{apprendimentoformale}{name={Apprendimento formale},description={\begin{tabular}{cp{12cm}}\toprule 2012		&Apprendimento erogato in un contesto organizzato e strutturato, specificamente dedicato all'apprendimento, che di norma porta all'ottenimento di qualifiche, generalmente sotto forma di certificati o diplomi; comprende sistemi di istruzione generale, formazione professionale iniziale e istruzione superiore~\cite{RA2012}\mioindex{2012}{Apprendimento!formale}. \\ \midrule 2013		& Qualsiasi attività intrapresa dalla persona in modo formale, non formale e informale, nelle varie fasi della vita, al fine di migliorare le conoscenze, le capacità e le competenze, in una prospettiva di crescita personale, civica, sociale e occupazionale~\cite{DL2013}\mioindex{2013}{Apprendimento!formale}. \\ \midrule 2018		& Apprendimento che si attua nel sistema di istruzione e formazione e nelle università e istituzioni di alta formazione artistica, musicale e coreutica, e che si conclude con il conseguimento di un titolo di studio o di una qualifica o diploma professionale, conseguiti anche in apprendistato, o di una certificazione riconosciuta, nel rispetto della legislazione vigente in materia di ordinamenti scolastici e universitari, a norma dell'articolo 2, comma 1, lettera b), del decreto legislativo 16 gennaio 2013, n. 13~\cite{DL2018a}\mioindex{2018}{Apprendimento!formale}. \\ \bottomrule \end{tabular}}}
\newglossaryentry{apprendimentoinformale}{name={Apprendimento informale},description={\begin{tabular}{cp{12cm}} \toprule 2012		&Apprendimento risultante dalle attività della vita quotidiana legate al lavoro, alla famiglia o al tempo libero e non strutturato in termini di obiettivi di apprendimento, di tempi o di risorse dell'apprendimento; esso può essere non intenzionale dal punto di vista del discente; esempi di risultati di apprendimento acquisiti mediante l'apprendimento informale sono le abilità acquisite durante le esperienze di vita e lavoro come la capacità di gestire progetti o le abilità ITC acquisite sul lavoro; le lingue e le abilità interculturali acquisite durante il soggiorno in un altro paese; le abilità ITC acquisite al di fuori del lavoro, le abilità acquisite nel volontariato, nelle attività culturali e sportive, nel lavoro, nell'animazione socio educativa e mediante attività svolte in casa (ad esempio l'accudimento dei bambini)~\cite{RA2012}\mioindex{2012}{Apprendimento!informale}. \\ \midrule 2013-18		& Apprendimento che, anche a prescindere da una scelta intenzionale, si realizza nello svolgimento, da parte di ogni persona, di attività nelle situazioni di vita quotidiana e nelle interazioni che in essa hanno luogo, nell'ambito del contesto di lavoro, familiare e del tempo libero~\cite{DL2013}\mioindex{2013}{Apprendimento!informale}~\cite{DL2018a}\mioindex{2018}{Apprendimento!informale}. \\ \bottomrule\end{tabular}}}
\newglossaryentry{apprendimentononformale}{name={Apprendimento non formale},description={\begin{tabular}{cp{12cm}}\toprule 2013		&Apprendimento caratterizzato da una scelta intenzionale della persona, che si realizza al di fuori dei sistemi indicati alla lettera b), in ogni organismo che persegua scopi educativi e formativi, anche del volontariato, del servizio civile nazionale e del privato sociale e nelle imprese~\cite{DL2013}\mioindex{2013}{Apprendimento!non formale}. \\ \midrule 2018		& Apprendimento caratterizzato da una scelta intenzionale della persona, che si realizza al di fuori dei sistemi indicati per l'apprendimento formale, in ogni organismo che persegua scopi educativi e formativi, anche del volontariato, del servizio civile nazionale e del privato sociale e nelle imprese, a norma dell'articolo 2, comma 1, lettera c), del decreto legislativo 16 gennaio 2013, n. 13~\cite{DL2018a}\mioindex{2018}{Apprendimento!non formale}. \\ \bottomrule\end{tabular}}}
\newglossaryentry{apprendimentopermanente}{name={Apprendimento permanente},description={\begin{tabular}{cp{12cm}} \toprule 2013		&Qualsiasi attività intrapresa dalla persona in modo formale, non formale e informale, nelle varie fasi della vita, al fine di migliorare le conoscenze, le capacità e le competenze, in una prospettiva di crescita personale, civica, sociale e occupazionale~\cite{DL2013}\mioindex{2013}{Apprendimento!permanente}. \\ \midrule 2013		& L'intero complesso di istruzione generale, istruzione e formazione professionale, istruzione non formale e apprendimento informale intrapresi nel corso della vita che comporta un miglioramento delle conoscenze, delle abilità e delle competenze, che può includere l'etica professionale~\cite{DI2013}\mioindex{2013}{Apprendimento!permanente}. \\ \bottomrule\end{tabular}}}
\newglossaryentry{apprendistatoduale}{name={Apprendistato duale},description={L'apprendistato duale è una tipologia di contratto a causa mista che prevede la concomitanza di istruzione e formazione professionale, Tutte le tipologie di contratto di apprendistato si possono ricondurre al sistema duale~\cite{DL2015c}.\mioindex{2015}{Apprendistato!duale}}}
\newglossaryentry{archivio}{name={Archivio},description={Qualsiasi insieme strutturato di dati personali accessibili secondo criteri determinati, indipendentemente dal fatto che tale insieme sia centralizzato, decentralizzato o ripartito in modo funzionale o geografico\mioindex{2016}{Archivio}~\cite{RE2016}~\cite{DL2018d}.}}
\newglossaryentry{associazionetemporaneaimprese}{name={Associazione temporanea di imprese},description={Designa un insieme di imprenditori, o fornitori, o prestatori di servizi, costituito, anche mediante scrittura privata, allo scopo di partecipare alla procedura di affidamento di uno specifico contratto pubblico, mediante presentazione di una unica offerta~\cite{DL2006}.\mioindex{2006}{Associazione!temporanea di imprese}}}
\newglossaryentry{associazionetemporanescopo}{name={Associazione temporanea di scopo},description={L'associazione temporanea di scopo è un accordo in base al quale i partecipanti conferiscono ad uno di essi un mandato di rappresentanza nei confronti di un soggetto finanziatore, per la realizzazione di un progetto di interesse comune.}}
\newglossaryentry{atecogg}{name={ATECO},description={Strumento adottato dall'Istituto nazionale di statistica (ISTAT) per classificare e rappresentare le attività economiche~\cite{DL2018a}.\mioindex{2018}{ATECO}}}
\newglossaryentry{atlantelavorodellequalificazioni}{name={Atlante del lavoro e delle qualificazioni},description={Dispositivo classificatorio e informativo realizzato sulla base delle sequenze descrittive della Classificazione dei settori economico-professionali, anche ai sensi dell'art. 8 del decreto legislativo n. 13 del 2013 e dell'art. 3, comma 5, del decreto interministeriale del 30 giugno 2015, e parte integrante dei sistemi informativi di cui agli articoli 13 e 15 del decreto legislativo n. 150 del 2015~\cite{DL2018}.\mioindex{2018}{Atlante del lavoro!e delle qualificazioni}}}
\newglossaryentry{attestazioneparteprima}{name={Attestazione di parte prima},description={Attestazione la cui validità delle informazioni contenute è data dalla autodichiarazione della persona, anche laddove attuata con un percorso accompagnato e realizzata attraverso procedure e modulistiche predefinite~\cite{DL2015a}.\mioindex{2015}{Attestazione!di parte prima}}}
\newglossaryentry{attestazioneparteseconda}{name={Attestazione di parte seconda},description={Attestazione rilasciata su responsabilità dell'ente titolato che eroga servizi di individuazione e validazione e certificazione delle competenze, in rapporto agli elementi di regolamentazione e garanzia del processo in capo all'ente titolare ai sensi del decreto legislativo 16 gennaio 2013, n. 13~\cite{DL2015a}.\mioindex{2015}{Attestazione!di parte seconda}}}
\newglossaryentry{attestazioneparteterza}{name={Attestazione di parte terza},description={Attestazione rilasciata su responsabilità dell'ente titolare, con il supporto dell'ente titolato che eroga i servizi di individuazione e validazione e certificazione delle competenze ai sensi del decreto legislativo 16 gennaio 2013, n. 13~\cite{DL2015a}.\mioindex{2015}{Attestazione!di parte terza}}}
\newglossaryentry{attivitlavororiservata}{name={Attività di lavoro riservata},description={Attività di lavoro riservata a persone iscritte in albi o elenchi ai sensi dell'art. 2229 del codice civile nonchè alle professioni sanitarie e ai mestieri artigianali, commerciali e di pubblico esercizio disciplinati da specifiche normative~\cite{DL2015a}.\mioindex{2015}{Attività!di lavoro riservata}}}
\newglossaryentry{autenticazioneinformatican}{name={Autenticazione informatica},description={L'insieme degli strumenti elettronici e delle procedure per la verifica anche indiretta dell'identità.\mioindex{2003}{Autenticazione!informatica}~\cite{DL2003}}}
\newglossaryentry{autenticazioneinformatica}{name={Autenticazione informatica},description={La validazione dell'insieme di dati attribuiti in modo esclusivo ed univoco ad un soggetto, che ne distinguono l'identità nei sistemi informativi, effettuata attraverso opportune tecnologie al fine di garantire la sicurezza dell'accesso.\mioindex{2005}{Autenticazione!informatica}~\cite{DL2005c}}}
\newglossaryentry{autenticazione}{name={Autenticazione},description={Un processo elettronico che consente di confermare l’identificazione elettronica di una persona fisica o giuridica, oppure l’origine e l’integrità di dati in forma elettronica.\mioindex{2014}{Autenticazione}~\cite{RE2014}}}
\newglossaryentry{autoritacontrollointeressata}{name={Autorità di controllo interessata},description={Un'autorità di controllo interessata dal trattamento di dati personali in quanto\begin{enumerate}\item il titolare del trattamento o il responsabile del trattamento è stabilito sul territorio dello Stato membro di tale autorità di controllo;\item gli interessati che risiedono nello Stato membro dell'autorità di controllo sono o sono probabilmente influenzati in modo sostanziale dal trattamento;\item oppure un reclamo è stato proposto a tale autorità di controllo\end{enumerate}\mioindex{2016}{Autorità!di controllo interessata}~\cite{RE2016}~\cite{DL2018d}.}}
\newglossaryentry{autoritacontrollo}{name={Autorità di controllo},description={L'autorità pubblica indipendente istituita da uno Stato membro ai sensi dell'articolo 51\mioindex{2016}{Autorità!di controllo}~\cite{RE2016}~\cite{DL2018d}.}}
\newglossaryentry{autoritcompetente}{name={Autorità competente},description={Qualsiasi autorità o organismo abilitato da uno Stato membro in particolare a rilasciare o a ricevere titoli di formazione e altri documenti o informazioni, nonché a ricevere le domande e ad adottare le decisioni di cui alla presente direttiva~\cite{DI2005}.\mioindex{2005}{Autorità!competente}}}
\newglossaryentry{autovalutazionedellecompetenze}{name={Autovalutazione delle competenze},description={Il processo di riflessione sistematica della persona sulle proprie competenze tramite il riferimento a descrizioni fisse delle competenze~\cite{DE2018}.\mioindex{2018}{Autovalutazione!delle competenze}\mioindex{2018}{Competenze!autovalutazione}}}
\newglossaryentry{azienda}{name={Azienda},description={Il complesso della struttura organizzata dal datore di lavoro pubblico o privato\mioindex{2008}{Azienda}~\cite{DL2008a}.}}
\newglossaryentry{bancadati}{name={Banca di dati},description={Qualsiasi complesso organizzato di dati personali, ripartito in una o più unità dislocate in uno o più siti.\mioindex{2003}{Banca!di dati}~\cite{DL2003}}}
\newglossaryentry{besg}{name={BES},description={Ogni alunno, con continuità o per determinati periodi, può manifestare Bisogni Educativi Speciali: o per motivi fisici, biologici, fisiologici o anche per motivi psicologici, sociali, rispetto ai quali è necessario che le scuole offrano adeguata e personalizzata risposta\mioindex{2012}{BES}.~\cite{DL2012}}}
\newglossaryentry{bilanciocompetenze}{name={Bilancio di competenze},description={Processo volto all'individuazione e all'analisi delle conoscenze, abilità e competenze di una persona, comprese attitudini e motivazioni, per definire un progetto professionale e/o pianificare un progetto di riorientamento o formazione professionale; lo scopo di un bilancio di competenze è di aiutare una persona ad analizzare il profilo professionale acquisito, a comprendere la propria posizione nel mondo del lavoro e a progettare una carriera professionale, o in taluni casi, a prepararsi in vista della convalida dei risultati dell'apprendimento non formale o informale~\cite{RA2012}.\mioindex{2012}{Competenze!bilancio}\mioindex{2012}{Bilancio!di competenze}}}
\newglossaryentry{bilanciopersonalegg}{name={Bilancio personale},description={Strumento che evidenzia i saperi e le competenze acquisiti da ciascuna studentessa e da ciascuno studente, anche in modo non formale e informale, idoneo a rilevare le potenzialità e le carenze riscontrate~\cite{DL2018a}.\mioindex{2018}{Bilancio!personale}}}
\newglossaryentry{blocco}{name={Blocco},description={La conservazione di dati personali con sospensione temporanea di ogni altra operazione del trattamento.\mioindex{2003}{Blocco}~\cite{DL2003}}}
\newglossaryentry{buoneprassi}{name={Buone prassi},description={Soluzioni organizzative o procedurali coerenti con la normativa vigente e con le norme di buona tecnica, adottate volontariamente e finalizzate a promuovere la salute e sicurezza sui luoghi di lavoro attraverso la riduzione dei rischi e il miglioramento delle condizioni di lavoro, elaborate e raccolte dalle regioni, dall'Istituto superiore per la prevenzione e la sicurezza del lavoro (ISPESL), dall'Istituto nazionale per l'assicurazione contro gli infortuni sul lavoro (INAIL) e dagli organismi paritetici di cui all'articolo 51, validate dalla Commissione consultiva permanente di cui all'articolo 6, previa istruttoria tecnica dell'ISPESL, che provvede a assicurarne la più ampia diffusione\mioindex{2008}{Buone!prassi}~\cite{DL2008a}.}}
\newglossaryentry{cartaidentitelettronica}{name={Carta d'identità elettronica},description={Il documento d'identità munito di fotografia del titolare rilasciato su supporto informatico dalle amministrazioni comunali con la prevalente finalità di dimostrare l'identità anagrafica del suo titolare.\mioindex{2005}{Carta!d'identità elettronica}~\cite{DL2005c}}}
\newglossaryentry{cartanazionaleservizi}{name={Carta nazionale dei servizi},description={Il documento rilasciato su supporto informatico per consentire l'accesso per via telematica ai servizi erogati dalle pubbliche amministrazioni.\mioindex{2005}{Carta!nazionale dei servizi}~\cite{DL2005c}}}
\newglossaryentry{certificatielettronici}{name={Certificati elettronici},description={Gli attestati elettronici che collegano i dati utilizzati per verificare le firme elettroniche ai titolari e confermano l'identità informatica dei titolari stessi.\mioindex{2005}{Certificati!elettronici}~\cite{DL2005c}}}
\newglossaryentry{certificatoautenticazionesitoweb}{name={Certificato di autenticazione di sito web},description={Un servizio elettronico di recapito certificato che soddisfa i requisiti di cui all’articolo 44. Vedi~\cref{sec:articolo44reg9102014}\mioindex{2014}{Certificato!di autenticazione di sito web}~\cite{RE2014}}}
\newglossaryentry{certificatofirmaelettronica}{name={Certificato di firma elettronica},description={Un attestato elettronico che collega i dati di convalida di una firma elettronica a una persona fisica e conferma almeno il nome o lo pseudonimo di tale persona.\mioindex{2014}{Certificato!di firma elettronica}~\cite{RE2014}}}
\newglossaryentry{certificatoqualificatoautenticazionesitoweb}{name={Certificato qualificato di autenticazione di sito web},description={Un certificato di autenticazione di sito web che è rilasciato da un prestatore di servizi fiduciari qualificato ed è conforme ai requisiti di cui all’allegato IV. Vedi~\cref{sec:allegatoIVreg9102014}\mioindex{2014}{Certificato!qualificato di autenticazione di sito web}~\cite{RE2014}}}
\newglossaryentry{certificatoqualificatofirmaelettronica}{name={Certificato qualificato di firma elettronica},description={Un certificato di firma elettronica che è rilasciato da un prestatore di servizi fiduciari qualificato ed è conforme ai requisiti di cui all’allegato I. Vedi~\cref{sec:allegatoIreg9102014}\mioindex{2014}{Certificato!qualificato di firma elettronica}~\cite{RE2014}}}
\newglossaryentry{certificatoqualificatosigilloelettronico}{name={Certificato qualificato di sigillo elettronico},description={Un certificato di sigillo elettronico che è rilasciato da un prestatore di servizi fiduciari qualificato ed è conforme ai requisiti di cui all’allegato III. Vedi~\cref{sec:allegatoIIIreg9102014}\mioindex{2014}{Certificato!qualificato di sigillo elettronico}~\cite{RE2014}},see={sigilloelettronico}}
\newglossaryentry{certificatoqualificato}{name={Certificato qualificato},description={Il certificato elettronico conforme ai requisiti di cui all'allegato I della direttiva 1999/93/CE, rilasciati da certificatori che rispondono ai requisiti di cui all'allegato II della medesima direttiva.\mioindex{2005}{Certificato!qualificato}~\cite{DL2005c}}}
\newglossaryentry{certificatore}{name={Certificatore},description={Il soggetto che presta servizi di certificazione delle firme elettroniche o che fornisce altri servizi connessi con queste ultime.\mioindex{2005}{Certificatore}~\cite{DL2005c}}}
\newglossaryentry{certificatosigilloelettronico}{name={Certificato di sigillo elettronico},description={I dati unici utilizzati dal creatore del sigillo elettronico per creare un sigillo elettronico.\mioindex{2014}{Certificato!di sigillo elettronico}~\cite{RE2014}},see={sigilloelettronico}}
\newglossaryentry{certificazionecompetenze}{name={Certificazione delle competenze},description={\begin{tabular}{cp{12cm}} \toprule 2005		&La certificazione delle competenze scaturisce dalla somma qualitativa e quantitativa delle rilevazioni e degli accertamenti effettuati nel percorso scolastico, coinvolge nella maniera professionalmente più alta i docenti, perché si assumono la responsabilità di certificarle a livello iniziale, intermedio ed esperto. È prevista anche una certificazione delle competenze degli allievi nel superamento delle prove di esame. Va sottolineato che questa competenza si aggiunge, e non si sostituisce, a quelle identificate nel Profilo~\cite{CIRC2005a}\mioindex{2005}{Competenze!certificazione}. \\ \midrule 2013		&Procedura di formale riconoscimento, da parte dell'ente titolato, in base alle norme generali, ai livelli essenziali delle prestazioni e agli standard minimi di cui al presente decreto, delle competenze acquisite dalla persona in contesti formali, anche in caso di interruzione del percorso formativo, o di quelle validate acquisite in contesti non formali e informali. La procedura di certificazione delle competenze si conclude con il rilascio di un certificato conforme agli standard minimi~\cite{DL2013}\mioindex{2013}{Competenze!certificazione}. \\ \midrule 2018		& Procedura di formale riconoscimento, da parte dell'ente titolato a norma dell'articolo 2, lettera g), del decreto legislativo 16 gennaio 2013, n. 13, in base alle norme generali, ai livelli essenziali delle prestazioni e agli standard minimi di cui al medesimo decreto legislativo, delle competenze acquisite dalla persona in contesti formali, anche in caso di interruzione del percorso formativo, o di quelle validate acquisite in contesti non formali e informali. La procedura di certificazione delle competenze si conclude con il rilascio di un certificato conforme agli standard minimi di cui all'articolo 6 del decreto legislativo n. 13 del 2013~\cite{DL2018a}\mioindex{2018}{Competenze!certificazione}. \\ \bottomrule \end{tabular}}}
\newglossaryentry{certificazione}{name={Certificazióne},description={L'insieme delle operazioni tecnico-amministrative che un organo tecnico, per lo più pubblico, espleta al fine di garantire la conformità di un prodotto o di un servizio alle norme vigenti.~\cite{TRECCANI2020a}}}
\newglossaryentry{cheating}{name={Cheating},description={Comportamento scorretto che non rende attendibili i risultati delle valutazioni scolastiche.}}
\newglossaryentry{chiamata}{name={Chiamata},description={La connessione istituita da un servizio telefonico accessibile al pubblico, che consente la comunicazione bidirezionale in tempo reale.\mioindex{2003}{Chiamata}~\cite{DL2003}}}
\newglossaryentry{chiaveprivata}{name={Chiave privata},description={L'elemento della coppia di chiavi asimmetriche, utilizzato dal soggetto titolare, mediante il quale si appone la firma digitale sul documento informatico.\mioindex{2005}{Chiave!privata}~\cite{DL2005c}}}
\newglossaryentry{chiavepubblica}{name={Chiave pubblica},description={L'elemento della coppia di chiavi asimmetriche destinato ad essere reso pubblico, con il quale si verifica la firma digitale apposta sul documento informatico dal titolare delle chiavi asimmetriche.\mioindex{2005}{Chiave!pubblica}~\cite{DL2005c}}}
\newglossaryentry{classiaggiornamento}{name={Classi di aggiornamento},description={Nella scuola mediaè data facoltà di istituire classi di aggiornamento che si affiancano alla prima e alla terza. Alla prima classe di aggiornamento possono accedere gli alunni bisognosi di particolari cure per frequentare con profitto la prima classe di scuola media. Alla terza classe di aggiornamento possono accedere gli alunni che non abbiano conseguito la licenza di scuola media perchè respinti. Le classi di aggiornamento non possono avere più di 15 alunni ciascuna: ad esse vengono destinati insegnanti particolarmente qualificati~\cite{LEGGE1962}\mioindex{1953}{Classi!aggiornamento}. Abbolite con legge 4 agosto 1977, n. 517~\cite{LEGGE1977}}}
\newglossaryentry{classidifferenziali}{name={Classi differenziali},description={Le classi differenziali non sono istituti scolastici a sé stanti, ma funzionano presso le comuni scuole elementari ed accolgono gli alunni nervosi, tardivi, instabili, i quali rivelano l'inadattabilità alla disciplina comune e ai normali metodi e ritmi d'insegnamento e possono raggiungere un livello migliore solo se l'insegnamento viene ad essi impartito con modi e forme particolari~\cite{CIRC1953}. Abbolite con legge 4 agosto 1977, n. 517~\cite{LEGGE1977}.\mioindex{1953}{Classi!differenziali}}}
\newglossaryentry{classificazione deisettorieconomicoprofessionali}{name={Classificazione dei settori economico professionali},description={\begin{tabular}{cp{12cm}} \toprule 2015-18		&Sistema di classificazione che, a partire dai codici di classificazione statistica ISTAT relativi alle attività economiche (ATECO) e alle professioni (Classificazione delle professioni), consente di aggregare in settori l'insieme delle attività e delle professionalità operanti sul mercato del lavoro, I settori economico-professionali sono articolati secondo una sequenza descrittiva che prevede la definizione di: comparti, processi di lavoro, aree di attività, attività di lavoro e ambiti tipologici di esercizio~\cite{DL2015a}\mioindex{2015}{Classificazione!dei settori economico professionali}~\cite{DL2018a}\mioindex{2018}{Classificazione!dei settori economico professionali}. \\ \midrule 2019		& La classificazione dei settori economico professionali (SEP) è stata ottenuta utilizzando i codici delle classificazioni adottate dall'ISTAT relativamente alle attività economiche (ATECO 2007) e alle professioni (Classificazione delle professioni 2011). La classificazione SEP è composta da 23 settori, più un settore denominato Area COMUNE, che comprende tutte quelle attività lavorative non caratterizzate da un settore di riferimento specifico~\cite{LG2019a}\mioindex{2019}{Classificazione!dei settori economico professionali}. \\ \bottomrule\end{tabular}}}
\newglossaryentry{classispeciali}{name={Classi speciali},description={Le classi speciali per minorati e quelle di differenziazione didattica sono istituti scolastici nei quali viene impartito l'insegnamento elementare ai fanciulli aventi determinate minorazioni fisiche o psichiche ed istituti nei quali vengono adottati speciali metodi didattici per l'insegnamento ai ragazzi anormali, es. scuole Montessori~\cite{CIRC1953}\mioindex{1953}{Classi!speciali}.}}
\newglossaryentry{codiceamministrazionedigitale}{name={Codice dell'amministrazione Digitale},description={Elenco di norme che regolano l'informatizzazione della pubblica amminitrazione. Nato con il dl. 7 marzo 2005 n. 82, è stato successivamente modificato con il dl. 22 agosto 2016 n 179 e poi dal dl. 13 dicembre 2017 n.217.\mioindex{2017}{Codice!dell'amministrazione Digitale}\mioindex{2016}{Codice!dell'amministrazione Digitale} \mioindex{2005}{Codice!dell'amministrazione Digitale}~\cite{DL2005c}~\cite{DL2016a}~\cite{DL2017d}}}
\newglossaryentry{comitatotecnicoscientifico}{name={Comitato Tecnico Scientifico},description={Composto da docenti e da esperti del mondo del lavoro, delle professioni e della ricerca scientifica e tecnologica, con funzioni consultive e di proposta per l'organizzazione delle aree di indirizzo e l'utilizzazione degli spazi di autonomia e flessibilità; ai componenti del comitato non spettano compensi ad alcun titolo~\cite{DL2010a}~\cite{DL2017a}.\mioindex{2010}{Comitato!Tecnico Scientifico}\mioindex{2017}{Comitato!Tecnico Scientifico}}}
\newglossaryentry{comorbilit}{name={Comorbilità},description={La comorbilità o comorbidità in ambito medico indica la coesistenza di più patologie diverse in uno stesso individuo.~\cite{Wikipedia2019c}}}
\newglossaryentry{competenzaalfabeticafunzionale}{name={Competenza alfabetica funzionale},description={La competenza alfabetica funzionale indica la capacità di individuare, comprendere, esprimere, creare e interpretare concetti, sentimenti, fatti e opinioni, in forma sia orale sia scritta, utilizzando materiali visivi, sonori e digitali attingendo a varie discipline e contesti. Essa implica l'abilità di comunicare e relazionarsi efficacemente con gli altri in modo opportuno e creativo. Il suo sviluppo costituisce la base per l'apprendimento successivo e l'ulteriore interazione linguistica, a seconda del contesto, la competenza alfabetica funzionale può essere sviluppata nella lingua madre, nella lingua dell'istruzione scolastica e/o nella lingua ufficiale di un paese o di una regione~\cite{RA2018}.\mioindex{2018}{Competenza!alfabetica funzionale}}}
\newglossaryentry{competenzaconsapevolezzaespressioneculturali}{name={Competenza consapevolezza ed espressione culturali},description={Consapevolezza dell'importanza dell'espressione creativa di idee, esperienze ed emozioni in un'ampia varietà di mezzi di comunicazione, compresi la musica, le arti dello spettacolo, la letteratura e le arti visive~\cite{RA2006}.\mioindex{2006}{Competenza!consapevolezza ed espressione culturali}}}
\newglossaryentry{competenzadigitalegg}{name={Competenza digitale},description={\begin{tabular}{cp{12cm}}\toprule 2006&La competenza digitale consiste nel saper utilizzare con dimestichezza e spirito critico le tecnologie della società dell'informazione (TSI) per il lavoro, il tempo libero e la comunicazione. Essa è supportata da abilità di base nelle TIC: l'uso del computer per reperire, valutare, conservare, produrre, presentare e scambiare informazioni nonché per comunicare e partecipare a reti collaborative tramite Internet~\cite{RA2006}\mioindex{2006}{Competenza!digitale}.\\\midrule 2018&La competenza digitale presuppone l'interesse per le tecnologie digitali e il loro utilizzo con dimestichezza e spirito critico e responsabile per apprendere, lavorare e partecipare alla società. Essa comprende l'alfabetizzazione informatica e digitale, la comunicazione e la collaborazione, l'alfabetizzazione mediatica, la creazione di contenuti digitali (inclusa la programmazione), la sicurezza (compreso l'essere a proprio agio nel mondo digitale e possedere competenze relative alla cibersicurezza), le questioni legate alla proprietà intellettuale, la risoluzione di problemi e il pensiero critico~\cite{RA2018}\mioindex{2018}{Competenza!digitale}.\end{tabular}}}
\newglossaryentry{competenzaiingegneria}{name={Competenza tecnologie e ingegneria},description={Le competenze in tecnologie e ingegneria sono applicazioni di tali conoscenze e metodologie per dare risposta ai desideri o ai bisogni avvertiti dagli esseri umani. La competenza in scienze, tecnologie e ingegneria implica la comprensione dei cambiamenti determinati dall'attività umana e della responsabilità individuale del cittadino~\cite{RA2018}\mioindex{2018}{Competenza!tecnologie e ingegneria}.}}
\newglossaryentry{competenzaimparareimparare}{name={Competenza imparare a imparare},description={Imparare a imparare è l'abilità di perseverare nell'apprendimento, di organizzare il proprio apprendimento anche mediante una gestione efficace del tempo e delle informazioni, sia a livello individuale che in gruppo, Questa competenza comprende la consapevolezza del proprio processo di apprendimento e dei propri bisogni, l'identificazione delle opportunità disponibili e la capacità di sormontare gli ostacoli per apprendere in modo efficace, Questa competenza comporta l'acquisizione, l'elaborazione e l'assimilazione di nuove conoscenze e abilità come anche la ricerca e l'uso delle opportunità di orientamento, Il fatto di imparare a imparare fa sì che i discenti prendano le mosse da quanto hanno appreso in precedenza e dalle loro esperienze di vita per usare e applicare conoscenze e abilità in tutta una serie di contesti: a casa, sul lavoro, nell'istruzione e nella formazione. La motivazione e la fiducia sono elementi essenziali perché una persona possa acquisire tale competenza~\cite{RA2006}\mioindex{2006}{Competenza!imparare a imparare}.}}
\newglossaryentry{competenzaimprenditoriale}{name={Competenza imprenditoriale},description={La competenza imprenditoriale si riferisce alla capacità di agire sulla base di idee e opportunità e di trasformarle in valori per gli altri, Si fonda sulla creatività, sul pensiero critico e sulla risoluzione di problemi, sull'iniziativa e sulla perseveranza, nonché sulla capacità di lavorare in modalità collaborativa al fine di programmare e gestire progetti che hanno un valore culturale, sociale o finanziario~\cite{RA2018}\mioindex{2018}{Competenza!imprenditoriale}.}}
\newglossaryentry{competenzainmateriaconsapevolezza}{name={Competenza in materia di consapevolezza ed espressione culturali},description={La competenza in materia di consapevolezza ed espressione culturali implica la comprensione e il rispetto di come le idee e i significati vengono espressi creativamente e comunicati in diverse culture e tramite tutta una serie di arti e altre forme culturali, presuppone l'impegno di capire, sviluppare ed esprimere le proprie idee e il senso della propria funzione o del proprio ruolo nella società in una serie di modi e contesti~\cite{RA2018}\mioindex{2018}{Competenza!in materia di consapevolezza ed espressione culturali}.}}
\newglossaryentry{competenzainmateriadicittadinanza}{name={Competenza in materia di cittadinanza},description={La competenza in materia di cittadinanza si riferisce alla capacità di agire da cittadini responsabili e di partecipare pienamente alla vita civica e sociale, in base alla comprensione delle strutture e dei concetti sociali, economici, giuridici e politici oltre che dell'evoluzione a livello globale e della sostenibilità~\cite{RA2018}\mioindex{2018}{Competenza!in materia di cittadinanza}.}}
\newglossaryentry{competenzainscienze}{name={Competenza in scienze},description={La competenza in scienze si riferisce alla capacità di spiegare il mondo che ci circonda usando l'insieme delle conoscenze e delle metodologie, comprese l'osservazione e la sperimentazione, per identificare le problematiche e trarre conclusioni che siano basate su fatti empirici, e alla disponibilità a farlo~\cite{RA2018}\mioindex{2018}{Competenza!in scienze}.}}
\newglossaryentry{competenzamatematica}{name={Competenza matematica},description={\begin{tabular}{cp{12cm}}\toprule 2006&La competenza matematica è l'abilità di sviluppare e applicare il pensiero matematico per risolvere una serie di problemi in situazioni quotidiane, partendo da una solida padronanza delle competenze aritmetico-matematiche, l'accento è posto sugli aspetti del processo e dell'attività oltre che su quelli della conoscenza. La competenza matematica comporta, in misura variabile, la capacità e la disponibilità a usare modelli matematici di pensiero (pensiero logico e spaziale) e di presentazione (formule, modelli, costrutti, grafici, carte)~\cite{RA2006}\mioindex{2006}{Competenza!matematica}. \\\midrule 2018& La competenza matematica è la capacità di sviluppare e applicare il pensiero e la comprensione matematici per risolvere una serie di problemi in situazioni quotidiane, partendo da una solida padronanza della competenza aritmetico-matematica, l'accento è posto sugli aspetti del processo e dell'attività oltre che sulla conoscenza. La competenza matematica comporta, a differenti livelli, la capacità di usare modelli matematici di pensiero e di presentazione (formule, modelli, costrutti, grafici, diagrammi) e la disponibilità a farlo~\cite{RA2018}\mioindex{2018}{Competenza!matematica}. \\\bottomrule\end{tabular}}}
\newglossaryentry{competenzamultilinguistica}{name={Competenza multilinguistica},description={Tale competenza definisce la capacità di utilizzare diverse lingue in modo appropriato ed efficace allo scopo di comunicare, In linea di massima essa condivide le abilità principali con la competenza alfabetica: si basa sulla capacità di comprendere, esprimere e interpretare concetti, pensieri, sentimenti, fatti e opinioni in forma sia orale sia scritta (comprensione orale, espressione orale, comprensione scritta ed espressione scritta) in una gamma appropriata di contesti sociali e culturali a seconda dei desideri o delle esigenze individuali. Le competenze linguistiche comprendono una dimensione storica e competenze interculturali, Tale competenza si basa sulla capacità di mediare tra diverse lingue e mezzi di comunicazione, come indicato nel quadro comune europeo di riferimento, Secondo le circostanze, essa può comprendere il mantenimento e l'ulteriore sviluppo delle competenze relative alla lingua madre, nonché l'acquisizione della lingua ufficiale o delle lingue ufficiali di un paese~\cite{RA2018}\mioindex{2018}{Competenza!multilinguistica}.}}
\newglossaryentry{competenzansoiniziativaimprenditorialit}{name={Competenza senso di iniziativa e di imprenditorialità},description={Il senso di iniziativa e l'imprenditorialità concernono la capacità di una persona di tradurre le idee in azione, In ciò rientrano la creatività, l'innovazione e l'assunzione di rischi, come anche la capacità di pianificare e di gestire progetti per raggiungere obiettivi, È una competenza che aiuta gli individui, non solo nella loro vita quotidiana, nella sfera domestica e nella società, ma anche nel posto di lavoro, ad avere consapevolezza del contesto in cui operano e a poter cogliere le opportunità che si offrono ed è un punto di partenza per le abilità e le conoscenze più specifiche di cui hanno bisogno coloro che avviano o contribuiscono ad un'attività sociale o commerciale. Essa dovrebbe includere la consapevolezza dei valori etici e promuovere il buon governo~\cite{RA2006}\mioindex{2006}{Competenza!senso di iniziativa e di imprenditorialità}.}}
\newglossaryentry{competenzapersonalesociale}{name={Competenza personale, sociale e capacità di imparare a imparare},description={La competenza personale, sociale e la capacità di imparare a imparare consiste nella capacità di riflettere su sé stessi, di gestire efficacemente il tempo e le informazioni, di lavorare con gli altri in maniera costruttiva, di mantenersi resilienti e di gestire il proprio apprendimento e la propria carriera. Comprende la capacità di far fronte all'incertezza e alla complessità, di imparare a imparare, di favorire il proprio benessere fisico ed emotivo, di mantenere la salute fisica e mentale, nonché di essere in grado di condurre una vita attenta alla salute e orientata al futuro, di empatizzare e di gestire il conflitto in un contesto favorevole e inclusivo~\cite{RA2018}.\mioindex{2018}{Competenza!personale, sociale e capacità di imparare a imparare}}}
\newglossaryentry{competenzasocialiciviche}{name={Competenza sociali e civiche},description={Queste includono competenze personali, interpersonali e interculturali e riguardano tutte le forme di comportamento che consentono alle persone di partecipare in modo efficace e costruttivo alla vita sociale e lavorativa, in particolare alla vita in società sempre più diversificate, come anche a risolvere i conflitti ove ciò sia necessario. La competenza civica dota le persone degli strumenti per partecipare appieno alla vita civile grazie alla conoscenza dei concetti e delle strutture sociopolitici e all'impegno a una partecipazione attiva e democratica~\cite{RA2006}.\mioindex{2006}{Competenza!sociali e civiche}}}
\newglossaryentry{competenza}{name={Competenza},description={\begin{tabular}{cp{12cm}}\toprule 2005&La competenza è l'agire personale di ciascuno, basato sulle conoscenze e abilità acquisite, adeguato, in un determinato contesto, in modo soddisfacente e socialmente riconosciuto, a rispondere ad un bisogno, a risolvere un problema, a eseguire un compito, a realizzare un progetto. Non è mai un agire semplice, atomizzato, astratto, ma è sempre un agire complesso che coinvolge tutta la persona e che connette in maniera unitaria e inseparabile i saperi (conoscenze) e i saper fare (abilità), i comportamenti individuali e relazionali, gli atteggiamenti emotivi, le scelte valoriali, le motivazioni e i fini. Per questo, nasce da una continua interazione tra persona, ambiente e società, e tra significati personali e sociali, impliciti ed espliciti~\cite{CIRC2005a}\mioindex{2005}{Competenza}. \\ \midrule 2008& Comprovata capacità di utilizzare conoscenze, abilità e capacità personali, sociali e/o metodologiche, in situazioni di lavoro o di studio e nello sviluppo professionale e personale, Nel contesto del Quadro europeo delle qualifiche le competenze sono descritte in termini di responsabilità e autonomia~\cite{RA2008}\mioindex{2008}{Competenza}. \\ \midrule 2013-18&Comprovata capacità di utilizzare, in situazioni di lavoro, di studio o nello sviluppo professionale e personale, un insieme strutturato di conoscenze e di abilità acquisite nei contesti di apprendimento formale, non formale o informale~\cite{DL2013}~\cite{DL2018a}\mioindex{2013}{Competenza}\mioindex{2018}{Competenza}. \\\midrule 2017& Comprovata capacità di utilizzare conoscenze, abilità e capacità personali, sociali e/o metodologiche in situazioni di lavoro o di studio e nello sviluppo professionale e personale~\cite{RA2017}\mioindex{2017}{Competenza}. \\ \midrule 2018& Ciò che una persona sa, capisce ed è capace di fare~\cite{DE2018}\mioindex{2018}{Competenza}.\\\bottomrule \end{tabular}}}
\newglossaryentry{competenzeComunicazione nella madrelingua}{name={Competenza comunicazione nella madrelingua},description={La comunicazione nella madrelingua è la capacità di esprimere e interpretare concetti, pensieri, sentimenti, fatti e opinioni in forma sia orale sia scritta (comprensione orale, espressione orale, comprensione scritta ed espressione scritta) e di interagire adeguatamente e in modo creativo sul piano linguistico in un'intera gamma di contesti culturali e sociali, quali istruzione e formazione, lavoro, vita domestica e tempo libero~\cite{RA2006}\mioindex{2006}{Competenza!comunicazione nella madrelingua}.}}
\newglossaryentry{competenzeComunicazionelinguestraniere}{name={Competenza comunicazione in lingue straniere},description={La comunicazione nelle lingue straniere condivide essenzialmente le principali abilità richieste per la comunicazione nella madrelingua: essa si basa sulla capacità di comprendere, esprimere e interpretare concetti, pensieri, sentimenti, fatti e opinioni in forma sia orale sia scritta — comprensione orale, espressione orale, comprensione scritta ed espressione scritta — in una gamma appropriata di contesti sociali e culturali — istruzione e formazione, lavoro, casa, tempo libero — a seconda dei desideri o delle esigenze individuali. La comunicazione nelle lingue straniere richiede anche abilità quali la mediazione e la comprensione interculturale, Il livello di padronanza di un individuo varia inevitabilmente tra le quattro dimensioni (comprensione orale, espressione orale, comprensione scritta ed espressione scritta) e tra le diverse lingue e a seconda del suo background sociale e culturale, del suo ambiente e delle sue esigenze e/o dei suoi interessi~\cite{RA2006}\mioindex{2006}{Competenza!comunicazione in lingue straniere}.}}
\newglossaryentry{competenzechiave}{name={Competenze chiave},description={Le competenze sono definite come una combinazione di conoscenze, abilità e atteggiamenti, in cui: d) la conoscenza si compone di fatti e cifre, concetti, idee e teorie che sono già stabiliti e che forniscono le basi per comprendere un certo settore o argomento; e) per abilità si intende sapere ed essere capaci di eseguire processi ed applicare le conoscenze esistenti al fine di ottenere risultati; f) gli atteggiamenti descrivono la disposizione e la mentalità per agire o reagire a idee, persone o situazioni. Le competenze chiave sono quelle di cui tutti hanno bisogno per la realizzazione e lo sviluppo personali, l'occupabilità, l'inclusione sociale, uno stile di vita sostenibile, una vita fruttuosa in società pacifiche, una gestione della vita attenta alla salute e la cittadinanza attiva. Esse si sviluppano in una prospettiva di apprendimento permanente, dalla prima infanzia a tutta la vita adulta, mediante l'apprendimento formale, non formale e informale in tutti i contesti, compresi la famiglia, la scuola, il luogo di lavoro, il vicinato e altre comunità~\cite{RA2018}\mioindex{2018}{Competenze!chiave}.}}
\newglossaryentry{competenzecompetenzabasecamposcientificotecnologico}{name={Competenza di base in campo scientifico e tecnologico},description={La competenza in campo scientifico si riferisce alla capacità e alla disponibilità a usare l'insieme delle conoscenze e delle metodologie possedute per spiegare il mondo che ci circonda sapendo identificare le problematiche e traendo le conclusioni che siano basate su fatti comprovati. La competenza in campo tecnologico è considerata l'applicazione di tale conoscenza e metodologia per dare risposta ai desideri o bisogni avvertiti dagli esseri umani. La competenza in campo scientifico e tecnologico comporta la comprensione dei cambiamenti determinati dall'attività umana e la consapevolezza della responsabilità di ciascun cittadino~\cite{RA2006}\mioindex{2006}{Competenza!di base in campo scientifico e tecnologico}.}}
\newglossaryentry{compitidirealt}{name={Compiti di realtà},description={I compiti di realtà si identificano nella richiesta rivolta allo studente di risolvere una situazione problematica, complessa e nuova, quanto più possibile vicina al mondo reale, utilizzando conoscenze e abilità già acquisite e trasferendo procedure e condotte cognitive in contesti e ambiti di riferimento moderatamente diversi da quelli resi familiari dalla pratica didattica, pur non escludendo prove che chiamino in causa una sola disciplina, si ritiene opportuno privilegiare prove per la cui risoluzione l'alunno debba richiamare in forma integrata, componendoli autonomamente, più apprendimenti acquisiti~\cite{LG2018}.\mioindex{2018}{Compiti!di realtà}}}
\newglossaryentry{comunicazioneelettronica}{name={Comunicazione elettronica},description={Ogni informazione scambiata o trasmessa tra un numero finito di soggetti tramite un servizio di comunicazione elettronica accessibile al pubblico. Sono escluse le informazioni trasmesse al pubblico tramite una rete di comunicazione elettronica, come parte di un servizio di radiodiffusione, salvo che le stesse informazioni siano collegate ad un abbonato o utente ricevente, identificato o identificabile.\mioindex{2003}{Comunicazione!elettronica}~\cite{DL2003}}}
\newglossaryentry{comunicazione}{name={Comunicazione},description={Il dare conoscenza dei dati personali a uno o più soggetti determinati diversi dall'interessato, dal rappresentante del titolare nel territorio dello Stato, dal responsabile e dagli incaricati, in qualunque forma, anche mediante la loro messa a disposizione o consultazione.\mioindex{2003}{Comunicazione}~\cite{DL2003}}}
\newglossaryentry{conoscenze}{name={Conoscenze},description={\begin{tabular}{cp{12cm}}\toprule 2005 & Le conoscenze rappresentano il sapere che costituisce il patrimonio di una cultura; sono un insieme di informazioni, nozioni, dati, principi, regole di comportamento, teorie, concetti codificati e conservati perché ritenuti degni di essere trasmessi alle nuove generazioni. Le conoscenze sono ordinate, nelle Indicazioni nazionali, per “discipline” e per “Educazione alla Convivenza civile” e costituiscono, unitamente alle abilità, gli “obiettivi specifici di apprendimento”~\cite{CIRC2005a}\mioindex{2005}{Conoscenze}. \\\midrule 2008&Risultato dell'assimilazione di informazioni attraverso l'apprendimento. Le conoscenze sono l'insieme di fatti, principi, teorie e pratiche che riguardano un ambito di lavoro o di studio, Nel contesto dell'EQF, le conoscenze sono descritte come teoriche e/o pratiche~\cite{RA2017}~\cite{RA2008}\mioindex{2017}{Conoscenze}\mioindex{2008}{Conoscenze}. \\\bottomrule\end{tabular}}}
\newglossaryentry{consensointeressato}{name={Consenso dell'interessato},description={Qualsiasi manifestazione di volontà libera, specifica, informata e inequivocabile dell'interessato, con la quale lo stesso manifesta il proprio assenso, mediante dichiarazione o azione positiva inequivocabile, che i dati personali che lo riguardano siano oggetto di trattamento\mioindex{2016}{Consenso!dell'interessato}~\cite{RE2016}~\cite{DL2018d}.}}
\newglossaryentry{consigliosuperiorepubblicaistruzione}{name={Consiglio Superiore della Pubblica Istruzione},description={Il Consiglio superiore della pubblica istruzione è organo di garanzia dell'unitarietà del sistema nazionale dell'istruzione, Ha compiti di supporto tecnico-scientifico per l'esercizio delle funzioni di governo nelle materie di “istruzione universitaria, ordinamenti scolastici, programmi scolastici, organizzazione generale dell'istruzione scolastica e stato giuridico del personale” (articolo 1, comma 3, lettera q), della legge 59 del 15 marzo 1997)~\cite{LEGGE1997}\mioindex{1997}{Consiglio!Superiore della Pubblica Istruzione}.}}
\newglossaryentry{convalidan}{name={Convalida},description={Dati utilizzati per convalidare una firma elettronica o un sigillo elettronico.\mioindex{2014}{Dati!di convalida}~\cite{RE2014}}}
\newglossaryentry{convalida}{name={Convalida},description={\begin{tabular}{cp{12cm}}\toprule 2012&Il processo mediante il quale un'autorità o un organismo competente conferma che un individuo ha acquisito, anche in un contesto di apprendimento non formale e informale, risultati dell'apprendimento misurati in relazione a uno standard appropriato e che si articola in quattro fasi distinte, vale a dire individuazione, documentazione, valutazione e certificazione dei risultati della valutazione sotto forma di qualifica piena, crediti o qualifica parziale, ove opportuno e in funzione delle circostanze nazionali~\cite{DE2018}~\cite{RA2012}\mioindex{2018}{Convalida}\mioindex{2012}{Convalida}.\\\midrule 2017&Processo in base al quale un'autorità competente conferma l'acquisizione, in un contesto di apprendimento non formale e informale, di risultati dell'apprendimento misurati in relazione a uno standard appropriato; si articola nelle seguenti quattro fasi distinte: individuazione, mediante un colloquio, delle esperienze specifiche dell'interessato; documentazione per rendere visibili le esperienze dell'interessato; valutazione formale di tali esperienze e certificazione dei risultati della valutazione, che può portare a una qualifica parziale o completa~\cite{RA2017}\mioindex{2017}{Convalida}. \\\bottomrule\end{tabular}}}
\newglossaryentry{cooperazioneapplicativa}{name={Cooperazione applicativa},description={La parte del Sistema Pubblico di Connettività finalizzata all'interazione tra i sistemi informatici dei soggetti partecipanti, per garantire l'integrazione dei metadati, delle informazioni, dei processi e procedimenti amministrativi.\mioindex{2016}{Cooperazione!applicativa}~\cite{DL2016a}}}
\newglossaryentry{cortegiustiziaunioneeuropea}{name={Corte di giustizia dell'Unione europea},description={\`{E} l’autorità giudiziaria dell’Unione europea e vigila, in collaborazione con gli organi giurisdizionali degli Stati membri, sull’applicazione e interpretazione uniforme del diritto dell’Unione.}}
\newglossaryentry{cp2011}{name={CP2011},description={La classificazione CP2011 fornisce uno strumento per ricondurre tutte le professioni esistenti nel mercato del lavoro all'interno di un numero limitato di raggruppamenti professionali, da utilizzare per comunicare, diffondere e scambiare dati statistici e amministrativi sulle professioni, comparabili a livello internazionale; non deve invece essere inteso come strumento di regolamentazione delle professioni.~\cite{ISTAT2020}\mioindex{2011}{Classificazione!CP2011}.}}
\newglossaryentry{creatoresigillo}{name={Creatore di un sigillo},description={Una persona giuridica che crea un sigillo elettronico.\mioindex{2014}{Creatore!di un sigillo}~\cite{RE2014}},see={sigilloelettronico}}
\newglossaryentry{credenzialiautenticazione}{name={Credenziali di autenticazione},description={I dati ed i dispositivi, in possesso di una persona, da questa conosciuti o ad essa univocamente correlati, utilizzati per l'autenticazione informatica.\mioindex{2003}{Credenziali!di autenticazione}~\cite{DL2003}}}
\newglossaryentry{crediti}{name={Crediti},description={Unità che confermano che una parte della qualifica, costituita da un insieme coerente di risultati dell'apprendimento, è stata valutata e convalidata da un'autorità competente, secondo una norma concordata; i crediti sono concessi da autorità competenti quando il soggetto ha conseguito i risultati dell'apprendimento definiti, comprovati da opportune valutazioni, e possono essere espressi con un valore quantitativo (ad esempio crediti o unità di credito), che indica il carico di lavoro ritenuto solitamente necessario affinché una persona consegua i risultati dell'apprendimento corrispondenti~\cite{RA2017}\mioindex{2017}{Credito}.}}
\newglossaryentry{creditoformativo}{name={Credito formativo},description={Per credito formativo si intende il valore attribuibile alle competenze, abilità e conoscenze acquisite nel percorso di apprendimento, certificate, validate e comunque riconoscibili ai fini dell'inserimento nel percorso di IP o di IeFP per il quale è stata presentata domanda di passaggio, anche in seguito ad eventuali verifiche in ingresso~\cite{LG2019a}\mioindex{2019}{Credito!formativo}.}}
\newglossaryentry{cyberbullismo}{name={Cyberbullismo},description={Si intende qualunque forma di pressione, aggressione, molestia, ricatto, ingiuria, denigrazione, diffamazione, furto d'identità, alterazione, acquisizione illecita, manipolazione, trattamento illecito di dati personali in danno di minorenni, realizzata per via telematica, nonchè la diffusione di contenuti on line aventi ad oggetto anche uno o più componenti della famiglia del minore il cui scopo intenzionale e predominante sia quello di isolare un minore o un gruppo di minori ponendo in atto un serio abuso, un attacco dannoso, o la loro messa in ridicolo\mioindex{2017}{Cyberbullismo}~\cite{LEGGE2017}.}}
\newglossaryentry{datibiometrici}{name={Dati biometrici},description={I dati personali ottenuti da un trattamento tecnico specifico relativi alle caratteristiche fisiche, fisiologiche o comportamentali di una persona fisica che ne consentono o confermano l'identificazione univoca, quali l'immagine facciale o i dati dattiloscopici\mioindex{2016}{Dato!biometrico}~\cite{RE2016}~\cite{DL2018d}.}}
\newglossaryentry{daticonvalida}{name={Dati di convalida},description={Dati utilizzati per convalidare una firma elettronica o un sigillo elettronico.\mioindex{2014}{Dati!di convalida}~\cite{RE2014}}}
\newglossaryentry{daticreazionefirmaelettronica}{name={Dati per la creazione di una firma elettronica},description={I dati unici utilizzati dal firmatario per creare una firma elettronica.\mioindex{2014}{Dati!per la creazione di una firma elettronica}~\cite{RE2014}}}
\newglossaryentry{daticreazionesigilloelettronico}{name={Dati per la creazione di un sigillo elettronico},description={I dati unici utilizzati dal creatore del sigillo elettronico per creare un sigillo elettronico.\mioindex{2014}{Dati!per la creazione di un sigillo elettronico}~\cite{RE2014}},see={sigilloelettronico}}
\newglossaryentry{datigenetici}{name={Dati genetici},description={I dati personali relativi alle caratteristiche genetiche ereditarie o acquisite di una persona fisica che forniscono informazioni univoche sulla fisiologia o sulla salute di detta persona fisica, e che risultano in particolare dall'analisi di un campione biologico della persona fisica in questione\mioindex{2016}{Dato!genetico}~\cite{RE2016}~\cite{DL2018d}.}}
\newglossaryentry{datigiudiziari}{name={Dati giudiziari},description={I dati personali idonei a rivelare provvedimenti di cui all'articolo 3, comma 1, lettere da a) a o) e da r) a u), del d.P.R. 14 novembre 2002, n. 313, in materia di casellario giudiziale, di anagrafe delle sanzioni amministrative dipendenti da reato e dei relativi carichi pendenti, o la qualità di imputato o di indagato ai sensi degli articoli 60 e 61 del codice di procedura penale.\mioindex{2003}{Dati!giudiziari}~\cite{DL2003}}}
\newglossaryentry{datiidentificativi}{name={Dati identificativi},description={I dati personali che permettono l'identificazione diretta dell'interessato.\mioindex{2003}{Dati!identificativi}~\cite{DL2003}}}
\newglossaryentry{datiidentificazionepersonale}{name={Dati di identificazione personale},description={Un’unità materiale e/o immateriale contenente dati di identificazione personale e utilizzata per l’autenticazione per un servizio online.\mioindex{2014}{Dati!di identificazione personale}~\cite{RE2014}}}
\newglossaryentry{datipersonali}{name={Dato personale},description={Informazione riguardante una persona fisica identificata o identificabile~\cite{DE2018}\mioindex{2018}{Dato!personale}.}}
\newglossaryentry{datirelativisalute}{name={Dati relativi alla salute},description={I dati personali attinenti alla salute fisica o mentale di una persona fisica, compresa la prestazione di servizi di assistenza sanitaria, che rivelano informazioni relative al suo stato di salute\mioindex{2016}{Dato!relativo alla salute}~\cite{RE2016}~\cite{DL2018d}.}}
\newglossaryentry{datirelativitraffico}{name={Dati relativi al traffico},description={Qualsiasi dato sottoposto a trattamento ai fini della trasmissione di una comunicazione su una rete di comunicazione elettronica o della relativa fatturazione.\mioindex{2003}{Dati!relativi al traffico}~\cite{DL2003}}}
\newglossaryentry{datirelativiubicazione}{name={Dati relativi all'ubicazione},description={Ogni dato trattato in una rete di comunicazione elettronica che indica la posizione geografica dell'apparecchiatura terminale dell'utente di un servizio di comunicazione elettronica accessibile al pubblico.\mioindex{2003}{Dati!relativi all'ubicazione}~\cite{DL2003}}}
\newglossaryentry{datisensibili}{name={Dati sensibili},description={I dati personali idonei a rivelare l'origine razziale ed etnica, le convinzioni religiose, filosofiche o di altro genere, le opinioni politiche, l'adesione a partiti, sindacati, associazioni od organizzazioni a carattere religioso, filosofico, politico o sindacale, nonchè i dati personali idonei a rivelare lo stato di salute e la vita sessuale.\mioindex{2003}{Dati!sensibili}~\cite{DL2003}}}
\newglossaryentry{datiterritoriali}{name={Dati territoriali},description={I dati che attengono, direttamente o indirettamente, a una località o a un'area geografica specifica.\mioindex{2016}{Dati!territoriali}~\cite{DL2016a}}}
\newglossaryentry{datitipoaperto}{name={Dati di tipo aperto},description={I dati che presentano le seguenti caratteristiche: 1) sono disponibili secondo i termini di una licenza o di una previsione normativa che ne permetta l'utilizzo da parte di chiunque, anche per finalità commerciali, in formato disaggregato; 2) sono accessibili attraverso le tecnologie dell'informazione e della comunicazione, ivi comprese le reti telematiche pubbliche e private, in formati aperti ai sensi della lettera l-bis), sono adatti all'utilizzo automatico da parte di programmi per elaboratori e sono provvisti dei relativi metadati; 3) sono resi disponibili gratuitamente attraverso le tecnologie dell'informazione e della comunicazione, ivi comprese le reti telematiche pubbliche e private, oppure sono resi disponibili ai costi marginali sostenuti per la loro riproduzione e divulgazione salvo quanto previsto dall'articolo 7 del decreto legislativo 24 gennaio 2006, n. 36.\mioindex{2017}{Dati!di tipo aperto}~\cite{DL2017d}}}
\newglossaryentry{datoanonimo}{name={Dato anonimo},description={Il dato che in origine, o a seguito di trattamento, non può essere associato ad un interessato identificato o identificabile.\mioindex{2003}{Dato!anonimo}~\cite{DL2003}}}
\newglossaryentry{datoconoscibilitalimitata}{name={Dato a conoscibilità limitata},description={Il dato la cui conoscibilità è riservata per legge o regolamento a specifici soggetti o categorie di soggetti.\mioindex{2005}{Dato!a conoscibilità limitata}~\cite{DL2005c}}}
\newglossaryentry{datopersonalen}{name={Dato personale},description={Qualunque informazione relativa a persona fisica, persona giuridica, ente od associazione, identificati o identificabili, anche indirettamente, mediante riferimento a qualsiasi altra informazione, ivi compreso un numero di identificazione personale.\mioindex{2003}{Dato!personale}~\cite{DL2003}}}
\newglossaryentry{datopersonale}{name={Dato personale},description={Qualsiasi informazione riguardante una persona fisica identificata o identificabile («interessato»); si considera identificabile la persona fisica che può essere identificata, direttamente o indirettamente, con particolare riferimento a un identificativo come il nome, un numero di identificazione, dati relativi all'ubicazione, un identificativo online o a uno o più elementi caratteristici della sua identità fisica, fisiologica, genetica, psichica, economica, culturale o sociale\mioindex{2018}{Dato!personale}\mioindex{2016}{Dato!personale}~\cite{RE2016}~\cite{DL2018d}.}}
\newglossaryentry{datopubblicheamministrazioni}{name={Dato delle pubbliche amministrazioni},description={Il dato formato, o comunque trattato da una pubblica amministrazione.\mioindex{2005}{Dato!delle pubbliche amministrazioni}~\cite{DL2005c}}}
\newglossaryentry{datopubblico}{name={Dato pubblico},description={Il dato conoscibile da chiunque.\mioindex{2005}{Dato!pubblico}~\cite{DL2005c}}}
\newglossaryentry{datorelavoro}{name={Datore di lavoro},description={Il soggetto titolare del rapporto di lavoro con il lavoratore o, comunque, il soggetto che, secondo il tipo e l'assetto dell'organizzazione nel cui ambito il lavoratore presta la propria attività, ha la responsabilità dell'organizzazione stessa o dell'unità produttiva in quanto esercita i poteri decisionali e di spesa. Nelle pubbliche amministrazioni di cui all'articolo 1, comma 2, del decreto legislativo 30 marzo 2001, n. 165, per datore di lavoro si intende il dirigente al quale spettano i poteri di gestione, ovvero il funzionario non avente qualifica dirigenziale, nei soli casi in cui quest'ultimo sia preposto ad un ufficio avente autonomia gestionale, individuato dall'organo di vertice delle singole amministrazioni tenendo conto dell'ubicazione e dell'ambito funzionale degli uffici nei quali viene svolta l'attività, e dotato di autonomi poteri decisionali e di spesa. In caso di omessa individuazione, o di individuazione non conforme ai criteri sopra indicati, il datore di lavoro coincide con l'organo di vertice medesimo\mioindex{2008}{Datore!di lavoro}~\cite{DL2008a}.}}
\newglossaryentry{destinatario}{name={Destinatario},description={La persona fisica o giuridica, l'autorità pubblica, il servizio o un altro organismo che riceve comunicazione di dati personali, che si tratti o meno di terzi. Tuttavia, le autorità pubbliche che possono ricevere comunicazione di dati personali nell'ambito di una specifica indagine conformemente al diritto dell'Unione o degli Stati membri non sono considerate destinatari; il trattamento di tali dati da parte di dette autorità pubbliche è conforme alle norme applicabili in materia di protezione dei dati secondo le finalità del trattamento\mioindex{2016}{Destinatario}~\cite{RE2016}~\cite{DL2018d}.}}
\newglossaryentry{diagnosi}{name={Diagnosi},description={In medicina, giudizio clinico che consiste nel riconoscere una condizione morbosa in base all’esame clinico del malato, e alle ricerche di laboratorio e strumentali.~\cite{TRECCANI2020}}}
\newglossaryentry{diagrammaGantt}{name={Diagramma di Gantt},description={Il diagramma di Gantt è uno strumento di supporto alla gestione dei progetti, così chiamato in ricordo dell'ingegnere statunitense Henry Lawrence Gantt (1861-1919), che si occupava di scienze sociali e che lo ideò nel 1917~\cite{Wikipedia2020a}.}}
\newglossaryentry{didatticaindividualizzata}{name={Didattica individualizzata},description={Consiste nelle attività di recupero individuale che può svolgere l'alunno per potenziare determinate abilità o per acquisire specifiche competenze, anche nell'ambito delle strategie compensative e del metodo di studio; tali attività individualizzate possono essere realizzate nelle fasi di lavoro individuale in classe o in momenti ad esse dedicati, secondo tutte le forme di flessibilità del lavoro scolastico consentite dalla normativa vigente.~\cite{LG2011}\mioindex{2011}{Didattica!individualizzata}.}}
\newglossaryentry{didatticapersonalizzata}{name={Didattica personalizzata},description={Sulla base di quanto indicato nella Legge 53/2003 e nel Decreto legislativo 59/2004, calibra l'offerta didattica, e le modalità relazionali, sulla specificità ed unicità a livello personale dei bisogni educativi che caratterizzano gli alunni della classe, considerando le differenze individuali soprattutto sotto il profilo qualitativo; si può favorire, così, l'accrescimento dei punti di forza di ciascun alunno, lo sviluppo consapevole delle sue ‘preferenzè e del suo talento. Nel rispetto degli obiettivi generali e specifici di apprendimento, la didattica personalizzata si sostanzia attraverso l'impiego di una varietà di metodologie e strategie didattiche, tali da promuovere le potenzialità e il successo formativo in ogni alunno: l'uso dei mediatori didattici (schemi, mappe concettuali, etc.), l'attenzione agli stili di apprendimento, la calibrazione degli interventi sulla base dei livelli raggiunti, nell'ottica di promuovere un apprendimento significativo.~\cite{LG2011}\mioindex{2011}{Didattica!personalizzata}.}}
\newglossaryentry{diffusione}{name={Diffusione},description={Il dare conoscenza dei dati personali a soggetti indeterminati, in qualunque forma, anche mediante la loro messa a disposizione o consultazione.\mioindex{2003}{Diffusione}~\cite{DL2003}}}
\newglossaryentry{dimensioneeuropeaorientamento}{name={Dimensione europea dell'orientamento},description={La cooperazione e il sostegno a livello di Unione volti a rafforzare politiche, sistemi e pratiche di orientamento all'interno dell'Unione~\cite{DE2018}\mioindex{2018}{Dimensione!europea dell'orientamento}.}}
\newglossaryentry{dirigenteazienda}{name={Dirigente d'azienda},description={Qualsiasi persona che abbia svolto in un'impresa del settore professionale corrispondente:i) la funzione di direttore d'azienda o di filiale, o ii) la funzione di institore o vice direttore d'azienda, se tale funzione implica una responsabilità corrispondente a quella dell'imprenditore o del direttore d'azienda rappresentato, o iii) la funzione di dirigente con mansioni commerciali e/o tecniche e responsabile di uno o più reparti dell'azienda~\cite{DI2005}\mioindex{2018}{Dirigente!d'azienda}.}}
\newglossaryentry{dirigente}{name={Dirigente},description={Persona che, in ragione delle competenze professionali e di poteri gerarchici e funzionali adeguati alla natura dell'incarico conferitogli, attua le direttive del datore di lavoro organizzando l'attività lavorativa e vigilando su di essa\mioindex{2008}{Dirigente}~\cite{DL2008a}.}}
\newglossaryentry{disagio}{name={Disagio},description={Senso di pena e di molestia provato per l’incapacità di adattarsi a un ambiente, a una situazione, anche per motivi morali, o più genericam. senso d’imbarazzo.~\cite{TRECCANI2020b}}}
\newglossaryentry{discalculia}{name={Discalculia},description={Disturbo specifico che si manifesta con una difficoltà negli automatismi del calcolo e dell'elaborazione dei numeri.\mioindex{2010}{Discalculia}~\cite{LEGGE2010a}}}
\newglossaryentry{disgrafia}{name={Disgrafia},description={Un disturbo specifico di scrittura che si manifesta in difficoltà nella realizzazione grafica.\mioindex{2010}{Disgrafia}~\cite{LEGGE2010a}}}
\newglossaryentry{dislessia}{name={Dislessia},description={Un disturbo specifico che si manifesta con una difficoltà nell'imparare a leggere, in particolare nella decifrazione dei segni linguistici, ovvero nella correttezza e nella rapidità della lettura.\mioindex{2010}{Dislessia}~\cite{LEGGE2010a}}}
\newglossaryentry{disordine}{name={Disordine},description={Mancanza o turbamento dell’ordine, e lo stato delle cose disordinate; confusione. Disservizio, irregolare funzionamento.~\cite{TRECCANI2020d}}}
\newglossaryentry{disortografia}{name={Disortografia},description={Un disturbo specifico di scrittura che si manifesta in difficoltà nei processi linguistici di transcodifica.\mioindex{2010}{Disortografia}~\cite{LEGGE2010a}}}
\newglossaryentry{disponibilita}{name={Disponibilità},description={La possibilità di accedere ai dati senza restrizioni non riconducibili a esplicite norme di legge.\mioindex{2005}{Disponibilità}~\cite{DL2005c}}}
\newglossaryentry{dispositivocreazionefirmaelettronicaqualificata}{name={Dispositivo per la creazione di una firma elettronica qualificata},description={Un dispositivo per la creazione di una firma elettronica che soddisfa i requisiti di cui all’allegato II. Vedi~\cref{sec:allegatoIIreg9102014}\mioindex{2014}{Dispositivo!per la creazione di una firma elettronica qualificata}~\cite{RE2014}},see={firmaelettronican}}
\newglossaryentry{dispositivocreazionefirmaelettronica}{name={Dispositivo per la creazione di una firma elettronica},description={Un software o hardware configurato utilizzato per creare una firma elettronica.\mioindex{2014}{Dispositivo!per la creazione di una firma elettronica}~\cite{RE2014}},see={firmaelettronican}}
\newglossaryentry{dispositivocreazionesigilloelettronicoqualificato}{name={Dispositivo per la creazione di un sigillo elettronico qualificato},description={Un dispositivo per la creazione di un sigillo elettronico che soddisfa mutatis mutandis i requisiti di cui all’allegato II.\mioindex{2014}{Dispositivo!per la creazione di un sigillo elettronico qualificato}~\cite{RE2014}},see={sigilloelettronico}}
\newglossaryentry{dispositivocreazionesigilloelettronico}{name={Dispositivo per la creazione di un sigillo elettronico},description={Un software o hardware configurato utilizzato per creare un sigillo elettronico.\mioindex{2014}{Dispositivo!per la creazione di un sigillo elettronico}~\cite{RE2014}},see={sigilloelettronico}}
\newglossaryentry{distrattore}{name={Distrattore},description={Risposta errata in una domanda a risposta multipla~\cite{Desimoni2017}.\mioindex{2017}{Distrattore}}}
\newglossaryentry{disturbo}{name={Disturbo},description={Leggera irregolarità o disordine nelle funzioni organiche, perturbazione del normale andamento di un fenomeno, del regolare funzionamento di un dispositivo o di una macchina, e simili.~\cite{TRECCANI2020c}}}
\newglossaryentry{documentoelettronico}{name={Documento elettronico},description={Qualsiasi contenuto conservato in forma elettronica, in particolare testo o registrazione sonora, visiva o audiovisiva.\mioindex{2014}{Documento!elettronico}~\cite{RE2014}}}
\newglossaryentry{documentoinformaticon}{name={Documento informatico},description={Il documento elettronico che contiene la rappresentazione informatica di atti, fatti o dati giuridicamente rilevanti.\mioindex{2016}{Documento!informatico}~\cite{DL2016a}}}
\newglossaryentry{documentoinformatico}{name={Documento informatico},description={La rappresentazione informatica di atti, fatti o dati giuridicamente rilevanti.\mioindex{2005}{Documento!informatico}~\cite{DL2005c}}}
\newglossaryentry{domiciliodigitalen}{name={Domicilio digitale},description={Un indirizzo elettronico eletto presso un servizio di posta elettronica certificata o un servizio elettronico di recapito certificato qualificato, come definito dal regolamento (UE) 23 luglio 2014 n. 910 del Parlamento europeo e del Consiglio in materia di identificazione elettronica e servizi fiduciari per le transazioni elettroniche nel mercato interno e che abroga la direttiva 1999/93/CE, di seguito "Regolamento eIDAS",valido ai fini delle comunicazioni elettroniche aventi valore legale.\mioindex{2017}{Domicilio!digitale}~\cite{DL2017d}}}
\newglossaryentry{domiciliodigitale}{name={Domicilio digitale},description={L'indirizzo di posta elettronica certificata o altro servizio elettronico di recapito certificato qualificato di cui al Regolamento (UE) 23 luglio 2014 n. 910 del Parlamento europeo e del Consiglio in materia di identificazione elettronica e servizi fiduciari per le transazioni elettroniche nel mercato interno e che abroga la direttiva 1999/93/CE, di seguito «Regolamento eIDAS», che consenta la prova del momento di ricezione di una comunicazione tra i soggetti di cui all'articolo 2, comma 2, e i soggetti giuridici, che sia basato su standard o norme riconosciute nell'ambito dell'unione europea.\mioindex{2016}{Domicilio!digitale}~\cite{DL2016a}}}
\newglossaryentry{dpog}{name={Data Protection Officer},description={Il DPO è un supervisore indipendente, il quale sarà designato obbligatoriamente, da soggetti apicali di tutte le pubbliche amministrazioni e nello specifico è previsto l'obbligo nel caso in cui “il trattamento è effettuato da un'autorità pubblica o da un organismo pubblico, eccettuate le autorità giurisdizionali quando esercitano le loro funzioni giurisdizionali”}}
\newglossaryentry{edpbg}{name={Comitato europeo per la protezione dei dati},description={Il comitato europeo per la protezione dei dati è un organo europeo indipendente, che contribuisce all'applicazione coerente delle norme sulla protezione dei dati in tutta l'Unione europea e promuove la cooperazione tra le autorità competenti per la protezione dei dati dell'UE.}}
\newglossaryentry{eidasg}{name={Regolamento eIDAS},description={Regolamento UE n° 910/2014 sull'identità digitale - ha l'obiettivo di fornire una base normativa a livello comunitario per i servizi fiduciari e i mezzi di identificazione elettronica degli stati membri~\cite{RE2014}.}}
\newglossaryentry{entepubblicotitolareg}{name={Ente pubblico titolare},description={Amministrazione pubblica, centrale, regionale e delle province autonome titolare, a norma di legge, della regolamentazione di servizi di individuazione e validazione e certificazione delle competenze, Nello specifico sono da intendersi enti pubblici titolari: 1) il Ministero dell'istruzione, dell'università e della ricerca, in materia di individuazione e validazione e certificazione delle competenze riferite ai titoli di studio del sistema scolastico e universitario; 2) le regioni e le province autonome di Trento e Bolzano, in materia di individuazione e validazione e certificazione di competenze riferite a qualificazioni rilasciate nell'ambito delle rispettive competenze; 3) il Ministero del lavoro e delle politiche sociali, in materia di individuazione e validazione e certificazione di competenze riferite a qualificazioni delle professioni non organizzate in ordini o collegi, salvo quelle comunque afferenti alle autorità competenti di cui al successivo punto 4; 4) il Ministero dello sviluppo economico e le altre autorità competenti ai sensi dell'articolo 5 del decreto legislativo 9 novembre 2007, n. 206, in materia di individuazione e validazione e certificazione di competenze riferite a qualificazioni delle professioni regolamentate a norma del medesimo decreto~\cite{DL2013}\mioindex{2013}{Ente!pubblico titolare}.}}
\newglossaryentry{entetitolato}{name={Ente titolato},description={Soggetto, pubblico o privato, ivi comprese le camere di commercio, industria, artigianato e agricoltura, autorizzato o accreditato dall'ente pubblico titolare, ovvero deputato a norma di legge statale o regionale, ivi comprese le istituzioni scolastiche, le università e le istituzioni dell'alta formazione artistica, musicale e coreutica, a erogare in tutto o in parte servizi di individuazione e validazione e certificazione delle competenze, in relazione agli ambiti di titolarità enti pubblico titolare~\cite{DL2013}\mioindex{2013}{Ente!titolato}.}}
\newglossaryentry{esabac}{name={EsaBac},description={Accordo fra lo stato italiano e lo stato francese che permette di conseguire con un unico esame l'Esame di Stato e il Baccalauréat francese\mioindex{2009}{EsaBac}~\cite{ACCO2009}.}}
\newglossaryentry{esperienzaprofessionale}{name={Esperienza professionale},description={L'esercizio effettivo e legittimo della professione in questione in uno Stato membro, a tempo pieno o a tempo parziale per un periodo equivalente~\cite{DI2013}\mioindex{2013}{Esperienza!professionale}.}}
\newglossaryentry{firmadigitale}{name={Firma digitale},description={Un particolare tipo di firma elettronica qualificata basata su un sistema di chiavi crittografiche, una pubblica e una privata, correlate tra loro, che consente al titolare tramite la chiave privata e al destinatario tramite la chiave pubblica, rispettivamente, di rendere manifesta e di verificare la provenienza e l'integrità di un documento informatico o di un insieme di documenti informatici.\mioindex{2005}{Firma!digitale}~\cite{DL2005c}}}
\newglossaryentry{firmaelettronicaavanzata}{name={Firma elettronica avanzata},description={Una firma elettronica che soddisfi i requisiti di cui all’articolo 26. Vedi~\cref{sec:articolo26reg9102014}\mioindex{2014}{Firma!elettronica avanzata}~\cite{RE2014}}}
\newglossaryentry{firmaelettronican}{name={Firma elettronica},description={Dati in forma elettronica, acclusi oppure connessi tramite associazione logica ad altri dati elettronici e utilizzati dal firmatario per firmare.\mioindex{2014}{Firma!elettronica}~\cite{RE2014}}}
\newglossaryentry{firmaelettronicaqualificatan}{name={Firma elettronica qualificata},description={Una firma elettronica avanzata creata da un dispositivo per la creazione di una firma elettronica qualificata e basata su un certificato qualificato per firme elettroniche.\mioindex{2014}{Firma!elettronica qualificata}~\cite{RE2014}},see={daticreazionefirmaelettronica} }
\newglossaryentry{firmaelettronicaqualificata}{name={Firma elettronica qualificata},description={La firma elettronica ottenuta attraverso una procedura informatica che garantisce la connessione univoca al firmatario e la sua univoca autenticazione informatica, creata con mezzi sui quali il firmatario può conservare un controllo esclusivo e collegata ai dati ai quali si riferisce in modo da consentire di rilevare se i dati stessi siano stati successivamente modificati, che sia basata su un certificato qualificato e realizzata mediante un dispositivo sicuro per la creazione della firma, quale l'apparato strumentale usato per la creazione della firma elettronica.\mioindex{2005}{Firma!elettronica qualificata}~\cite{DL2005c}}}
\newglossaryentry{firmaelettronica}{name={Firma elettronica},description={L'insieme dei dati in forma elettronica, allegati oppure connessi tramite associazione logica ad altri dati elettronici, utilizzati come metodo di autenticazione informatica.\mioindex{2005}{Firma!elettronica}~\cite{DL2005c}}}
\newglossaryentry{firmatario}{name={Firmatario},description={Una persona fisica che crea una firma elettronica.\mioindex{2014}{Firmatario}~\cite{RE2014}}}
\newglossaryentry{formatoaperto}{name={Formato aperto},description={Un formato di dati reso pubblico, documentato esaustivamente e neutro rispetto agli strumenti tecnologici necessari per la fruizione dei dati stessi.\mioindex{2017}{Formato!aperto}~\cite{DL2017d}}}
\newglossaryentry{formazioneregolamentata}{name={Formazione regolamentata},description={Qualsiasi formazione specificamente orientata all'esercizio di una professione determinata e consistente in un ciclo di studi completato, eventualmente, da una formazione professionale, un tirocinio professionale o una pratica professionale. La struttura e il livello della formazione professionale, del tirocinio professionale o della pratica professionale sono stabiliti dalle disposizioni legislative, regolamentari o amministrative dello Stato membro in questione e sono soggetti a controllo o autorizzazione dell'autorità designata a tal fine~\cite{DI2005}\mioindex{2005}{Formazione!regolamentata}.}}
\newglossaryentry{formazione}{name={Formazione},description={Processo educativo attraverso il quale trasferire ai lavoratori ed agli altri soggetti del sistema di prevenzione e protezione aziendale conoscenze e procedure utili alla acquisizione di competenze per lo svolgimento in sicurezza dei rispettivi compiti in azienda e alla identificazione, alla riduzione e alla gestione dei rischi\mioindex{2008}{Formazione}~\cite{DL2008a}.}}
\newglossaryentry{fruibilitdato}{name={Fruibilità di un dato},description={La possibilità di utilizzare il dato anche trasferendolo nei sistemi informativi automatizzati di un'altra amministrazione.\mioindex{2005}{Fruibilità!di un dato}~\cite{DL2005c}}}
\newglossaryentry{garante}{name={Garante},description={L'autorità di cui all'articolo 153, istituita dalla legge 31 dicembre 1996, n. 675.\mioindex{2003}{Garante}~\cite{DL2003}}}
\newglossaryentry{gdprg}{name={Regolamento GDPR},description={Regolamento Ue 2016/679, noto come GDPR (General Data Protection Regulation) – relativo alla protezione delle persone fisiche con riguardo al trattamento e alla libera circolazione dei dati personali~\cite{RE2016}~\cite{DL2018d}.}}
\newglossaryentry{gestioneinformaticadocumenti}{name={Gestione informatica dei documenti},description={L'insieme delle attività finalizzate alla registrazione e segnatura di protocollo, nonchè alla classificazione, organizzazione, assegnazione, reperimento e conservazione dei documenti amministrativi formati o acquisiti dalle amministrazioni, nell'ambito del sistema di classificazione d'archivio adottato, effettuate mediante sistemi informatici.\mioindex{2005}{Gestione informatica!dei documenti}~\cite{DL2005c}}}
\newglossaryentry{gruppoimprenditoriale}{name={Gruppo imprenditoriale},description={Un gruppo costituito da un'impresa controllante e dalle imprese da questa controllate\mioindex{2016}{Gruppo!imprenditoriale}~\cite{RE2016}~\cite{DL2018d}.}}
\newglossaryentry{icfg}{name={ICF},description={La classificazione ICF fornisce un linguaggio standard e unificato che serve da modello di riferimento per la descrizione della salute e degli stati ad essa correlati.\mioindex{2001}{ICF}.~\cite{OMS2001}}}
\newglossaryentry{idc10g}{name={IDC-10},description={L'ICD è uno standard di classificazione per gli studi statistici ed epidemiologici, nonché valido strumento di gestione di salute e igiene pubblica. È oggi alla decima edizione (ICD-10), approvata nel 1990 durante la 43ª Assemblea mondiale della sanità dell'OMS e utilizzata a partire dal 1994\mioindex{1994}{ICD-10}.~\cite{Wikipedia2019d}}}
\newglossaryentry{identificazioneelettronica}{name={Identificazione elettronica},description={Il processo per cui si fa uso di dati di identificazione personale in forma elettronica che rappresentano un’unica persona fisica o giuridica, o un’unica persona fisica che rappresenta una persona giuridica.\mioindex{2014}{Identificazione!elettronica}~\cite{RE2014}}}
\newglossaryentry{impresa}{name={Impresa},description={La persona fisica o giuridica, indipendentemente dalla forma giuridica rivestita, che eserciti un'attività economica, comprendente le società di persone o le associazioni che esercitano regolarmente un'attività economica\mioindex{2016}{Impresa}~\cite{RE2016}~\cite{DL2018d}.}}
\newglossaryentry{incaricati}{name={Incaricati},description={Le persone fisiche autorizzate a compiere operazioni di trattamento dal titolare o dal responsabile.\mioindex{2003}{Incaricati}~\cite{DL2003}}}
\newglossaryentry{indentitadigitale}{name={Identità digitale},description={La rappresentazione informatica della corrispondenza tra un utente e i suoi attributi identificativi, verificata attraverso l'insieme dei dati raccolti e registrati in forma digitale secondo le modalità fissate nel decreto attuativo dell'articolo 64.\mioindex{2016}{Identità!digitale}~\cite{DL2016a}}}
\newglossaryentry{indirizzistudio}{name={Indirizzi di studio},description={Gli indirizzi di studio sono strutturati in: a) attività e insegnamenti di istruzione generale, comuni a tutti gli indirizzi, riferiti all'asse culturale dei linguaggi, all'asse matematico e all'asse storico sociale; b) attività e insegnamenti di indirizzo riferiti all'asse scientifico, tecnologico e professionale e, nel caso di presenza di una seconda lingua straniera, all'asse dei linguaggi~\cite{LG2019a}\mioindex{2019}{Indirizzi!di studio}.}}
\newglossaryentry{individuazionevalidazionecompetenze}{name={Individuazione e validazione delle competenze},description={Processo che conduce al riconoscimento, da parte dell'ente titolato in base alle norme generali, ai livelli essenziali delle prestazioni e agli standard minimi di cui al presente decreto, delle competenze acquisite dalla persona in un contesto non formale o informale. Ai fini della individuazione delle competenze sono considerate anche quelle acquisite in contesti formali. La validazione delle competenze può essere seguita dalla certificazione delle competenze ovvero si conclude con il rilascio di un documento di validazione conforme agli standard minimi~\cite{DL2013}\mioindex{2013}{Competenze!individuazione e validazione}.}}
\newglossaryentry{informazione}{name={Informazione},description={Complesso delle attività dirette a fornire conoscenze utili alla identificazione, alla riduzione e alla gestione dei rischi in ambiente di lavoro\mioindex{2008}{Informazione}~\cite{DL2008a}.}}
\newglossaryentry{insegnantedisostegno}{name={Insegnante di sostegno},description={Nell'ambito di tali attività la scuola attua forme di integrazione a favore degli alunni portatori di handicaps con la prestazione di insegnanti specializzati assegnati ai sensi dell'articolo 9 del decreto del Presidente della Repubblica 31 ottobre 1975, n. 970, anche se appartenenti a ruoli speciali, o ai sensi del quarto comma dell'articolo 1 della legge 24 settembre 1971, n. 820. Devono inoltre essere assicurati la necessaria integrazione specialistica, il servizio socio-psicopedagogico e forme particolari di sostegno secondo le rispettive, competenze dello Stato e degli enti locali preposti, nei limiti delle relative disponibilità di bilancio e sulla base del programma predisposto dal consiglio scolastico distrettuale.~\cite{LEGGE1977}. Al fine di agevolare l'attuazione del diritto allo studio e la piena formazione della personalità degli alunni, la programmazione educativa può comprendere attività scolastiche di integrazione anche a carattere interdisciplinare, organizzate per gruppi di alunni della stessa classe o di classi diverse, ed iniziative di sostegno, anche allo scopo di realizzare interventi individualizzati in relazione alle esigenze dei singoli alunni.	Nell'ambito della programmazione di cui al precedente comma sono previste forme di integrazione e di sostegno a favore degli alunni portatori di handicaps da realizzare mediante la utilizzazione dei docenti, di ruolo o incaricati a tempo indeterminato, in servizio nella scuola media e in possesso di particolari titoli di specializzazione, che ne facciano richiesta, entro il limite di una unità per ciascuna classe che accolga alunni portatori di handicaps e nel numero massimo di sei ore settimanali. Le classi che accolgono alunni portatori di handicaps sono costituite con un massimo di 20 alunni. In tali classi devono essere assicurati la necessaria integrazione specialistica, il servizio socio-psico-pedagogico e forme particolari di sostegno secondo le rispettive competenze dello Stato e degli enti locali preposti, nei limiti delle relative disponibilità di bilancio e sulla base del programma predisposto dal consiglio scolastico distrettuale. Le attività di cui al primo comma del presente articolo si svolgono periodicamente in sostituzione delle normali attività didattiche e fino ad un massimo di 160 ore nel corso dell'anno scolastico con particolare riguardo al tempo iniziale e finale del periodo delle lezioni, secondo un programma di iniziative di integrazione e di sostegno che dovrà essere elaborato dal collegio dei docenti sulla base di criteri generali indicati dal consiglio di istituto e delle proposte dei consigli di classe.~\cite{LEGGE1977}\mioindex{1977}{Insegnanate!sostegno}.}}
\newglossaryentry{interessato}{name={Interessato},description={La persona fisica, la persona giuridica, l'ente o l'associazione cui si riferiscono i dati personali.\mioindex{2003}{Interessato}~\cite{DL2003}}}
\newglossaryentry{interoperabilita}{name={Interoperabilità},description={Caratteristica di un sistema informativo, le cui interfacce sono pubbliche e aperte, di interagire in maniera automatica con altri sistemi informativi per lo scambio di informazioni e l'erogazione di servizi.\mioindex{2016}{Interoperabilità}~\cite{DL2016a}}}
\newglossaryentry{interoperabilittecnica}{name={Interoperabilità tecnica},description={La capacità dei sistemi di tecnologia dell'informazione e della comunicazione di interagire in modo da consentire la condivisione di informazioni, mediante un accordo fra tutte le parti e i titolari delle informazioni~\cite{DE2018}.}}
\newglossaryentry{isfolg}{name={Isfol},description={Istituto per lo sviluppo della formazione professionale dei lavoratori - è un ente nazionale di ricerca sottoposto alla vigilanza del Ministero del Lavoro e delle politiche sociali. Dal 1//12/2016 è diventato INAPP\mioindex{2016}{INAPP}~\cite{Isfol1972}}}
\newglossaryentry{istituzionescolasticaaccreditata}{name={Istituzione scolastica accreditata},description={l'istituzione scolastica di I.P, cui é riconosciuta l'idoneità ad erogare percorsi di IeFP~\cite{DL2018b}\mioindex{2018}{Istituzione scolastica!accreditata}.}}
\newglossaryentry{istituzioniscolasticheIPgg}{name={Istituzioni scolastiche di I.P.},description={\begin{tabular}{cp{12cm}} \toprule 2018		&Le istituzioni scolastiche di IP, sono scuole territoriali dell'innovazione, aperte al territorio e concepite come laboratori di ricerca, sperimentazione e innovazione didattica. Esse definiscono i Piani triennali dell'offerta formativa secondo i principi e le finalità indicati all'articolo 1 del decreto legislativo, tenuto conto delle richieste degli studenti e delle famiglie per realizzare attività finalizzate al raggiungimento degli obiettivi formativi considerati prioritari a norma dell'articolo 1, comma 7, della legge n. 107 del 2015~\cite{DL2018a}\mioindex{2018}{Istituzione scolastica!di IP}.\\ \midrule 2018		& Istituzioni scolastiche che offrono percorsi di istruzione professionale a norma del decreto legislativo decreto legislativo 13 aprile 2017, n. 61~\cite{DL2018a}\mioindex{2018}{Istituzione scolastica!di IP}.\\\bottomrule\end{tabular}}}
\newglossaryentry{lavoratore}{name={Lavoratore},description={Persona che, indipendentemente dalla tipologia contrattuale, svolge un'attività lavorativa nell'ambito dell'organizzazione di un datore di lavoro pubblico o privato, con o senza retribuzione, anche al solo fine di apprendere un mestiere, un'arte o una professione, esclusi gli addetti ai servizi domestici e familiari. Al lavoratore così definito è equiparato l'allievo degli istituti di istruzione ed universitari e il partecipante ai corsi di formazione professionale nei quali si faccia uso di laboratori, attrezzature di lavoro in genere, agenti chimici, fisici e biologici, ivi comprese le apparecchiature fornite di videoterminali limitatamente ai periodi in cui l'allievo sia effettivamente applicato alla strumentazioni o ai laboratori in questione\mioindex{2008}{Lavoratore}~\cite{DL2008a}.}}
\newglossaryentry{lavoroagile}{name={Lavoro agile},description={La prestazione lavorativa viene eseguita, in parte all'interno di locali aziendali e in parte all'esterno senza una postazione fissa, entro i soli limiti di durata massima dell'orario di lavoro giornaliero e settimanale, derivanti dalla legge e dalla contrattazione collettiva.\mioindex{2017}{Lavoro!agile}~\cite{LEGGE2017a}},see={telelavoro}}
\newglossaryentry{limitazioneditrattamento}{name={Limitazione di trattamento},description={Il contrassegno dei dati personali conservati con l'obiettivo di limitarne il trattamento in futuro\mioindex{2016}{Trattamento!limitazione}~\cite{RE2016}~\cite{DL2018d}.}}
\newglossaryentry{lineeguida}{name={Linee guida},description={Atti di indirizzo e coordinamento per l'applicazione della normativa in materia di salute e sicurezza predisposti dai Ministeri, dalle regioni, dall'ISPESL e dall'INAIL e approvati in sede di Conferenza permanente per i rapporti tra lo Stato, le regioni e le province autonome di Trento e di Bolzano\mioindex{2008}{Linee!guida}~\cite{DL2008a}.}}
\newglossaryentry{livelloavanzato}{name={Livello avanzato},description={Lo studente svolge compiti e problemi complessi in situazioni anche non note, mostrando padronanza nell'uso delle conoscenze e delle abilità, Sa proporre e sostenere le proprie opinioni e assumere autonomamente decisioni consapevoli~\cite{DL2010}\mioindex{2010}{Livello!avanzato}.}}
\newglossaryentry{livellobase}{name={Livello base},description={Lo studente svolge compiti semplici in situazioni note, mostrando di possedere conoscenze ed abilità essenziali e di saper applicare regole e procedure fondamentali~\cite{DL2010}\mioindex{2010}{Livello!base}.}}
\newglossaryentry{livellointermedio}{name={Livello intermedio},description={Lo studente svolge compiti e risolve problemi complessi in situazioni note, compie scelte consapevoli, mostrando di saper utilizzare le conoscenze e le abilita acquisite~\cite{DL2010}\mioindex{2010}{Livello!intermedio}.}}
\newglossaryentry{logit}{name={Logit},description={\`{E} l’unità di misura utilizzata per esprimere la probabilità di un evento e corrisponde al logaritmo del rapporto tra la probabilità di fornire una risposta corretta e la probabilità di fornire una risposta errata~\cite{Desimoni2018}.}}
\newglossaryentry{malattia}{name={Malattia},description={Condizione abnorme e insolita di un organismo vivente, animale o vegetale, caratterizzata da disturbi funzionali, da alterazioni o lesioni – osservabili o presumibili, locali o generali – e, nel caso di animali a elevata organizzazione nervosa, da comportamenti inconsueti riconducibili a sofferenza psicofisica (nel caso specifico dell’uomo si considera la mutata percezione dello stato del proprio corpo, o cenestesi, che può assumere l’intensità dell’allarme da pericolo per la sopravvivenza); in senso più strettamente fisiopatologico, si intende per malattia un’alterazione transitoria e reversibile (almeno entro limiti che consentono di sopravvivere a soggetti non più capaci di guarire, quali anziani, lungodegenti, affetti da malattie terminali) concernente quei processi fisico-chimici, detti omeostatici, attraverso i quali l’organismo preserva la propria individualità in equilibrio dinamico con l’ambiente, e il cui fattore scatenante può essere occasionale, ambientale o interno all’organismo, nonché di natura fisica, chimica, organica, ereditaria o psicosomatica (la reversibilità, almeno parziale, e la transitorietà della malattia si precisano in contrapp. allo stato patologico che risulta invece irreversibile).~\cite{TRECCANI2020e}},see={salute}}
\newglossaryentry{maternage}{name={Maternage},description={Una Regione o PA sprovvista di alcune qualificazioni può attingere al Repertorio di un'altra Regione e trasferire singole qualificazioni professionali nel proprio. Il Maternage è possibile anche attingendo al bacino informatico comune (il QNQR), che raccoglie tutte le qualificazioni professionali regionali esistenti.}}
\newglossaryentry{medicocompetente}{name={Medico competente},description={Medico in possesso di uno dei titoli e dei requisiti formativi e professionali di cui all'articolo 38, che collabora, secondo quanto previsto all'articolo 29, comma 1, con il datore di lavoro ai fini della valutazione dei rischi ed è nominato dallo stesso per effettuare la sorveglianza sanitaria e per tutti gli altri compiti di cui al presente decreto\mioindex{2008}{Medico!competente}~\cite{DL2008a}.}}
\newglossaryentry{mezziidentificazioneelettronica}{name={Mezzi di identificazione elettronica},description={Un’unità materiale e/o immateriale contenente dati di identificazione personale e utilizzata per l’autenticazione per un servizio online.\mioindex{2014}{Identificazione!elettronica}~\cite{RE2014}}}
\newglossaryentry{misuredispensative}{name={Misure dispensative},description={Sono strumenti didattici e tecnologici che sostituiscono o facilitano la prestazione richiesta nell'abilità deficitaria. Tali strumenti sollevano l'alunno o lo studente con DSA da una prestazione resa difficoltosa dal disturbo, senza peraltro facilitargli il compito dal punto di vista cognitivo.L'adozione delle misure dispensative, al fine di non creare percorsi immotivatamente facilitati, che non mirano al successo formativo degli alunni e degli studenti con DSA, dovrà essere sempre valutata sulla base dell'effettiva incidenza del disturbo sulle prestazioni richieste, in modo tale, comunque, da non differenziare, in ordine agli obiettivi, il percorso di apprendimento dell'alunno o dello studente in questione.~\cite{LG2011}\mioindex{2011}{Misure!dispensative}.}}
\newglossaryentry{misureminime}{name={Misure minime},description={Il complesso delle misure tecniche, informatiche, organizzative, logistiche e procedurali di sicurezza che configurano il livello minimo di protezione richiesto in relazione ai rischi previsti nell'articolo 31. Vedi~\cref{sec:allegatoArt31DL30giugno2003n196}\mioindex{2003}{Misure!minime}~\cite{DL2003}}}
\newglossaryentry{motiviimperativiinteressegenerale}{name={Motivi imperativi di interesse generale},description={Motivi riconosciuti tali dalla giurisprudenza della Corte di giustizia dell'Unione europea~\cite{DI2013}\mioindex{2010}{Motivi imperativi!di interesse generale}.}}
\newglossaryentry{mutilatiinvalidicivili}{name={Mutilati ed invalidi civili},description={Cittadini affetti da minorazioni congenite o acquisite, anche a carattere progressivo, compresi gli irregolari psichici per oligofrenie di carattere organico o dismetabolico, insufficienze mentali derivanti da difetti sensoriali e funzionali che abbiano subito una riduzione permanente della capacità lavorativa non inferiore a un terzo o, se minori di anni 18, che abbiano difficoltà persistenti a svolgere i compiti e le funzioni proprie della loro età. Sono esclusi gli invalidi per cause di guerra, di lavoro, di servizio, nonchè i ciechi e i sordomuti per i quali provvedono altre leggi~\cite{LEGGE1971}\mioindex{1971}{Mutilati ed invalidi civili}.}}
\newglossaryentry{neurotipico}{name={Neurotipico},description={Il termine Neurotipico, oppure NT, una forma abbreviata per neurologicamente tipico, è un neologismo nato in seno alla comunità autistica per identificare le persone che non sono nello spettro autistico; è derivato dalla corrispondente parola inglese neurotypical. L'espressione ha finito per indicare chiunque non abbia una condizione neurologica atipica; in altre parole, si riferisce alle persone che non hanno disturbi dati da condizioni quali autismo, dislessia, disprassia o ADHD.~\cite{Wikipedia2019e}\mioindex{2019}{Neurotipico}.}}
\newglossaryentry{ngn}{name={Next Generation Network},description={Una Next Generation Network (NGN o "rete di prossima generazione") è una rete basata su commutazione a pacchetto in grado di fornire servizi - inclusi servizi di telecomunicazioni - ed in grado di far uso di molteplici tecnologie a banda larga con QoS, nella quale le funzionalità correlate alla fornitura dei servizi siano indipendenti dalle tecnologie di trasporto utilizzate. Offre un accesso non limitato agli utenti a diversi service provider. Supporta una mobilità generalizzata consentendo la fornitura consistente ed ubiqua di servizi agli utenti.\mioindex{2004}{Next Generation!Network}~\cite{ITU2004}}}
\newglossaryentry{nomenclaturaclassificazioneUnitprofessionaligg}{name={Nomenclatura e classificazione delle Unità professionali (NUP)},description={Strumento, adottato dall'ISTAT, per classificare e rappresentare le professioni; costituisce, a norma dell'articolo 3, comma 5, del decreto legislativo decreto legislativo 13 aprile 2017, n. 61 l'ulteriore riferimento, oltre al codice ATECO, per la declinazione degli indirizzi di studio da parte delle istituzioni scolastiche che offrono percorsi di istruzione professionale, in coerenza con le richieste del territorio secondo le priorità indicate dalle regioni nella propria programmazione e nei limiti degli spazi di flessibilità di cui all' articolo 6, comma 1, lettera b) del medesimo decreto legislativo~\cite{DL2018a}\mioindex{2018}{Nomenclatura!classificazione delle Unità professionali}.}}
\newglossaryentry{nomenclaturaunitprofessionali}{name={Nomenclatura delle unità professionali},description={Si tratta di uno strumento per ricondurre tutte le professioni esistenti nel mercato del lavoro all'interno di un numero limitato di raggruppamenti professionali. Ogni professione è definita dall'insieme di attività lavorative concretamente svolte. La logica utilizzata per aggregare professioni diverse all'interno di un medesimo raggruppamento si basa sul concetto di competenza, visto nella sua duplice dimensione del livello e del campo delle competenze richieste per l'esercizio della professione. Il livello di competenza è definito in funzione della complessità, dell'estensione dei compiti svolti, del livello di responsabilità e di autonomia decisionale che caratterizza la professione; il campo di competenza coglie, invece, le differenze nei domini settoriali, negli ambiti disciplinari delle conoscenze applicate, nelle attrezzature utilizzate, nei materiali lavorati, nel tipo di bene prodotto o servizio erogato nell'ambito della professione. La Nomenclatura delle Unità Professionali si inserisce tra gli interventi finalizzati alla definizione e alla messa a punto di un sistema nazionale di osservazione dei fabbisogni professionali~\cite{LG2019a}\mioindex{2019}{Nomenclatura!delle Unità professionali}.}}
\newglossaryentry{normevincolantiimpresa}{name={Norme vincolanti d'impresa},description={Le politiche in materia di protezione dei dati personali applicate da un titolare del trattamento o responsabile del trattamento stabilito nel territorio di uno Stato membro al trasferimento o al complesso di trasferimenti di dati personali a un titolare del trattamento o responsabile del trattamento in uno o più paesi terzi, nell'ambito di un gruppo imprenditoriale o di un gruppo di imprese che svolge un'attività economica comune\mioindex{2016}{Norme!vincolanti d'impresa}~\cite{RE2016}~\cite{DL2018d}.}}
\newglossaryentry{obiettivispecificiapprendimento}{name={Obiettivi specifici di apprendimento},description={Gli obiettivi specifici di apprendimento (OSA) indicano le conoscenze (il sapere) e le abilità (il saper fare) che tutte le scuole del territorio nazionale sono tenute ad utilizzare per progettare e organizzare autonomamente i piani di studio personalizzati che aiutino a trasformare le capacità di ciascun alunno in competenze~\cite{CIRC2005a}.}}
\newglossaryentry{obiezionepertinentemotivata}{name={Obiezione pertinente e motivata},description={Un'obiezione al progetto di decisione sul fatto che vi sia o meno una violazione del presente regolamento, oppure che l'azione prevista in relazione al titolare del trattamento o responsabile del trattamento sia conforme al presente regolamento, la quale obiezione dimostra chiaramente la rilevanza dei rischi posti dal progetto di decisione riguardo ai diritti e alle libertà fondamentali degli interessati e, ove applicabile, alla libera circolazione dei dati personali all'interno dell'Unione\mioindex{2016}{Obiezione!pertinente e motivata}~\cite{RE2016}~\cite{DL2018d}.}}
\newglossaryentry{organismiparitetici}{name={Organismi paritetici},description={Organismi costituiti a iniziativa di una o più associazioni dei datori e dei prestatori di lavoro comparativamente più rappresentative sul piano nazionale, quali sedi privilegiate per: la programmazione di attività formative e l'elaborazione e la raccolta di buone prassi a fini prevenzionistici; lo sviluppo di azioni inerenti alla salute e alla sicurezza sul lavoro; l'assistenza alle imprese finalizzata all'attuazione degli adempimenti in materia; ogni altra attività o funzione assegnata loro dalla legge o dai contratti collettivi di riferimento\mioindex{2008}{Organismi!paritetici}~\cite{DL2008a}.}}
\newglossaryentry{organismodirittopubblico}{name={Organismo di diritto pubblico},description={Un organismo definito all’articolo 2, paragrafo 1, punto 4, della direttiva 2014/24/UE del Parlamento europeo e del Consiglio.\mioindex{2014}{Organismo!di diritto pubblico}~\cite{RE2014}}}
\newglossaryentry{organismonazionaleitalianodiaccreditamento}{name={Organismo nazionale italiano di accreditamento},description={Organismo nazionale di accreditamento designato dall'Italia in attuazione del regolamento (CE) n. 765/2008 del Parlamento europeo e del Consiglio del 9 luglio 2008~\cite{DL2013}.\mioindex{2008}{Organismo!nazionale italiano di accreditamento}}}
\newglossaryentry{organismosettorepubblico}{name={Organismo del settore pubblico},description={Un’autorità statale, regionale o locale, un organismo di diritto pubblico o un’associazione formata da una o più di tali autorità o da uno o più di tali organismi di diritto pubblico, oppure un soggetto privato incaricato da almeno un’autorità, un organismo o un’associazione di cui sopra di fornire servizi pubblici, quando agisce in base a tale mandato.\mioindex{2014}{Organismo!del settore pubblico}~\cite{RE2014}}}
\newglossaryentry{organismovalutazioneconformita}{name={Organismo di valutazione della conformità},description={Un organismo ai sensi dell’articolo 2, punto 13, del regolamento (CE) n. 765/2008, che è accreditato a norma di detto regolamento come competente a effettuare la valutazione della conformità del prestatore di servizi fiduciari qualificato e dei servizi fiduciari qualificati da esso prestati.\mioindex{2014}{Organismo!di valutazione della conformità}~\cite{RE2014}},see={serviziofiduciario}}
\newglossaryentry{organizzazioneinternazionale}{name={Organizzazione internazionale},description={Un'organizzazione e gli organismi di diritto internazionale pubblico a essa subordinati o qualsiasi altro organismo istituito da o sulla base di un accordo tra due o più Stati\mioindex{2016}{Organizzazione!internazionale}~\cite{RE2016}~\cite{DL2018d}.}}
\newglossaryentry{organizzazionesettorialeinternazionale}{name={Organizzazione settoriale internazionale},description={Associazione di organizzazioni nazionali, anche, ad esempio, di datori di lavoro e organismi professionali, che rappresenta gli interessi di settori nazionali~\cite{RA2008}.\mioindex{2008}{Organizzazione!settoriale internazionale}}}
\newglossaryentry{orientamento}{name={Orientamento},description={Un processo continuativo che consente alle persone di identificare le proprie capacità, competenze e interessi attraverso una serie di attività individuali e collettive che servono a prendere decisioni in materia di istruzione, formazione e occupazione e a gestire i propri percorsi personali nell'ambito dell'istruzione, del lavoro e in altri contesti in cui è possibile acquisire o sfruttare tali capacità e competenze~\cite{DE2018}.\mioindex{2018}{Orientamento}}}
\newglossaryentry{originaliunici}{name={Originali non unici},description={I documenti per i quali sia possibile risalire al loro contenuto attraverso altre scritture o documenti di cui sia obbligatoria la conservazione, anche se in possesso di terzi.\mioindex{2005}{Originali!non unici}~\cite{DL2005c}}}
\newglossaryentry{parolachiave}{name={Parola chiave},description={Componente di una credenziale di autenticazione associata ad una persona ed a questa nota, costituita da una sequenza di caratteri o altri dati in forma elettronica.\mioindex{2003}{Parola!chiave}~\cite{DL2003}}}
\newglossaryentry{partefacenteaffidamentocertificazione}{name={Parte facente affidamento sulla certificazione},description={Un processo elettronico che consente di confermare l’identificazione elettronica di una persona fisica o giuridica, oppure l’origine e l’integrità di dati in forma elettronica.\mioindex{2014}{Parte facente affidamento!sulla certificazione}~\cite{RE2014}}}
\newglossaryentry{peig}{name={PEI},description={Il PEI di cui all'articolo 12, comma 5, della legge 5 febbraio 1992, n. 104, come modificato dal presente decreto: \begin{enumerate}\item è elaborato e approvato dai docenti contitolari o dal consiglio di classe, con la partecipazione dei genitori o dei soggetti che ne esercitano la responsabilità, delle figure professionali specifiche interne ed esterne all'istituzione scolastica che interagiscono con la classe e con la bambina o il bambino, l'alunna o l'alunno, la studentessa o lo studente con disabilità nonchè con il supporto dell'unità di valutazione multidisciplinare; \\ \item tiene conto della certificazione di disabilità e del Profilo di funzionamento; \\ \item individua strumenti, strategie e modalità per realizzare un ambiente di apprendimento nelle dimensioni della relazione, della socializzazione, della comunicazione, dell'interazione, dell'orientamento e delle autonomie; \\ \item esplicita le modalità didattiche e di valutazione in relazione alla programmazione individualizzata; \\ \item definisce gli strumenti per l'effettivo svolgimento dell'alternanza scuola-lavoro, assicurando la partecipazione dei soggetti coinvolti nel progetto di inclusione; \\ \item indica le modalità di coordinamento degli interventi ivi previsti e la loro interazione con il Progetto individuale; \\ \item è redatto all'inizio di ogni anno scolastico di riferimento, a partire dalla scuola dell'infanzia, ed è aggiornato in presenza di nuove e sopravvenute condizioni di funzionamento della persona. Nel passaggio tra i gradi di istruzione, compresi i casi di trasferimento fra scuole, è assicurata l'interlocuzione tra i docenti della scuola di provenienza e quelli della scuola di destinazione; \\ \item è soggetto a verifiche periodiche nel corso dell'anno scolastico al fine di accertare il raggiungimento degli obiettivi e apportare eventuali modifiche ed integrazioni.\end{enumerate}~\cite{DL2017c}\mioindex{2017}{PEI}.}}
\newglossaryentry{percorsiiefpigg}{name={Percorsi di IeFP},description={I percorsi di istruzione e formazione professionale per il conseguimento di qualifiche e diplomi professionali di cui al Capo III del decreto legislativo 17 ottobre 2005, n. 226~\cite{DL2018a}.\mioindex{2018}{Percorsi!IeFP}}}
\newglossaryentry{pericolo}{name={Pericolo},description={Proprietà o qualità intrinseca di un determinato fattore avente il potenziale di causare danni\mioindex{2008}{Pericolo}~\cite{DL2008a}.}}
\newglossaryentry{personafisica}{name={Persona fisica},description={Una persona fisica, in diritto, è un essere umano dotato di capacità giuridica, quindi soggetto di diritto.\mioindex{}{Persona!fisica}~\cite{Wikipedia2020b}}}
\newglossaryentry{personagiuridica}{name={Persona giuridica},description={Una persona giuridica è un soggetto di diritto.\mioindex{}{Persona!giuridica}~\cite{Wikipedia2020b}}}
\newglossaryentry{personahandicappata}{name={Persona handicappata},description={ \'{E} persona handicappata colui che presenta una minorazione fisica, psichica o sensoriale, stabilizzata o progressiva, che è causa di difficoltà di apprendimento, di relazione o di integrazione lavorativa e tale da determinare un processo di svantaggio sociale o di emarginazione~\cite{LEGGE1992}\mioindex{1992}{Persona!handicappata}.}}
\newglossaryentry{pianistudiopersonalizzati}{name={Piani di studio personalizzati},description={Il Piano di Studi personalizzato è l'insieme delle Unità di apprendimento concretamente realizzate, sia nel tempo scuola obbligatorio sia in quello opzionale facoltativo, e rappresenta il progetto realizzato dall'équipe pedagogica, in cooperazione con le famiglie e gli stessi alunni, per l'educazione di ciascuno. Ha come punto di riferimento obbligato le competenze espresse nel Profilo educativo, culturale e professionale dello studente alla fine del primo ciclo, che vengono promosse a partire dalle capacità di quegli alunni, in quel determinato contesto, modellando in obiettivi formativi gli obiettivi specifici di apprendimento elencati nelle Indicazioni nazionali. Il piano di studio è impostato nelle sue linee generali all'inizio dell'anno scolastico, tenendo conto anche di tutti gli apprendimenti non formali e informali acquisiti dagli alunni, ma si definisce riflessivamente e compiutamente solo durante e al termine delle attività realizzate~\cite{CIRC2005a}.\mioindex{2005}{Piani!studio personalizzati}}}
\newglossaryentry{pianodidatticopersonalizzato}{name={Piano Didattico Personalizzato},description={Documento in cui viene scritto quanto rilevato e cosa dovrebbe venir fatto a scuola nei confronti dell'alunno con DSA: riporta le difficoltà e le potenzialità dello studente, le azioni intraprese dai docenti, i contenuti degli incontri scuola/famiglia e i risultati raggiunti. La redazione del documento prevede una fase preparatoria di dialogo fra docenti, famiglia e clinici (se invitati e autorizzati a partecipare), nel rispetto dei reciproci ruoli e competenze~\cite{CNOP2016}\mioindex{2016}{Piano!didattico personalizzato}.}}
\newglossaryentry{piattaformaonline}{name={Piattaforma online},description={Un'applicazione basata sul web che fornisce informazioni e strumenti agli utenti finali e permette loro di portare a termine compiti specifici online~\cite{DE2018}.\mioindex{2005}{Piattaforma!online}}}
\newglossaryentry{postaelettronica}{name={Posta elettronica},description={Messaggi contenenti testi, voci, suoni o immagini trasmessi attraverso una rete pubblica di comunicazione, che possono essere archiviati in rete o nell'apparecchiatura terminale ricevente, fino a che il ricevente non ne ha preso conoscenza.\mioindex{2003}{Posta!elettronica}~\cite{DL2003}}}
\newglossaryentry{preposto}{name={Preposto},description={Persona che, in ragione delle competenze professionali e nei limiti di poteri gerarchici e funzionali adeguati alla natura dell'incarico conferitogli, sovrintende alla attività lavorativa e garantisce l'attuazione delle direttive ricevute, controllandone la corretta esecuzione da parte dei lavoratori ed esercitando un funzionale potere di iniziativa\mioindex{2008}{Preposto}~\cite{DL2008a}.}}
\newglossaryentry{prestatoreservizifiduciariqualificato}{name={Prestatore di servizi fiduciari qualificato},description={Un prestatore di servizi fiduciari che presta uno o più servizi fiduciari qualificati e cui l’organismo di vigilanza assegna la qualifica di prestatore di servizi fiduciari qualificato.\mioindex{2014}{Prestatore!di servizi fiduciari qualificato}~\cite{RE2014}},see={serviziofiduciario}}
\newglossaryentry{prestatoreservizifiduciari}{name={Prestatore di servizi fiduciari},description={Una persona fisica o giuridica che presta uno o più servizi fiduciari, o come prestatore di servizi fiduciari qualificato o come prestatore di servizi fiduciari non qualificato.\mioindex{2014}{Prestatore!di servizi fiduciari}~\cite{RE2014}},see={serviziofiduciario}}
\newglossaryentry{prevenzione}{name={Prevenzione},description={Il complesso delle disposizioni o misure necessarie anche secondo la particolarità del lavoro, l'esperienza e la tecnica, per evitare o diminuire i rischi professionali nel rispetto della salute della popolazione e dell'integrità dell'ambiente esterno\mioindex{2008}{Prevenzione}~\cite{DL2008a}.}}
\newglossaryentry{prodotto}{name={Prodotto},description={Un hardware o software o i loro componenti pertinenti, destinati a essere utilizzati per la prestazione di servizi fiduciari.\mioindex{2014}{Prodotto}~\cite{RE2014}},see={serviziofiduciario}}
\newglossaryentry{professionalitdellavoro}{name={Professionalità del lavoro},description={Risiede nell'assumere responsabilità in riferimento ad uno scopo definito e nella capacità di apprendere anche dall'esperienza, ovvero di trovare soluzioni creative ai problemi sempre nuovi che si pongono~\cite{DL2017a}.\mioindex{2017}{Professionalità!lavoro}}}
\newglossaryentry{professioneregolamentata}{name={Professione regolamentata},description={Attività, o insieme di attività professionali, l'accesso alle quali e il cui esercizio, o una delle cui modalità di esercizio, sono subordinati direttamente o indirettamente, in forza di norme legislative, regolamentari o amministrative, al possesso di determinate qualifiche professionali; in particolare costituisce una modalità di esercizio l'impiego di un titolo professionale riservato da disposizioni legislative, regolamentari o amministrative a chi possiede una specifica qualifica professionale, Quando non si applica la prima frase, è assimilata ad una professione regolamentata una professione di cui al paragrafo 2~\cite{DI2005}.\mioindex{2005}{Professione!regolamentata}}}
\newglossaryentry{profilazione}{name={Profilazione},description={Qualsiasi forma di trattamento automatizzato di dati personali consistente nell'utilizzo di tali dati personali per valutare determinati aspetti personali relativi a una persona fisica, in particolare per analizzare o prevedere aspetti riguardanti il rendimento professionale, la situazione economica, la salute, le preferenze personali, gli interessi, l'affidabilità, il comportamento, l'ubicazione o gli spostamenti di detta persona fisica\mioindex{2016}{Profilazione}~\cite{RE2016}~\cite{DL2018d}.}}
\newglossaryentry{profiloautorizzazione}{name={Profilo di autorizzazione},description={L'insieme delle informazioni, univocamente associate ad una persona, che consente di individuare a quali dati essa puo' accedere, nonchè i trattamenti ad essa consentiti.\mioindex{2003}{Profilo!di autorizzazione}~\cite{DL2003}}}
\newglossaryentry{profilofunzionamento}{name={Profilo di funzionamento},description={Il Profilo di funzionamento di cui all'articolo 12, comma 5, della legge 5 febbraio 1992, n. 104, che ricomprende la diagnosi funzionale e il profilo dinamico-funzionale, come modificato dal presente decreto, è redatto dall'unità di valutazione multidisciplinare di cui al decreto del Presidente della Repubblica 24 febbraio 1994, composta da: \begin{enumerate}\item un medico specialista o un esperto della condizione di salute della persona; \\ \item uno specialista in neuropsichiatria infantile;\\ \item un terapista della riabilitazione; \item un assistente sociale o un rappresentante dell'Ente locale di competenza che ha in carico il soggetto. \end{enumerate} Il Profilo di funzionamento di cui all'articolo 12, comma 5, della legge 5 febbraio 1992, n. 104, come modificato dal presente decreto: \begin{enumerate}\item è il documento propedeutico e necessario alla predisposizione del Progetto Individuale e del PEI; \\ \item definisce anche le competenze professionali e la tipologia delle misure di sostegno e delle risorse strutturali necessarie per l'inclusione scolastica; \\ \item è redatto con la collaborazione dei genitori della bambina o del bambino, dell'alunna o dell'alunno, della studentessa o dello studente con disabilità, nonchè con la partecipazione di un rappresentante dell'amministrazione scolastica, individuato preferibilmente tra i docenti della scuola frequentata; \\ \item è aggiornato al passaggio di ogni grado di istruzione, a partire dalla scuola dell'infanzia, nonchè in presenza di nuove e sopravvenute condizioni di funzionamento della persona.\\ \end{enumerate}~\cite{DL2017c}\mioindex{2017}{Profilo! di funzionamento}.}}
\newglossaryentry{profiloprofessionalegg}{name={Profilo professionale},description={Insieme dei contenuti «tipici» delle funzioni/mansioni di una specifica categoria di professioni omogenee rispetto a competenze, abilità, conoscenze ed attività lavorative svolte~\cite{DL2018a}.\mioindex{2018}{Profilo!professionale}}}
\newglossaryentry{profilouscitaciascunindirizzoigg}{name={Profilo di uscita di ciascun indirizzo},description={Profilo formativo inteso come standard formativo in uscita dagli indirizzi di studio, quale insieme compiuto e riconoscibile di competenze descritte secondo una prospettiva di validità e spendibilità in molteplici contesti lavorativi del settore economico-professionale correlato~\cite{DL2018a}.\mioindex{2018}{Profilo!uscita di ciascun indirizzo}}}
\newglossaryentry{progettoformativoindividuale}{name={Progetto formativo individuale},description={\begin{tabular}{cp{12cm}}\toprule 2017 & Redatto dal Consiglio di classe entro il 31 gennaio del primo anno di frequenza, In esso sono evidenziati i saperi e le competenze acquisiti dallo studente anche in modo non formale e informale, ai fini di un apprendimento personalizzato, idoneo a consentirgli di proseguire con successo, anche attraverso l'esplicitazione delle sue motivazioni allo studio, le aspettative per le scelte future, le difficoltà incontrate e le potenzialità rilevate~\cite{DL2017a}\mioindex{2017}{Progetto!formativo individuale}.\\\midrule 2018&Progetto che ha il fine di motivare e orientare la studentessa e lo studente nella progressiva costruzione del proprio percorso formativo e lavorativo, di supportarli per migliorare il successo formativo e di accompagnarli negli eventuali passaggi tra i sistemi formativi di cui all'articolo 8 del decreto legislativo 13 aprile 2017, n. 61, con l'assistenza di un tutor individuato all'interno del consiglio di classe, Il progetto formativo individuale si basa sul bilancio personale, è effettuato nel primo anno di frequenza del percorso di istruzione professionale ed è aggiornato per tutta la sua durata\mioindex{2018}{Progetto!formativo individuale}~\cite{DL2018a}. \\\bottomrule\end{tabular}}}
\newglossaryentry{progettoindividuale}{name={Progetto individuale},description={Nell'ambito delle risorse disponibili in base ai piani di cui agli articoli 18 e 19, il progetto individuale comprende, oltre alla valutazione diagnostico-funzionale, le prestazioni di cura e di riabilitazione a carico del Servizio sanitario nazionale, i servizi alla persona a cui provvede il comune in forma diretta o accreditata, con particolare riferimento al recupero e all'integrazione sociale, nonchè le misure economiche necessarie per il superamento di condizioni di povertà, emarginazione ed esclusione sociale. Nel progetto individuale sono definiti le potenzialità e gli eventuali sostegni per il nucleo familiare~\cite{LEGGE2000}\mioindex{2000}{Progetto!individuale}.}}
\newglossaryentry{provaattitudinale}{name={Prova attitudinale},description={Una verifica riguardante le conoscenze, le abilità e le competenze professionali del richiedente, effettuata o riconosciuta dalle autorità competenti dello Stato membro ospitante allo scopo di valutare l'idoneità del richiedente a esercitare in tale Stato membro una professione regolamentata. Per consentire che la verifica sia effettuata, le autorità competenti predispongono un elenco delle materie che, in base a un confronto tra la formazione e l'istruzione richiesta nello Stato membro ospitante e quella ricevuta dal richiedente, non sono coperte dal diploma o dai titoli di formazione del richiedente. La prova attitudinale deve tener conto del fatto che il richiedente è un professionista qualificato nello Stato membro d'origine o di provenienza. Essa verte su materie da scegliere tra quelle che figurano nell'elenco e la cui conoscenza è essenziale per poter esercitare la professione in questione nello Stato membro ospitante, Tale prova può altresì comprendere la conoscenza delle regole professionali applicabili alle attività in questione nello Stato membro ospitante. Le modalità dettagliate della prova attitudinale nonché lo status di cui gode, nello Stato membro ospitante, il richiedente che desidera prepararsi alla prova attitudinale in detto Stato membro sono determinate dalle autorità competenti di detto Stato membro\mioindex{2013}{Prova!attitudinale}~\cite{DI2013}.}}
\newglossaryentry{provaoggettiva}{name={Prova oggettiva},description={Un prova si dice oggettiva quando la correzione avviene secondo un protocollo stabilito a priori che rende l’esito della correzione tendenzialmente indipendente dal soggetto che la effettua~\cite{Desimoni2017}.\mioindex{2017}{Prova!oggettiva}}}
\newglossaryentry{provastandardizzata}{name={Prova standardizzata},description={Le prove standardizzate strutturalmente devono garantire a tutti i soggetti ai quali una prova è somministrata le stesse condizioni di lavoro: stessa prova e stesso tempo a disposizione~\cite{Desimoni2017}\mioindex{2017}{Prova!strutturata}.}}
\newglossaryentry{pseudonimizzazione}{name={Pseudonimizzazione},description={Il trattamento dei dati personali in modo tale che i dati personali non possano più essere attribuiti a un interessato specifico senza l'utilizzo di informazioni aggiuntive, a condizione che tali informazioni aggiuntive siano conservate separatamente e soggette a misure tecniche e organizzative intese a garantire che tali dati personali non siano attribuiti a una persona fisica identificata o identificabile\mioindex{2016}{Pseudonimizzazione}~\cite{RE2016}~\cite{DL2018d}.}}
\newglossaryentry{pubblicheamministrazionicentrali}{name={Pubbliche amministrazioni centrali},description={Le amministrazioni dello Stato, ivi compresi gli istituti e scuole di ogni ordine e grado e le istituzioni educative, le aziende ed amministrazioni dello Stato ad ordinamento autonomo, le istituzioni universitarie, gli enti pubblici non economici nazionali, l'Agenzia per la rappresentanza negoziale delle pubbliche amministrazioni (ARAN), le agenzie di cui al decreto legislativo 30 luglio 1999, n. 300.\mioindex{2005}{Pubbliche!amministrazioni centrali}~\cite{DL2005c}}}
\newglossaryentry{quadronazionalequalifiche}{name={Quadro nazionale delle qualifiche},description={Strumento di classificazione delle qualifiche in funzione di una serie di criteri basati sul raggiungimento di livelli di apprendimento specifici; esso mira a integrare e coordinare i sottosistemi nazionali delle qualifiche e a migliorare la trasparenza, l'accessibilità, la progressione e la qualità delle qualifiche rispetto al mercato del lavoro e alla società civile\mioindex{2008}{Quadro!nazionale delle qualifiche}\mioindex{2012}{Quadro!nazionale delle qualifiche}\mioindex{2017}{Quadro!nazionale delle qualifiche}~\cite{RA2017}~\cite{RA2012}~\cite{RA2008}.}}
\newglossaryentry{qualificainternazionale}{name={Qualifica internazionale},description={Qualifica, rilasciata da un organismo internazionale legalmente costituito (associazione, organizzazione, settore o impresa) o da un organismo nazionale che agisce a nome di un organismo internazionale, che è utilizzata in più di un paese e include i risultati dell'apprendimento, valutati facendo riferimento alle norme stabilite da un organismo internazionale\mioindex{2017}{Qualifica!internazionale}~\cite{RA2017}.}}
\newglossaryentry{qualificaprofessionale}{name={Qualifica professionale},description={La qualifica attestata da un titolo di formazione, un attestato di competenza e/o un'esperienza professionale\mioindex{2005}{Qualifica!professionale}~\cite{DI2005}.}}
\newglossaryentry{qualificazioneinternazionale}{name={Qualificazione internazionale},description={Qualificazione rilasciata da un organismo internazionale legalmente costituito o da un organismo nazionale che agisce a nome di un organismo internazionale, che è utilizzata in più di un Paese e include i risultati di apprendimento, valutati facendo riferimento alle norme stabilite da un organismo internazionale~\cite{DL2018}\mioindex{2018}{Qualificazione!internazionale}.}}
\newglossaryentry{qualificazione}{name={Qualificazione},description={Titolo di istruzione e di formazione, ivi compreso quello di istruzione e formazione professionale, o di qualificazione professionale rilasciato da un ente pubblico titolato nel rispetto delle norme generali, dei livelli essenziali delle prestazioni e degli standard minimi~\cite{DL2013}~\cite{DL2018a}.\mioindex{2013}{Qualificazione}\mioindex{2018}{Qualificazione}}}
\newglossaryentry{qualifica}{name={Qualifica},description={Risultato formale di un processo di valutazione e convalida, acquisito quando un'autorità competente stabilisce che una persona ha conseguito i risultati dell'apprendimento rispetto a standard predefiniti\mioindex{2017}{Qualifica}\mioindex{2018}{Qualifica}\mioindex{2012}{Qualifica}\mioindex{2018}{Qualifica}~\cite{RA2017}~\cite{DE2018}~\cite{RA2012}~\cite{RA2008}.}}
\newglossaryentry{rappresentantelavoratorisicurezza}{name={Rappresentante dei lavoratori per la sicurezza},description={Persona eletta o designata per rappresentare i lavoratori per quanto concernegli aspetti della salute e della sicurezza durante il lavoro\mioindex{2008}{Rappresentante!dei lavoratori per la sicurezza}~\cite{DL2008a}.}}
\newglossaryentry{rappresentante}{name={Rappresentante},description={La persona fisica o giuridica stabilita nell'Unione che, designata dal titolare del trattamento o dal responsabile del trattamento per iscritto ai sensi dell'articolo 27, li rappresenta per quanto riguarda gli obblighi rispettivi a norma del presente regolamento\mioindex{2016}{Rappresentante}~\cite{RE2016}~\cite{DL2018d}.}}
\newglossaryentry{readingliteracy}{name={Reading literacy},description={Competenza in lettura}}
\newglossaryentry{referenziazione}{name={Referenziazione},description={Il processo istituzionale e tecnico che associa le qualificazioni rilasciate nell'ambito del Sistema nazionale di certificazione delle competenze a uno degli otto livelli del QNQ. La referenziazione delle qualificazioni italiane al garantisce la referenziazione delle stesse al Quadro europeo delle qualifiche\mioindex{2018}{Referenziazione}~\cite{DL2018}.}}
\newglossaryentry{regimeidentificazioneelettronica}{name={Regime di identificazione elettronica},description={Un sistema di identificazione elettronica per cui si forniscono mezzi di identificazione elettronica alle persone fisiche o giuridiche, o alle persone fisiche che rappresentano persone giuridiche.\mioindex{2014}{Regime di identificazione!elettronica}~\cite{RE2014}}}
\newglossaryentry{responsabile!trattamento}{name={Responsabile del trattamento},description={La persona fisica o giuridica, l'autorità pubblica, il servizio o altro organismo che tratta dati personali per conto del titolare del trattamento\mioindex{2016}{Responsabile!del trattamento}~\cite{RE2016}~\cite{DL2018d}.}}
\newglossaryentry{responsabileservizioprevenzioneprotezione}{name={Responsabile del servizio di prevenzione e protezione},description={Persona in possesso delle capacità e dei requisiti professionali di cui all'articolo 32 designata dal datore di lavoro, a cui risponde, per coordinare il servizio di prevenzione e protezione dai rischi\mioindex{2008}{Responsabile!del servizio di prevenzione e protezione}~\cite{DL2008a}.}}
\newglossaryentry{responsabile}{name={Responsabile},description={La persona fisica, la persona giuridica, la pubblica amministrazione e qualsiasi altro ente, associazione od organismo preposti dal titolare al trattamento di dati personali.\mioindex{2003}{Responsabile}~\cite{DL2003}}}
\newglossaryentry{responsabilitaeautonomia}{name={Responsabilità e autonomia},description={Capacità del discente di applicare le conoscenze e le abilità in modo autonomo e responsabile\mioindex{2017}{Responsabilità!e autonomia}~\cite{RA2017}.}}
\newglossaryentry{responsabilitsocialeimprese}{name={Responsabilità sociale delle imprese},description={Integrazione volontaria delle preoccupazioni sociali ed ecologiche delle aziende e organizzazioni nelle loro attività commerciali e nei loro rapporti con le parti interessate\mioindex{2008}{Responsabilità!sociale delle imprese}~\cite{DL2008a}.}}
\newglossaryentry{retepubblicacomunicazioni}{name={Rete pubblica di comunicazioni},description={La connessione istituita da un servizio telefonico accessibile al pubblico, che consente la comunicazione bidirezionale in tempo reale.\mioindex{2003}{Rete!pubblica di comunicazioni}~\cite{DL2003}},see={reticomunicazioneelettronica}}
\newglossaryentry{reticomunicazioneelettronica}{name={Reti di comunicazione elettronica},description={La connessione istituita da un servizio telefonico accessibile al pubblico, che consente la comunicazione bidirezionale in tempo reale.\mioindex{2003}{Reti!di comunicazione elettronica}~\cite{DL2003}}}
\newglossaryentry{riconoscimentoformazioneprecedente}{name={Riconoscimento della formazione precedente},description={Convalida dei risultati di apprendimento, nel quadro dell'istruzione formale o dell'apprendimento non formale o informale, acquisiti prima della richiesta di convalida\mioindex{2012}{Riconoscimento!della formazione precedente}~\cite{RA2012}.}}
\newglossaryentry{riconoscimento}{name={Riconoscimento formale dei risultati dell'apprendimento},description={Processo in base al quale un'autorità competente dà valore ufficiale ai risultati dell'apprendimento acquisiti a fini di studi ulteriori o di occupazione, mediante i) il rilascio di qualifiche (certificati, diplomi o titoli), ii) la convalida dell'apprendimento non formale e informale, iii) il riconoscimento di equivalenze, il rilascio di crediti o la concessione di deroghe\mioindex{2012}{Riconoscimento!formale dei risultati dell'apprendimento}~\cite{RA2017}.}}
\newglossaryentry{rilevazioneaccertamentocompetenze}{name={Rilevazione e accertamento delle competenze},description={Accertare e certificare la competenza di una persona richiede strumenti caratterizzati da accuratezza e attendibilità che, a differenza di quelli utilizzati per valutare soltanto la padronanza delle conoscenze e delle abilità, eccedono, senza escluderle, le consuete modalità valutative scolastiche disciplinari (test, prove oggettive, interrogazioni, saggi brevi, ecc.), ma richiedono anche osservazioni sistematiche prolungate nel tempo, valutazioni collegiali dei docenti che coinvolgano anche attori esterni alla scuola, a partire dalla famiglia, autovalutazioni dell'allievo, diari, storie fotografiche e filmati, coinvolgimento di esperti e simili. Il livello di accettabilità della competenza manifestata in situazione scaturisce dalla somma di queste condivisioni e coinvolge nella maniera professionalmente più alta i docenti che si assumono la responsabilità di certificarla\mioindex{2005}{Rilevazione e accertamento!delle competenze}\mioindex{2005}{Competenze!rilevazione e accertamento}~\cite{CIRC2005a}.}}
\newglossaryentry{rischio}{name={Rischio},description={Probabilità di raggiungimento del livello potenziale di danno nelle condizioni di impiego o di esposizione ad un determinato fattore o agente oppure alla loro combinazione\mioindex{2008}{Rischio}~\cite{DL2008a}.}}
\newglossaryentry{risorseeducativeaperte}{name={Risorse educative aperte (OER)},description={Materiale digitalizzato messo gratuitamente e liberamente a disposizione di docenti, studenti, e chiunque studi in maniera autonoma, per l'uso e il riuso nell'insegnamento, l'apprendimento e la ricerca; esse comprendono materiale didattico, strumenti informatici per lo sviluppo, l'uso e la diffusione dei contenuti, e risorse per l'applicazione come le licenze aperte; le OER fanno anche riferimento a una somma di beni digitali che possono essere modificati e che offrono vantaggi senza che ne sia limitata la possibilità di utilizzo da parte di altri\mioindex{2012}{Risorse!educative aperte}~\cite{RA2012}.}}
\newglossaryentry{risultatidellapprendimento}{name={Risultati dell'apprendimento},description={\begin{tabular}{cp{12cm}}\toprule 2008-2017 & Descrizione di ciò che un discente conosce, capisce ed è in grado di realizzare al termine di un processo di apprendimento; sono definiti in termini di conoscenze, abilità e responsabilità e autonomia\mioindex{2008}{Risultati dell'apprendimento}\mioindex{2017}{Risultati dell'apprendimento}~\cite{RA2017}~\cite{RA2008}.\\\midrule 2012&Descrizione di ciò che un discente conosce, capisce ed è in grado di realizzare al termine di un processo di apprendimento definito in termini di conoscenze, abilità e competenze\mioindex{2012}{Risultati dell'apprendimento}~\cite{RA2012}. \\\bottomrule\end{tabular}}}
\newglossaryentry{saluteg}{name={Salute},description={Stato di completo benessere fisico, mentale e sociale, non consistente solo in un'assenza di malattia o d'infermità\mioindex{2008}{Salute}~\cite{DL2008a}.}}
\newglossaryentry{salute}{name={Salute},description={Stato di benessere fisico e di armonico equilibrio psichico dell’organismo umano (e analogam. negli animali, con riguardo alle condizioni fisiche), in quanto esente da malattie, da imperfezioni e disturbi organici o funzionali.~\cite{TRECCANI2020f}},see={malattia}}
\newglossaryentry{scopiscientifici}{name={Scopi scientifici},description={Le finalità di studio e di indagine sistematica finalizzata allo sviluppo delle conoscenze scientifiche in uno specifico settore.\mioindex{2003}{Scopi!scientifici}~\cite{DL2003}}}
\newglossaryentry{scopistatistici}{name={Scopi statistici},description={Le finalità di indagine statistica o di produzione di risultati statistici, anche a mezzo di sistemi informativi statistici.\mioindex{2003}{Scopi!statistici}~\cite{DL2003}}}
\newglossaryentry{scopistorici}{name={Scopi storici},description={Le finalità di studio, indagine, ricerca e documentazione di figure, fatti e circostanze del passato.\mioindex{2003}{Scopi!storici}~\cite{DL2003}}}
\newglossaryentry{segnaleavvertimento}{name={Segnale di avvertimento},description={Un segnale che avverte di un rischio o pericolo\mioindex{2008}{Segnale!di avvertimento}~\cite{DL2008a}.}}
\newglossaryentry{segnaledivieto}{name={Segnale di divieto},description={Un segnale che vieta un comportamento che potrebbe far correre o causare un pericolo\mioindex{2008}{Segnale!di divieto}~\cite{DL2008a}.}}
\newglossaryentry{segnaleprescrizione}{name={Segnale di prescrizione},description={Un segnale che avverte di un rischio o pericolo\mioindex{2008}{Segnale!di avvertimento}~\cite{DL2008a}.}}
\newglossaryentry{segnaleticasicurezza}{name={Segnaletica di sicurezza},description={Una segnaletica che, riferita ad un oggetto, ad una attività o ad una situazione determinata, fornisce una indicazione o una prescrizione concernente la sicurezza o la salute sul luogo di lavoro, e che utilizza, a seconda dei casi, un cartello, un colore, un segnale luminoso o acustico, una comunicazione verbale o un segnale gestuale\mioindex{2008}{Segnaletica!di sicurezza}~\cite{DL2008a}.}}
\newglossaryentry{servizidiautenticazione}{name={Servizi di autenticazione},description={Processi tecnici, quali le firme elettroniche e l'autenticazione di siti web, che consentono agli utenti di verificare le informazioni, come ad esempio l'identità, attraverso Europass~\cite{DE2018}.}}
\newglossaryentry{serviziocomunicazioneelettronica}{name={Servizio di comunicazione elettronica},description={I servizi consistenti esclusivamente o prevalentemente nella trasmissione di segnali su reti di comunicazioni elettroniche, compresi i servizi di telecomunicazioni e i servizi di trasmissione nelle reti utilizzate per la diffusione circolare radiotelevisiva, nei limiti previsti dall'articolo 2, lettera c), della direttiva 2002/21/CE del Parlamento europeo e del Consiglio, del 7 marzo 2002.\mioindex{2003}{Servizio!di comunicazione elettronica}~\cite{DL2003}},see={reticomunicazioneelettronica}}
\newglossaryentry{servizioelettronicorecapitocertificato}{name={Servizio elettronico di recapito certificato},description={Un servizio che consente la trasmissione di dati fra terzi per via elettronica e fornisce prove relative al trattamento dei dati trasmessi, fra cui prove dell’avvenuto invio e dell’avvenuta ricezione dei dati, e protegge i dati trasmessi dal rischio di perdita, furto, danni o di modifiche non autorizzate.\mioindex{2014}{Servizio!elettronico di recapito certificato}~\cite{RE2014}}}
\newglossaryentry{servizioelettronicorecapitoqualificatocertificato}{name={Servizio elettronico di recapito qualificato certificato},description={Un servizio elettronico di recapito certificato che soddisfa i requisiti di cui all’articolo 44. Vedi~\cref{sec:articolo44reg9102014}\mioindex{2014}{Servizio!elettronico di recapito qualificato certificato}~\cite{RE2014}}}
\newglossaryentry{serviziofiduciarioqualificato}{name={Servizio fiduciario qualificato},description={Un servizio fiduciario che soddisfa i requisiti pertinenti stabiliti nel presente regolamento.\mioindex{2014}{Servizio!fiduciario qualificato}~\cite{RE2014}}}
\newglossaryentry{serviziofiduciario}{name={Servizio fiduciario},description={Un servizio elettronico fornito normalmente dietro remunerazione e consistente nei seguenti elementi:\begin{enumerate}\item creazione, verifica e convalida di firme elettroniche, sigilli elettronici o validazioni temporali elettroniche, servizi elettronici di recapito certificato e certificati relativi a tali servizi;oppure\item creazione, verifica e convalida di certificati di autenticazione di siti web; o\item conservazione di firme, sigilli o certificati elettronici relativi a tali servizi;\end{enumerate}\mioindex{2014}{Servizio!fiduciario}~\cite{RE2014}}}
\newglossaryentry{servizioprevenzioneprotezionerischi}{name={Servizio di prevenzione e protezione dai rischi},description={Insieme delle persone, sistemi e mezzi esterni o interni all'azienda finalizzati all'attività di prevenzione e protezione dai rischi professionali per i lavoratori\mioindex{2008}{Servizio!di prevenzione e protezione dai rischi}~\cite{DL2008a}.}}
\newglossaryentry{servizioreteonline}{name={Servizio in rete o on-line},description={Qualsiasi servizio di una amministrazione pubblica fruibile a distanza per via elettronica.\mioindex{2017}{Servizio!in rete o on-line}~\cite{DL2017d}}}
\newglossaryentry{serviziosocietainformazione}{name={Servizio della società dell'informazione},description={Il servizio definito all'articolo 1, paragrafo 1, lettera b), della direttiva (UE) 2015/1535 del Parlamento europeo e del Consiglio\mioindex{2016}{Servizio!della società dell'informazione}~\cite{RE2016}~\cite{DL2018d}.}}
\newglossaryentry{serviziovaloreaggiunto}{name={Servizio a valore aggiunto},description={Il servizio che richiede il trattamento dei dati relativi al traffico o dei dati relativi all'ubicazione diversi dai dati relativi al traffico, oltre a quanto è necessario per la trasmissione di una comunicazione o della relativa fatturazione.\mioindex{2003}{Servizio!a valore aggiunto}~\cite{DL2003}}}
\newglossaryentry{settore}{name={Settore},description={Raggruppamento di attività professionali in base a funzione economica, prodotto, servizio o tecnologia principali\mioindex{2008}{Settore}~\cite{RA2008}.}}
\newglossaryentry{sigilloelettronicoavanzato}{name={Sigillo elettronico avanzato},description={Un sigillo elettronico che soddisfi i requisiti sanciti all’articolo 36.\mioindex{2014}{sigillo!elettronico avanzato}~\cite{RE2014}}}
\newglossaryentry{sigilloelettronicoqualificato}{name={Sigillo elettronico qualificato},description={Un sigillo elettronico avanzato creato da un dispositivo per la creazione di un sigillo elettronico qualificato e basato su un certificato qualificato per sigilli elettronici.\mioindex{2014}{Sigillo!elettronico qualificato}~\cite{RE2014}}}
\newglossaryentry{sigilloelettronico}{name={Sigillo elettronico},description={Dati in forma elettronica, acclusi oppure connessi tramite associazione logica ad altri dati in forma elettronica per garantire l’origine e l’integrità di questi ultimi.\mioindex{2014}{Sigillo!elettronico}~\cite{RE2014}}}
\newglossaryentry{sistemaautorizzazione}{name={Sistema di autorizzazione},description={L'insieme degli strumenti e delle procedure che abilitano l'accesso ai dati e alle modalità di trattamento degli stessi, in funzione del profilo di autorizzazione del richiedente.\mioindex{2003}{Sistema!di autorizzazione}~\cite{DL2003}}}
\newglossaryentry{sistemaeuropeoaccumulazione}{name={Sistema europeo di accumulazione e trasferimento dei crediti o crediti ECTS},description={Il sistema di crediti per l'istruzione superiore utilizzato nello Spazio europeo dell'istruzione superiore\mioindex{2013}{Sistema!europeo di accumulazione e trasferimento dei crediti o crediti ECTS}~\cite{DI2013}.}}
\newglossaryentry{sistemanazionalecertificazionecompetenze}{name={Sistema nazionale di certificazione delle competenze},description={l'insieme dei servizi di individuazione e validazione e certificazione delle competenze erogati nel rispetto delle norme generali, dei livelli essenziali delle prestazioni e degli standard minimi\mioindex{2013}{Sistema!nazionale di certificazione delle competenze}\mioindex{2018}{Sistema!nazionale di certificazione delle competenze}~\cite{DL2013}~\cite{DL2018a}.}}
\newglossaryentry{sistemanazionaledellequalifiche}{name={Sistema nazionale delle qualifiche},description={Complesso delle attività di uno Stato membro connesse con il riconoscimento dell'apprendimento e altri meccanismi che mettono in relazione istruzione e formazione al mercato del lavoro e alla società civile. Ciò comprende l'elaborazione e l'attuazione di disposizioni e processi istituzionali in materia di garanzia della qualità, valutazione e rilascio delle qualifiche, Un sistema nazionale delle qualifiche può essere composto da vari sottosistemi e può comprendere un quadro nazionale delle qualifiche\mioindex{2017}{Sistema!nazionale delle qualifiche}\mioindex{2008}{Sistema!nazionale delle qualifiche}~\cite{RA2017}~\cite{RA2008}.}}
\newglossaryentry{sistemapromozionesaluteesicurezza}{name={Sistema di promozione della salute e sicurezza},description={Complesso dei soggetti istituzionali che concorrono, con la partecipazione delle parti sociali, alla realizzazione dei programmi di intervento finalizzati a migliorare le condizioni di salute e sicurezza dei lavoratori\mioindex{2008}{Sistema!di promozione della salute e sicurezza}~\cite{DL2008a}.}}
\newglossaryentry{sistemidicrediti}{name={Sistemi di crediti},description={Strumenti di trasparenza volti ad agevolare il riconoscimento dei crediti, Tali sistemi possono comprendere tra l'altro equivalenze, esenzioni, possibilità di accumulare e trasferire unità/moduli, autonomia degli erogatori che possono personalizzare i percorsi nonché convalida dell'apprendimento non formale e informale\mioindex{2017}{Sistemi!crediti}~\cite{RA2017}.}}
\newglossaryentry{softskill}{name={Softskill},description={Competenze trasversali e trasferibili attraverso la dimensione operativa del fare: Capacità di interagire e lavorare con gli altri, capacità di risoluzione di problemi, creatività, pensiero critico, consapevolezza, resilienza e capacità di individuare le forme di orientamento e sostegno disponibili per affrontare la complessità e l'incertezza dei cambiamenti, preparandosi alla natura mutante delle economie moderne e delle società complesse\mioindex{2019}{Softskill}~\cite{LG2019}}}
\newglossaryentry{sorveglianzasanitaria}{name={Sorveglianza sanitaria},description={Insieme degli atti medici, finalizzati alla tutela dello stato di salute e sicurezza dei lavoratori, in relazione all'ambiente di lavoro, ai fattori di rischio professionali e alle modalità di svolgimento dell'attività lavorativa\mioindex{2008}{Sorveglianza! sanitaria}~\cite{DL2008a}.}}
\newglossaryentry{stakeholder}{name={Stakeholder},description={Tutti i soggetti, individui od organizzazioni, attivamente coinvolti in un'iniziativa economica (progetto, azienda), il cui interesse è negativamente o positivamente influenzato dal risultato dell'esecuzione, o dall'andamento, dell'iniziativa e la cui azione o reazione a sua volta influenza le fasi o il completamento di un progetto o il destino di un'organizzazione\mioindex{2019}{Stakeholder}~\cite{Wikipedia2019}}}
\newglossaryentry{standardaperti}{name={Standard aperti},description={Standard tecnici che sono stati elaborati nell'ambito di un processo collaborativo e sono stati pubblicati per essere utilizzati liberamente da tutti i soggetti interessati; ~\cite{DE2018}.}}
\newglossaryentry{standardformativoregionale}{name={Standard formativo regionale},description={Regolamentazione regionale in materia di IeFP che, nel rispetto dei livelli essenziali delle prestazioni di cui al capo III del decreto legislativo n. 226 del 2005, definisce in particolare: a) la durata, l'articolazione e gli obiettivi dei percorsi di IeFP; b) le modalità per l'effettuazione delle prove finali di accertamento degli allievi e di certificazione finale e intermedia delle competenze acquisite anche in contesti non formali e informali, nonchè di riconoscimento dei crediti, spendibili nel sistema di istruzione, formazione e lavoro; c) la modulazione temporale tra attività formativa e alternanza scuola lavoro nonchè dell'apprendistato ai sensi dell'art. 43 del decreto legislativo n. 81 del 2015\mioindex{2018}{Standard!formativo regionale}~\cite{DL2018b}.}}
\newglossaryentry{standardminimiprocesso}{name={Standard minimi di processo},description={Con riferimento al processo di individuazione e validazione e alla procedura di certificazione, l'ente pubblico titolare assicura quali standard minimi:\begin{enumerate}\item l'articolazione nelle seguenti fasi: \begin{enumerate}\item identificazione: fase finalizzata a individuare e mettere in trasparenza le competenze della persona riconducibili a una o più qualificazioni; in caso di apprendimenti non formali e informali questa fase implica un supporto alla persona nell'analisi e documentazione dell'esperienza di apprendimento e nel correlarne gli esiti a una o più qualificazioni;\item valutazione: fase finalizzata all'accertamento del possesso delle competenze riconducibili a una o più qualificazioni; nel caso di apprendimenti non formali e informali questa fase implica l'adozione di specifiche metodologie valutative e di riscontri e prove idonei a comprovare le competenze effettivamente possedute;\item attestazione: fase finalizzata al rilascio di documenti di validazione o certificati, standardizzati ai sensi del presente decreto, che documentano le competenze individuate e validate o certificate riconducibili a una o più qualificazioni;\end{enumerate}\item l'adozione di misure personalizzate di informazione e orientamento in favore dei destinatari dei servizi di individuazione e validazione e certificazione delle competenze.\end{enumerate}\mioindex{2013}{Standard!minimi di processo}~\cite{DL2013}.}}
\newglossaryentry{standardminimiservizio}{name={Standard minimi di servizio},description={Gli standard minimi di servizio costituiscono livelli essenziali delle prestazioni da garantirsi su tutto il territorio nazionale, anche in riferimento alla individuazione e validazione degli apprendimenti non formali e informali e al riconoscimento dei crediti formativi, Gli standard minimi di servizio costituiscono riferimento per gli enti pubblici titolari nella definizione di standard minimi di erogazione dei servizi da parte degli enti titolati\mioindex{2013}{Standard!minimi di servizio}~\cite{DL2013}.}}
\newglossaryentry{strumenticompensativi}{name={Strumenti compensativi},description={Sono strumenti didattici e tecnologici che sostituiscono o facilitano la prestazione richiesta nell'abilità deficitaria. Tali strumenti sollevano l'alunno o lo studente con DSA da una prestazione resa difficoltosa dal disturbo, senza peraltro facilitargli il compito dal punto di vista cognitivo.~\cite{LG2011}\mioindex{2011}{Strumenti!compensativi}.}}
\newglossaryentry{strumentielettronici}{name={Strumenti elettronici},description={Gli elaboratori, i programmi per elaboratori e qualunque dispositivo elettronico o comunque automatizzato con cui si effettua il trattamento.\mioindex{2003}{Strumenti!elettronici}~\cite{DL2003}}}
\newglossaryentry{supplementi Europass}{name={Supplementi Europass},description={Una serie di documenti, come ad esempio i supplementi al diploma e i supplementi al certificato, rilasciati dalle autorità od organismi competenti~\cite{DE2018}.}}
\newglossaryentry{supplementoaldiploma}{name={Supplemento al diploma},description={Un documento allegato a un diploma di istruzione superiore rilasciato dalle autorità od organismi competenti allo scopo di facilitare la comprensione da parte di terzi — soprattutto in un altro paese — dei risultati di apprendimento ottenuti dal titolare della qualifica come pure della natura, del livello, del contesto, del contenuto e dello status dell'istruzione e della formazione completata e delle competenze acquisite\mioindex{2018}{Supplemento!al diploma}~\cite{DE2018}.}}
\newglossaryentry{supplementocertificato}{name={Supplemento al certificato},description={Un documento accluso a un certificato di istruzione e formazione professionale o a un certificato professionale rilasciato dalle autorità od organismi competenti allo scopo di facilitare la comprensione da parte di terzi — soprattutto in un altro paese — dei risultati di apprendimento ottenuti dal titolare della qualifica come pure della natura, del livello, del contesto, del contenuto e dello status dell'istruzione e della formazione completata e delle competenze acquisite\mioindex{2018}{Supplemento!al certificato}~\cite{DE2018}.}}
\newglossaryentry{teachingtest}{name={Teaching to the test},description={Insegnamento basato al superamento di un test.}}
\newglossaryentry{telelavoro}{name={Telelavoro},description={La prestazione di lavoro eseguita dal dipendente di una delle amministrazioni pubbliche di cui all'articolo 1, comma 2, del decreto legislativo 3 febbraio 1993, n. 29, in qualsiasi luogo ritenuto idoneo, collocato al di fuori della sede di lavoro, dove la prestazione sia tecnicamente possibile, con il prevalente supporto di tecnologie dell'informazione e della comunicazione, che consentano il collegamento con l'amministrazione cui la prestazione stessa inerisce.\mioindex{1999}{Telelavoro}~\cite{DL1999a}},see={lavoroagile}}
\newglossaryentry{terzo}{name={Terzo},description={La persona fisica o giuridica, l'autorità pubblica, il servizio o altro organismo che non sia l'interessato, il titolare del trattamento, il responsabile del trattamento e le persone autorizzate al trattamento dei dati personali sotto l'autorità diretta del titolare o del responsabile\mioindex{2016}{Terzo}~\cite{RE2016}~\cite{DL2018d}.}}
\newglossaryentry{tesseraprofessionaleeuropea}{name={Tessera professionale europea},description={Un certificato elettronico attestante o che il professionista ha soddisfatto tutte le condizioni necessarie per fornire servizi, su base temporanea e occasionale, in uno Stato membro ospitante o il riconoscimento delle qualifiche professionali ai fini dello stabilimento in uno Stato membro ospitante\mioindex{2013}{Tessera!professionale europea}~\cite{DI2013}.}}
\newglossaryentry{tirociniodiadattamento}{name={Tirocinio di adattamento},description={l'esercizio di una professione regolamentata nello Stato membro ospitante sotto la responsabilità di un professionista qualificato, accompagnato eventualmente da una formazione complementare, Il tirocinio è oggetto di una valutazione. Le modalità del tirocinio di adattamento e della sua valutazione nonché lo status di tirocinante migrante sono determinati dalle autorità competenti dello Stato membro ospitante. Lo status di cui il tirocinante gode nello Stato membro ospitante, soprattutto in materia di diritto di soggiorno nonché di obblighi, diritti e benefici sociali, indennità e retribuzione, è stabilito dalle autorità competenti di detto Stato membro conformemente al diritto comunitario applicabile\mioindex{2005}{Tirocinio!adattamento}~\cite{DI2005}.}}
\newglossaryentry{tirocinioprofessionale}{name={Tirocinio professionale},description={Fatto salvo l'articolo 46, paragrafo 4, un periodo di pratica professionale effettuato sotto supervisione, purché costituisca una condizione per l'accesso a una professione regolamentata e che può svolgersi durante o dopo il completamento di un'istruzione che conduce a un diploma\mioindex{2013}{Tirocinio!professionale}~\cite{DI2013}.}}
\newglossaryentry{titolaredato}{name={Titolare del dato},description={Uno dei soggetti di cui all'articolo 2, comma 2, che ha originariamente formato per uso proprio o commissionato ad altro soggetto il documento che rappresenta il dato, o che ne ha la disponibilità.\mioindex{2016}{Titolare!del dato}~\cite{DL2016a}}}
\newglossaryentry{titolaren}{name={Titolare},description={La persona fisica, la persona giuridica, la pubblica amministrazione e qualsiasi altro ente, associazione od organismo cui competono, anche unitamente ad altro titolare, le decisioni in ordine alle finalità, alle modalità del trattamento di dati personali e agli strumenti utilizzati, ivi compreso il profilo della sicurezza.\mioindex{2003}{Titolare}~\cite{DL2003}}}
\newglossaryentry{titolaretrattamento}{name={Titolare del trattamento},description={La persona fisica o giuridica, l'autorità pubblica, il servizio o altro organismo che, singolarmente o insieme ad altri, determina le finalità e i mezzi del trattamento di dati personali; quando le finalità e i mezzi di tale trattamento sono determinati dal diritto dell'Unione o degli Stati membri, il titolare del trattamento o i criteri specifici applicabili alla sua designazione possono essere stabiliti dal diritto dell'Unione o degli Stati membri\mioindex{2016}{Titolare!del trattamento}~\cite{RE2016}~\cite{DL2018d}.}}
\newglossaryentry{titolare}{name={Titolare},description={La persona fisica cui è attribuita la firma elettronica e che ha accesso ai dispositivi per la creazione della firma elettronica.\mioindex{2005}{Titolare}~\cite{DL2005c}}}
\newglossaryentry{titoloformazione}{name={Titolo di formazione},description={Diplomi, certificati e altri titoli rilasciati da un'autorità di uno Stato membro designata ai sensi delle disposizioni legislative, regolamentari e amministrative di tale Stato membro e che sanciscono una formazione professionale acquisita in maniera preponderante nella Comunità~\cite{DI2005}.}}
\newglossaryentry{traghettamento}{name={Traghettamento},description={Con apposito Protocollo una Regione o PA può trasferire un intero Repertorio di qualificazioni professionali e di standard di certificazione di un'altra Regione e traghettarlo nella propria\mioindex{}{Traghettamento}.}}
\newglossaryentry{trasferimentodicrediti}{name={Trasferimento di crediti},description={Processo che consente ai soggetti che hanno accumulato crediti in un contesto di farli valutare e riconoscere in un altro contesto\mioindex{2017}{Trasferimento!crediti}~\cite{RA2017}.}}
\newglossaryentry{trattamenton}{name={Trattamento},description={Qualunque operazione o complesso di operazioni, effettuati anche senza l'ausilio di strumenti elettronici, concernenti la raccolta, la registrazione, l'organizzazione, la conservazione, la consultazione, l'elaborazione, la modificazione, la selezione, l'estrazione, il raffronto, l'utilizzo, l'interconnessione, il blocco, la comunicazione, la diffusione, la cancellazione e la distruzione di dati, anche se non registrati in una banca di dati.\mioindex{2003}{Trattamento}~\cite{DL2003}}}
\newglossaryentry{trattamento}{name={Trattamento},description={Qualsiasi operazione o insieme di operazioni, compiute con o senza l'ausilio di processi automatizzati e applicate a dati personali o insiemi di dati personali, come la raccolta, la registrazione, l'organizzazione, la strutturazione, la conservazione, l'adattamento o la modifica, l'estrazione, la consultazione, l'uso, la comunicazione mediante trasmissione, diffusione o qualsiasi altra forma di messa a disposizione, il raffronto o l'interconnessione, la limitazione, la cancellazione o la distruzione\mioindex{2018}{Trattamento}\mioindex{2016}{Trattamento}~\cite{RE2016}~\cite{DL2018d}.}}
\newglossaryentry{unitapprendimento}{name={Unità di apprendimento},description={\begin{tabular}{cp{12cm}}\toprule 2005 & Dopo aver identificato l'apprendimento unitario da promuovere (ad esempio, un campo unitario e significativo di esperienze e di possibile competenza, problemi da risolvere, compiti da eseguire o progetti da realizzare, ecc.), l'unità di apprendimento precisa gli obiettivi formativi coinvolti, gli itinerari educativi e didattici ritenuti necessari per raggiungerli e i compiti unitari in situazione che, osservati e analizzati, possono alla fine documentare il perseguimento degli obiettivi formativi posti. L'unità di apprendimento sottintende il principio che l'unico insegnamento efficace è quello che si trasforma in apprendimento degli allievi, e che ogni apprendimento significativo non è mai parziale o segmentato, ma sempre unitario, nel senso che sollecita tutte le dimensioni della persona e coinvolge più prospettive disciplinari\mioindex{2005}{Unità!di apprendimento}~\cite{CIRC2005a}.\\\midrule 2018&Insieme autonomamente significativo di competenze, abilità e conoscenze in cui è organizzato il percorso formativo della studentessa e dello studente; costituisce il necessario riferimento per la valutazione, la certificazione e il riconoscimento dei crediti, soprattutto nel caso di passaggi ad altri percorsi di istruzione e formazione. Le UdA partono da obiettivi formativi adatti e significativi, sviluppano appositi percorsi di metodo e di contenuto, tramite i quali si valuta il livello delle conoscenze e delle abilità acquisite e la misura in cui la studentessa e lo studente hanno maturato le competenze attese\mioindex{2018}{Unità!di apprendimento}~\cite{DL2018a}. \\\bottomrule\end{tabular}}}
\newglossaryentry{unitproduttiva}{name={Unità produttiva},description={Stabilimento o struttura finalizzati alla produzione di beni o all'erogazione di servizi, dotati di autonomia finanziaria e tecnico funzionale\mioindex{2008}{Unità!produttiva}~\cite{DL2008a}.}}
\newglossaryentry{utente}{name={Utente},description={Qualsiasi persona fisica che utilizza un servizio di comunicazione elettronica accessibile al pubblico, per motivi privati o commerciali, senza esservi necessariamente abbonata.\mioindex{2003}{Utente}~\cite{DL2003}}}
\newglossaryentry{validazionetemporaleelettronicaqualificata}{name={Validazione temporale elettronica qualificata},description={Una validazione temporale elettronica che soddisfa i requisiti di cui all’articolo 42. Vedi~\cref{sec:articolo42reg9102014}\mioindex{2014}{Validazione!temporale elettronica qualificata}~\cite{RE2014}}}
\newglossaryentry{validazionetemporaleelettronica}{name={Validazione temporale elettronica},description={Dati in forma elettronica che collegano altri dati in forma elettronica a una particolare ora e data, così da provare che questi ultimi esistevano in quel momento.\mioindex{2014}{Validazione!temporale elettronica}~\cite{RE2014}}}
\newglossaryentry{validazionetemporale}{name={Validazione temporale},description={Il risultato della procedura informatica con cui si attribuiscono, ad uno o più documenti informatici, una data ed un orario opponibili ai terzi.\mioindex{2005}{Validazione!temporale}~\cite{DL2005c}}}
\newglossaryentry{valutazionecomportamento}{name={Valutazione del comportamento},description={La valutazione del comportamento si riferisce allo sviluppo delle competenze di cittadinanza. Lo Statuto delle studentesse e degli studenti, il Patto educativo di corresponsabilità e i regolamenti approvati dalle istituzioni scolastiche ne costituiscono i riferimenti essenziali~\cite{DL2017}\mioindex{2017}{Valutazione!del comportamento}.}}
\newglossaryentry{valutazionedellecompetenze}{name={Valutazione delle competenze},description={Il processo o il metodo utilizzato per valutare, misurare e infine descrivere, mediante l'autovalutazione o la valutazione certificata da terzi o entrambe, le competenze individuali acquisite in contesti formali, non formali o informali\mioindex{2018}{Valutazione!delle competenze}~\cite{DE2018}.}}
\newglossaryentry{valutazioneg}{name={Valutazione},description={La valutazione ha per oggetto il processo formativo e i risultati di apprendimento delle alunne e degli alunni, delle studentesse e degli studenti delle istituzioni scolastiche del sistema nazionale di istruzione e formazione, ha finalità formativa ed educativa e concorre al miglioramento degli apprendimenti e al successo formativo degli stessi, documenta lo sviluppo dell'identità personale e promuove la autovalutazione di ciascuno in relazione alle acquisizioni di conoscenze, abilità e competenze~\cite{DL2017}\mioindex{2017}{Valutazione}.}}
\newglossaryentry{valutazionerischi}{name={Valutazione dei rischi},description={Valutazione globale e documentata di tutti i rischi per la salute e sicurezza dei lavoratori presenti nell'ambito dell'organizzazione in cui essi prestano la propria attività, finalizzata ad individuare le adeguate misure di prevenzione e di protezione e ad elaborare il programma delle misure atte a garantire il miglioramento nel tempo dei livelli di salute e sicurezza\mioindex{2008}{Valutazione!dei rischi}~\cite{DL2008a}.}}
\newglossaryentry{violazionedatipersonali}{name={Violazione dei dati personali},description={La violazione di sicurezza che comporta accidentalmente o in modo illecito la distruzione, la perdita, la modifica, la divulgazione non autorizzata o l'accesso ai dati personali trasmessi, conservati o comunque trattati\mioindex{2016}{Violazione!dei dati personali}~\cite{RE2016}~\cite{DL2018d}.}}
\newglossaryentry{comunicazionen}{name={Comunicazione},description={Il dare conoscenza dei dati personali a uno o più   soggetti    determinati    diversi    dall'interessato,    dal rappresentante del titolare nel territorio dell'Unione  europea,  dal responsabile o dal  suo  rappresentante  nel  territorio  dell'Unione europea,  dalle   persone   autorizzate,   ai   sensi   dell'articolo 2-quaterdecies, al trattamento dei dati personali  sotto  l'autorità diretta del titolare o del responsabile, in  qualunque  forma,  anche mediante la loro  messa  a  disposizione,  consultazione  o  mediante interconnessione. Vedi~\cref{sec:DL10agosto2018n101Art2quaterdecies}\mioindex{2018}{Comunicazione}~\cite{DL2018e}},see={diffusionen}}  
\newglossaryentry{diffusionen}{name={Diffusione},description={il dare conoscenza dei dati personali a soggetti indeterminati, in qualunque forma, anche mediante  la  loro  messa  a disposizione o consultazione.\mioindex{2018}{Diffusione}~\cite{DL2018e}},see={comunicazionen}}  

\newglossaryentry{culpavigilando}{name={Culpa in vigilando},description={Art. 2048 Codice civile.\begin{itemize}\item Il padre e la madre, o il tutore sono responsabili del danno cagionato dal fatto illecito dei figli minori non emancipati o delle persone soggette alla tutela, che abitano con essi. La stessa disposizione si applica all'affiliante. \item I precettori e coloro che insegnano un mestiere o un'arte sono responsabili del danno cagionato dal fatto illecito dei loro allievi e apprendisti  nel tempo in cui sono sotto la loro vigilanza.\item Le persone indicate dai commi precedenti sono liberate dalla responsabilità soltanto se provano di non aver potuto impedire il fatto.\end{itemize}}}


